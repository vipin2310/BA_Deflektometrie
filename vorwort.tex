Die vorliegende Bachelorarbeit entstand während meiner Tätigkeit als Bachelorand bei der NeuroCheck GmbH.
Zuvor war ich bei der NeuroCheck GmbH seit Beginn meines Bachelor-Studiums als Werkstudent angestellt.
Die Werkstudentenanstellung fand im Rahmen des kooperativen Mathematikstudiengangs (Bachelor Mathematik - Studienvariante \glqq Mathe$^2$\grqq) der Hochschule für Technik Stuttgart statt.
Die NeuroCheck GmbH kooperierte als eines der ersten Unternehmen mit der Hochschule für Technik Stuttgart im Rahmen der Studienvariante.
Dabei war es mir möglich, das Erlernte aus der Hochschule stetig in der Praxis zu vertiefen.
Die Erfahrungen und Kenntnisse, die ich in der NeuroCheck GmbH sammeln konnte, haben mir auch im Studium an vielen Stellen sehr weitergeholfen.
Mit diesem praktischen Hintergrund hatte ich eine sehr schöne und erfolgreiche Studienzeit im Bachelor-Studiengang Mathematik an der Hochschule für Technik Stuttgart.

\p
An dieser Stelle möchte ich all jenen danken, die mich im Rahmen dieser Bacherlorarbeit begleitet haben.
Mein ganz besonderer Dank gilt meinem Betreuer der Hochschule für Technik Stuttgart, Herrn Prof. Dr.-Ing. Uwe Müßigmann, der mich mit stetiger Anteilnahme und konstruktiver Kritik unterstützte.
Meinem Betreuer der NeuroCheck GmbH, Herrn Dipl.-Ing. Sebastian Lichtenberg, gilt mein herzlicher Dank für die vertiefenden Diskussionen über das Thema und sein Interesse an dieser Arbeit.
Außerdem bedanke ich mich für die freundliche Unterstützung meiner Kollegen der NeuroCheck GmbH, die mir jeder Zeit mit ihrem Rat zur Seite standen.
Zuletzt geht mein Dank an meine Eltern, die mich stets im Laufe meines Studiums entlastet haben.