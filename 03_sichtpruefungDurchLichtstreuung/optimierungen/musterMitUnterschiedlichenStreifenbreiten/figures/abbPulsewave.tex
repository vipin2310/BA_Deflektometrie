\begin{adjustbox}{width=\textwidth}
	\begin{tikzpicture}
	
		% Koordinatensystem
		\draw[thick,-stealth,black] (0,0)--(16,0) node[below] {$x$};
		\draw[thick,-stealth,black] (0,0)--(0,4.25) node[left] {$m_1(x,y_0)$};
		\draw[thick,black] (0,0) -- (0,-0.1) node[anchor=north,fill=white] {$0$};
		\draw[thick,black] (15,0) -- (15,-0.1) node[anchor=north,fill=white] {\acrshort{lwidth}};
		\draw[thick,black] (0,0) -- (-0.1,0) node[anchor=east,fill=white] {$0$};
		\draw[thick,black] (0,3.75) -- (-0.1,3.75) node[anchor=east,fill=white] {$255$};

		% Funktion, erste Periode
		\draw[blue, thick] (0,3.75) -- (1,3.75);
		\draw[blue, thick] (1,3.75) -- (1,0);
		\draw[blue, thick] (1,0) -- (3.75,0);
		\draw[blue, thick] (3.75,0) -- (3.75,3.75);
		
		% Funktion, zweite Periode
		\draw[blue, thick] (3.75,3.75) -- (4.75,3.75);
		\draw[blue, thick] (4.75,3.75) -- (4.75,0);
		\draw[blue, thick] (4.75,0) -- (7.5,0);
		\draw[blue, thick] (7.5,0) -- (7.5,3.75);
		
		% Funktion, dritte Periode
		\draw[blue, thick] (7.5,3.75) -- (8.5,3.75);
		\draw[blue, thick] (8.5,3.75) -- (8.5,0);
		\draw[blue, thick] (8.5,0) -- (11.25,0);
		\draw[blue, thick] (11.25,0) -- (11.25,3.75);
		
		% Funktion, vierte Periode
		\draw[blue, thick] (11.25,3.75) -- (12.25,3.75);
		\draw[blue, thick] (12.25,3.75) -- (12.25,0);
		\draw[blue, thick] (12.25,0) -- (15,0);

		\end{tikzpicture}
\end{adjustbox}
\caption[Grauwerteverlauf in Form einer Impulsschwingung bzw. \textit{engl: pulse wave, rectangular wave}]{Impulsschwingung, \textit{engl: pulse wave, rectangular wave}, eines Streifenmusters mit Ausbreitungsrichtung in $x$ bei fester, aber beliebiger Zeile $y_0$.}