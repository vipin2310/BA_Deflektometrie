\begin{adjustbox}{width=\textwidth}
	\begin{tikzpicture}
	
		% Koordinatensystem
		\draw[thick,-stealth,black] (0,0)--(16,0) node[below] {$x$};
		\draw[thick,-stealth,black] (0,0)--(0,4.25) node[left] {$m_1(x,y_0)$};
		\draw[thick,black] (0,0) -- (0,-0.1) node[anchor=north,fill=white] {$0$};
		\draw[thick,black] (15,0) -- (15,-0.1) node[anchor=north,fill=white] {\acrshort{lwidth}};
		\draw[thick,black] (0,0) -- (-0.1,0) node[anchor=east,fill=white] {$0$};
		\draw[thick,black] (0,3.75) -- (-0.1,3.75) node[anchor=east,fill=white] {$255$};

		% Funktion, erste Periode
		\draw[blue, thick] (0,3.75) -- (1.5,3.75);
		\draw[blue, thick] (1.5,3.75) -- (1.5,0);
		\draw[blue, thick] (1.5,0) -- (3,0);
		\draw[blue, thick] (3,0) -- (3,3.75);
		
		% Funktion, zweite Periode
		\draw[blue, thick] (3,3.75) -- (4.5,3.75);
		\draw[blue, thick] (4.5,3.75) -- (4.5,0);
		\draw[blue, thick] (4.5,0) -- (6,0);
		\draw[blue, thick] (6,0) -- (6,3.75);
		
		% Funktion, dritte Periode
		\draw[blue, thick] (6,3.75) -- (7.5,3.75);
		\draw[blue, thick] (7.5,3.75) -- (7.5,0);
		\draw[blue, thick] (7.5,0) -- (9,0);
		\draw[blue, thick] (9,0) -- (9,3.75);
		
		% Funktion, vierte Periode
		\draw[blue, thick] (9,3.75) -- (10.5,3.75);
		\draw[blue, thick] (10.5,3.75) -- (10.5,0);
		\draw[blue, thick] (10.5,0) -- (12,0);
		\draw[blue, thick] (12,0) -- (12,3.75);
		
		% Funktion, fünfte Periode
		\draw[blue, thick] (12,3.75) -- (13.5,3.75);
		\draw[blue, thick] (13.5,3.75) -- (13.5,0);
		\draw[blue, thick] (13.5,0) -- (15,0);

		\end{tikzpicture}
\end{adjustbox}
\caption[Rechteckschwingung bzw. \textit{eng: square wave}]{Rechteckschwingung, \textit{eng: square wave}, eines Streifenmusters mit Ausbreitungsrichtung in $x$ bei fester, aber beliebiger Zeile $y_0$.}