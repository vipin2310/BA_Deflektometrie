Die folgende Abbildung \ref{img:pruefaufbau} zeigt eine Skizze des verwendeten Prüfaufbaus.
Zur Vereinfachung wird die Skizze ohne Halterungen erstellt.

\begin{figure}[H]
	\centering
	\includegraphics[width=0.5\textwidth]{03_sichtpruefungDurchLichtstreuung/pruefaufbau/figures/aufbau}
	\caption[Prüfaufbau]{Prüfaufbau (Abbildung nicht maßstabsgetreu)}
	\label{img:pruefaufbau}
\end{figure}

\noindent
Die Parameter des Prüfaufbaus wie z. B. Kameraeinstellungen und Entfernungen lassen sich aus Abbildung \ref{img:pruefaufbau} entnehmen.
Es gilt zu beachten, dass die Entfernung zur Kamera nicht am Objektiv, sondern an der Sensorebene gemessen wird.
Der Grund liegt darin, dass Objektive unterschiedliche Größen haben, weshalb die Entfernung andernfalls für verschiedene Objektive unterschiedlich sein würde.
Die Sensorebene einer Kamera ist die Position des Kamerasensors, an der das einfallende Licht aufgenommen wird.
Zur Erzeugung des Bildmaterials wird ein Monitor als Durchlichtbeleuchtung verwendet, um die speziellen Eigenschaften der Lichtstreuung an den Oberflächenbeschädigungen zunutze zu machen.
Auf diese Eigenschaften wird im Kapitel \ref{sec:verfahren} weiter eingegangen.
Die Objekte zwischen der Kamera und der Beleuchtung sind dabei transparente Brillengläser mit Oberflächenbeschädigungen.
Die Oberflächenbeschädigungen umfassen Eingravierungen, Kratzer und andere mögliche Fehlstellen.