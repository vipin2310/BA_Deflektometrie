Dieses Kapitel legt den Schwerpunkt auf die Erkennung von kleineren Oberflächendefekten auf transparenten Prüfobjekten, wie z. B. Kratzer.
Dabei sollen insbesondere Beschädi\-gungen von Brillengläsern kenntlich gemacht werden.

%Prüfaufbau
{
	\FloatBarrier
    \section{Prüfaufbau}
    \label{sec:pruefaufbau}
    Die folgende Abbildung \ref{img:pruefaufbau} zeigt eine Skizze des verwendeten Prüfaufbaus.
Zur Vereinfachung wird die Skizze ohne Halterungen erstellt.

\begin{figure}[H]
	\centering
	\includegraphics[width=0.5\textwidth]{03_sichtpruefungDurchLichtstreuung/pruefaufbau/figures/aufbau}
	\caption[Prüfaufbau]{Prüfaufbau (Abbildung nicht maßstabsgetreu)}
	\label{img:pruefaufbau}
\end{figure}

\noindent
Die Parameter des Prüfaufbaus wie z. B. Kameraeinstellungen und Entfernungen lassen sich aus Abbildung \ref{img:pruefaufbau} entnehmen.
Es gilt zu beachten, dass die Entfernung zur Kamera nicht am Objektiv, sondern an der Sensorebene gemessen wird.
Der Grund liegt darin, dass Objektive unterschiedliche Größen haben, weshalb die Entfernung andernfalls für verschiedene Objektive unterschiedlich sein würde.
Die Sensorebene einer Kamera ist die Position des Kamerasensors, an der das einfallende Licht aufgenommen wird.
Zur Erzeugung des Bildmaterials wird ein Monitor als Durchlichtbeleuchtung verwendet, um die speziellen Eigenschaften der Lichtstreuung an den Oberflächenbeschädigungen zunutze zu machen.
Auf diese Eigenschaften wird im Kapitel \ref{sec:verfahren} weiter eingegangen.
Die Objekte zwischen der Kamera und der Beleuchtung sind dabei transparente Brillengläser mit Oberflächenbeschädigungen.
Die Oberflächenbeschädigungen umfassen Eingravierungen, Kratzer und andere mögliche Fehlstellen.
}

%Verfahren
{
	\FloatBarrier
    \section{Verfahren}
    \label{sec:verfahren}
    Es soll ein Verfahren beschrieben werden, dass es ermöglicht Kratzer und ähnliche Defekte, insbesondere auf transparenten Objekten, mittels Methoden aus der Deflektometrie sichtbar zu machen.
Der Ansatz zur Erkennung basiert auf der Idee aus Abbildung \ref{img:scratch}.
Dabei nutzt man die abweichende Lichtstreuung an Kratzern und anderen Oberflächenbeschädigungen im Vergleich zur idealen Oberfläche des Objekts.
Das beschriebene Verfahren lässt sich mit Anpassung des Versuchsaufbaus auch auf spiegelnde Oberflächen anwenden.
Dabei gilt zu beachten, dass Spiegelbilder anstelle von Bildern der Durchlichtprojektionen ausgewertet werden.

\begin{figure}[H]
	\centering
	\includegraphics[width=\textwidth]{03_sichtpruefungDurchLichtstreuung/verfahren/figures/scratch_reflection_with_images}
	\caption[Lichtbrechung an einem Kratzer]{Querschnitt eines Brillenglases mit Lichtbrechung an einem Kratzer. Die blauen Stellen entsprechen den hellen Streifen, die schwarzen Stellen den dunklen Streifen auf dem Monitor. (Abbildung nicht maßstabsgetreu)}
	\label{img:lightreflection}
\end{figure}

\noindent
In Abbildung \ref{img:lightreflection} wird schematisch die Überlegung hinter dem Ansatz dargestellt.
Man nimmt ein Streifenmuster und projiziert dieses auf ein spiegelndes Prüfobjekt.
Für ein transparentes Prüfobjekt kann man das Streifenmuster als Durchlichtbeleuchtung von unten projizieren.
An den Hell-Dunkelübergängen fällt Licht vom hellen Streifen in den Kratzer.
Durch den Kratzer werden manche Lichtstrahlen so gestreut, dass diese an der Stelle des dunklen Streifens in den Kamerasensor gelangen (siehe roten Lichtstrahl in Abbildung \ref{img:lightreflection}).
Man erkennt im Kamerabild eine lokale Fehlstelle, da der Kratzer heller ist als der umliegende dunkle Streifen.
Analog dazu erkennt man im hellen Streifen lokal eine etwas dunklere Stelle.
Durch Anpassung der Kameraeinstellungen kann man beeinflussen, wie stark man den Kratzer sieht.
Z. B. kann dies durch die Erhöhung der Belichtungszeit oder weitere Öffnung der Blende geschehen.
Dadurch wird ein Oberflächendefekt im dunklen Streifen zwar besser und stärker sichtbar, allerdings ist es möglich die Informationen über den Defekt im hellen Streifen zu verlieren.
Dies liegt daran, dass auch die dunklere Stelle im hellen Streifen so hell werden kann, dass sie nicht mehr von dem hellen Streifen selbst zu unterscheiden ist.
Dieses Problem erkennt man in der Abbildung \ref{img:scratches}.

\begin{figure}[H]
	\centering
	\includegraphics[width=\textwidth]{03_sichtpruefungDurchLichtstreuung/verfahren/figures/visibleScratch}
	\caption[Kratzer]{Kratzer an Hell-Dunkel-Übergang. Links mit weniger weit geöffneten Blende im Vergleich zu rechts.}
	\label{img:scratches}
\end{figure}

\noindent
Trotz der fehlenden Information hat das rechte Bild den Vorteil, dass durch die größere Differenz zwischen dem Oberflächendefekt und dem Hintergrund eine bessere Erkennung möglich ist.
Je ausgeprägter der Oberflächendefekt in der Oberfläche ist, desto mehr Licht wird an der Stelle gestreut und dieser wird besser erkennbar.

\begin{figure}[H]
	\centering
	\includegraphics[width=\textwidth]{03_sichtpruefungDurchLichtstreuung/verfahren/figures/minorScratch}
	\caption[Eingravierung im Glas]{Schlecht erkennbare Eingravierung im Glas, nach Verschiebung des Streifenmusters.}
	\label{img:engraving}
\end{figure}

\noindent
In Abbildung \ref{img:engraving} stellt man fest, dass kleine Defekte der Oberflächenstruktur, wie hier z. B. die Eingravierung, nur zum Teil und besonders in der Nähe der Übergänge zu erkennen sind.

\p
Daraus lassen sich bestimmte Folgerungen ziehen.
Zunächst decken solche Streifenmuster nur unmittelbar an den Übergängen zuverlässig Defekte auf.
Das bedeutet, um Defekte an bestimmten Stellen zu erfassen, muss das verwendete Streifenmuster an den Stellen Übergänge haben.
Das bedeutet auch, dass Muster mit schmaleren Streifen, aufgrund weiterer Übergänge, besser geeignet sind, um auch kleinere Oberflächendefekte sichtbar zu machen.
Allerdings führt dies auch dazu, dass stets nur kleine Teile der Defekte zu erkennen sind.
Als Lösung dieses Problems kann man mehrere Streifenmuster verwenden, deren Streifen stets in ihrer Ausbreitungsrichtung verschoben sind.
Verknüpft man die sichtbaren Teile der Defekte, kann man in einem vollständigen Gesamtbild, alle Oberflächendefekte ab einer bestimmten Mindesttiefe sichtbar machen.
Die Mindesttiefe hängt dabei von den Kameraeinstellungen, der Beleuchtungsstärke und den verwendeten Streifenmustern ab.
}

%Einsatz von mehreren Streifenmustern
{
	\FloatBarrier
    \section{Einsatz von mehreren Streifenmustern}
    \label{sec:einsatzVonMehrerenStreifenmustern}
    Die verwendeten Streifenmuster haben entlang ihrer Ausbreitungsrichtung den Grauwerteverlauf einer Rechteckschwingung.

\noindent
Die periodische Einheitsrechteckschwingung $s_f(t)$ sei definiert durch (vgl. \cite{squareWave}):
%
\begin{equation} \label{eq:einheitsrechteckschwingung}
	s_f(t) := \sign \left( \sin \left(2 \pi f t \right) \right)
\end{equation}
%
$\sign(x)$ bezeichnet hierbei die Vorzeichenfunktion mit:
%
\begin{equation*}
	\sign(x) := 
		\begin{cases}
	      -1 & \text{für $x < 0$}\\
	      0 & \text{für $x = 0$}\\
	      1 & \text{für $x > 0$}
	    \end{cases} 
\end{equation*}
%
Die Periodendauer $T$ der Einheitsrechteckschwingung $s_f(t)$ steht über den Kehrwert im Zusammenhang mit der Frequenz $f$:
\begin{equation*}
	T = \dfrac{1}{f}
\end{equation*}
Für $f = \tfrac{1}{3}$ und $f = \tfrac{1}{4}$ sieht das Schaubild der Funktion aus wie in Abbildung \ref{tikz:abbRechteckschwingung}:
% Abbildung: Rechteckschwingung
{
	\begin{figure}[H]
		\centering
		\begin{adjustbox}{width=\textwidth}
	\begin{tikzpicture}
	
		% Koordinatensystem
		\draw[thick,-stealth,black] (0,0)--(16,0) node[below] {$x$};
		\draw[thick,-stealth,black] (0,0)--(0,4.25) node[left] {$m_1(x,y_0)$};
		\draw[thick,black] (0,0) -- (0,-0.1) node[anchor=north,fill=white] {$0$};
		\draw[thick,black] (15,0) -- (15,-0.1) node[anchor=north,fill=white] {\acrshort{lwidth}};
		\draw[thick,black] (0,0) -- (-0.1,0) node[anchor=east,fill=white] {$0$};
		\draw[thick,black] (0,3.75) -- (-0.1,3.75) node[anchor=east,fill=white] {$255$};

		% Funktion, erste Periode
		\draw[blue, thick] (0,3.75) -- (1.5,3.75);
		\draw[blue, thick] (1.5,3.75) -- (1.5,0);
		\draw[blue, thick] (1.5,0) -- (3,0);
		\draw[blue, thick] (3,0) -- (3,3.75);
		
		% Funktion, zweite Periode
		\draw[blue, thick] (3,3.75) -- (4.5,3.75);
		\draw[blue, thick] (4.5,3.75) -- (4.5,0);
		\draw[blue, thick] (4.5,0) -- (6,0);
		\draw[blue, thick] (6,0) -- (6,3.75);
		
		% Funktion, dritte Periode
		\draw[blue, thick] (6,3.75) -- (7.5,3.75);
		\draw[blue, thick] (7.5,3.75) -- (7.5,0);
		\draw[blue, thick] (7.5,0) -- (9,0);
		\draw[blue, thick] (9,0) -- (9,3.75);
		
		% Funktion, vierte Periode
		\draw[blue, thick] (9,3.75) -- (10.5,3.75);
		\draw[blue, thick] (10.5,3.75) -- (10.5,0);
		\draw[blue, thick] (10.5,0) -- (12,0);
		\draw[blue, thick] (12,0) -- (12,3.75);
		
		% Funktion, fünfte Periode
		\draw[blue, thick] (12,3.75) -- (13.5,3.75);
		\draw[blue, thick] (13.5,3.75) -- (13.5,0);
		\draw[blue, thick] (13.5,0) -- (15,0);

		\end{tikzpicture}
\end{adjustbox}
\caption[Rechteckschwingung bzw. \textit{eng: square wave}]{Rechteckschwingung, \textit{eng: square wave}, eines Streifenmusters mit Ausbreitungsrichtung in $x$ bei fester, aber beliebiger Zeile $y_0$.}
		\label{tikz:abbRechteckschwingung}
	\end{figure}
}

\noindent
Mithilfe der Einheitsrechteckschwingung aus Gleichung \ref{eq:einheitsrechteckschwingung} lässt sich ein Streifenmuster mit Ausbreitungsrichtung in $x$ ausdrücken durch:
\begin{equation} \label{eq:rstreifenmuster}
	\begin{gathered}
		m_k(x,y) = A_m 
		\left(
			1 + s_f \left(x - \dfrac{1}{2\pi f} \psi_k \right)
		\right),\\
		f = \dfrac{N_p}{\acrshortmath{lwidth}},
		\quad
		\psi_k = (k - 1)\dfrac{2\pi}{N_{shift}},
		\quad
		k \in \lbrace 1,\ldots,N_{shift}\rbrace 
	\end{gathered}
\end{equation}
%
Das Streifenmuster aus Gleichung \ref{eq:rstreifenmuster} ist durch die Periodizität der Einheitsrechteckschwingung auch periodisch zur Ausbreitungsrichtung.
$A_m$ bezeichnet die Amplitude, $f$ die Frequenz, $N_p$ die Anzahl der Perioden über die Monitorbreite \acrshort{lwidth}, $N_{shift}$ die Anzahl der Phasenverschiebungen und $\psi_k$ die Phasenverschiebung des $k$-ten Musters.
Analog zu Gleichung \ref{eq:rstreifenmuster} lassen sich auch Streifenmuster mit Ausbreitungsrichtung in $y$ über die Monitorhöhe \acrshort{lheight} schreiben.
Das Bild eines vertikalen Streifenmusters, d. h. mit Ausbreitungsrichtung in $x$, nach Gleichung \ref{eq:rstreifenmuster} wird in Abbildung \ref{img:rechteckStreifenmuster} dargestellt.
%
\begin{figure}[H]
	\centering
	\includegraphics[frame,width=0.5\textwidth]{03_sichtpruefungDurchLichtstreuung/einsatzVonMehrerenStreifenmustern/figures/rechteckStreifenmuster}
	\caption[Rechteckförmiges Streifenmuster]{Streifenmuster nach Gleichung \ref{eq:rstreifenmuster}, mit $A_m = 127.5$, $N_p = 6$, $\acrshortmath{lwidth} = 384$ und $\psi_0 = 0$. Die Breite der Streifen betragen jeweils 32 Pixel.}
	\label{img:rechteckStreifenmuster}
\end{figure}

\noindent
Die verschiedenen Streifenmuster $m_k$ können nach Gleichung \ref{eq:rstreifenmuster} als zueinander phasenverschoben bezeichnet werden.
Die Phasenverschiebung $\psi_k$ wird durch einen Phasenwinkel im Bogenmaß angegeben.
Eine Phasenverschiebung von $ \pi $ bedeutet dementsprechend eine Phasenverschiebung um eine halbe Periode des Musters.
Anschaulich stellt man fest, dass für gleich breite helle und dunkle Streifen diese ihre Positionen tauschen.
Dies kann man sich zunutze machen, denn das bedeutet, dass die Schnittmenge der dunklen Streifen in den beiden Streifenmustern am kleinsten ist.
Da bestimmte Fehlstellen entweder in den dunklen oder in den weißen Streifen deutlich zu erkennen sind, ergänzen sich die beiden Streifenmuster durch die sichtbaren Fehlstellen.
Verknüpft man die Kamerabilder von solchen Mustern, dann kann man damit die meiste Information aus zwei Bildern extrahieren.
Durch zusätzliche Bildaufnahmen mit verschobenen Streifenmustern kann man detailliertere Oberflächeninformationen von dem Prüfobjekt gewinnen.

\begin{figure}[H]
	\centering
	\includegraphics[width=\textwidth]{03_sichtpruefungDurchLichtstreuung/einsatzVonMehrerenStreifenmustern/figures/imageToLink}
	\caption[Zu verknüpfende Bilder]{Kameraaufnahme eines Prüfobjekts unter Projektion von Streifenmustern mit einer Phasenverschiebung von $ \pi $ zueinander.}
	\label{img:imageToLink}
\end{figure}

\noindent
Wie man erkennt, sind die Streifen der beiden Bilder genau zueinander versetzt.
Die Auffälligkeiten in den Bildern z. B. Eingravierungen sind oft entweder im dunklen oder im hellen Streifen zu erkennen.
Durch den Unterschied zum Streifenhintergrund erfasst man gewisse Oberflächeninformation des Prüfobjekts.
Das heißt, dass die beiden Bilder sich durch den Versatz in ihrer Oberflächeninformation ergänzen.
Zur Verknüpfung der Information in einem Gesamtbild überlegt man sich, wie bestimmte Defekte in den beiden Bildern aussehen.
Dabei fällt auf, dass sich Defekt- bzw. Fehlstellen in zwei Fällen abzeichnen (vgl. Abbildung \ref{img:imageToLink}).

\p
\textbf{Fall 1: z. B. Kratzer}
\nopagebreak
\par\medskip\noindent
\begin{tabular}{@{} p{0.438888889\textwidth} c p{0.438888889\textwidth} @{}}
$Muster ~m_1$ &  & $Muster ~m_2$ \\ 
	Helle Fragmente in dunklen Streifen & $ \longleftrightarrow $ & Helle Fragmente in hellen Streifen \\ 
\end{tabular}

\p
\textbf{Fall 2: z. B. Partikel}
\nopagebreak
\par\medskip\noindent
\begin{tabular}{@{} p{0.438888889\textwidth} c p{0.438888889\textwidth} @{}}
	$Muster ~m_1$ &  & $Muster ~m_2$ \\ 
	Dunkle Fragmente in hellen Streifen & $ \longleftrightarrow $ & Dunkle Fragmente in dunklen Streifen \\ 
\end{tabular}

\p
Das Muster $m_1$ und das Muster $m_2$ sind Streifenmuster, die zueinander um $ \pi $ phasenverschoben sind.
Für eine Verknüpfung von Bildern errechnet man ein neues Bild, indem man zwei Bilder punktweise zusammen verrechnet.
Das bedeutet, um für das Ergebnisbild den Bildpunkt an der Stelle $ (x,y) $ zu berechnen, nimmt man sich die beiden Bildpunkte der Eingangsbilder an derselben Stelle $ (x,y) $.
Daraus folgt auch, dass die zu verrechnenden Bilder dieselbe Größe haben müssen.
Diese Bedingung ist hier durch dieselben Kameraeinstellungen gegeben.


\p
Unter Berücksichtigung dieser beiden Fälle soll man eine Verknüpfung für diese Bilder aufstellen, sodass die Oberflächendefekte und Fehlstellen hervorgehoben werden.
Um die Fehlstellen vom Typ \textit{Fall 1} zu erkennen, reicht es aus, für alle Bildpunkte zu untersuchen, ob einer der beiden Bildpunkte dunkel ist.
Ist das erfüllt, dann wird der Bildpunkt zum Hintergrund hinzugefügt.
Dies kann man erreichen, indem man punktweise das Minimum der Bilder bestimmt.
Dadurch würden nur Defekte von \textit{Fall 1} hell sein und die restlichen Bildpunkte dunkel.
Da \textit{Fall 2} genau umgekehrt zu \textit{Fall 1} ist, kann man analog vorgehen, um die Defekte von \textit{Fall 2} zu erkennen.
Das heißt, dass punktweise das Maximum der Bilder bestimmt wird.
Alle Bildpunkte, die nicht in beiden Bildern dunkel sind, werden damit hell.
In Abbildung \ref{img:minAndMaxLink} sollen diese Verknüpfungen dargestellt werden.

\begin{figure}[H]
	\centering
	\includegraphics[width=\textwidth]{03_sichtpruefungDurchLichtstreuung/einsatzVonMehrerenStreifenmustern/figures/minAndMaxLink}
	\caption[Verknüpfte Bilder über Minimierung und Maximierung]{Verknüpfte Bilder, um Defekte von \textit{Fall 1} (rot umrahmt) und Defekte von \textit{Fall 2} (grün umrahmt) isoliert voneinander zu betrachten. Links über Minimierung und rechts über Maximierung verknüpft. Die verknüpften Quellbilder sind in Abbildung \ref{img:imageToLink} einzusehen.}
	\label{img:minAndMaxLink}
\end{figure}

\noindent
In den Bildern aus Abbildung \ref{img:minAndMaxLink} sind noch horizontale Streifen zu erkennen.
Diese sind keine Defekte, sondern entstehen aus Überlappungen der Streifenmuster in den Kamerabildern.
Auf diese \glqq Fehler\grqq ~und Möglichkeiten zur Beseitigung dieser wird im nächsten Abschnitt \ref{sec:optimierungen} ~eingegangen.

\p
Als Nächstes sollen beide Fälle in einem Gesamtbild kenntlich gemacht werden.
Hierfür macht man sich die Gemeinsamkeiten von \textit{Fall 1} und \textit{Fall 2} zunutze.
Man kann feststellen, dass die Helligkeit der Defekte in beiden Kamerabildern trotz Veränderung der Muster ungefähr gleich bleibt.
Verknüpft man die beiden Bilder durch die punktweise betragsmäßige Differenz, werden Defekte aus den beiden Fällen dunkel.
Die restliche, normal-spiegelnde Oberfläche wird hell, da jeder sonstige Bildpunkt in einem Muster dunkel und im anderen Muster hell erscheinen sollte, also eine hohe Differenz ergibt.
Die Ausnahme bilden dabei auch hier die Überlappungen von Streifen (siehe Abbildung \ref{img:diffImage}).

\begin{figure}[H]
	\centering
	\includegraphics[width=0.5\textwidth]{03_sichtpruefungDurchLichtstreuung/einsatzVonMehrerenStreifenmustern/figures/diffImage}
	\caption[Verknüpfte Bilder über Differenz]{Über betragsmäßige Differenz verknüpfte Bilder.}
	\label{img:diffImage}
\end{figure}
}

%Optimierungen
{
	\FloatBarrier
    \section{Optimierungen}
    \label{sec:optimierungen}
    Nach dem Verknüpfen der Bilder verbleiben neben den Defekten noch schmale horizontale Streifen (vgl. Abbildung \ref{img:minAndMaxLink} und \ref{img:diffImage}).
Wenn man die beiden Kamerabilder mit phasenverschobenen Streifenmustern übereinanderlegt, bilden sich Überlappungen der Streifen.
Diese können durch eine Reihe von Ungenauigkeiten im gesamten Prozess entstehen.
Zum besseren Verständnis muss auf die Differenzen zwischen dem aufgenommenen Kamerabild und der erzeugten Muster auf dem Bildschirm eingegangen werden.

% Unterschiede zwischen Kameraufnahme und Monitorbild
{
	\FloatBarrier
    \subsection{Unterschiede zwischen Kameraufnahme und Monitorbild}
    \label{sub:unterschiedeKameraUndMonitor}
    Das Kameraobjektiv hat bestimmte Einstellungsmöglichkeiten, darunter die Blende und der Fokus.
Aufgrund der festen Brennweite des verwendeten Objektivs wird der Einfluss der Brennweite nicht genauer betrachtet.
Die entscheidenden Einstellungen für diesen Prozess sind also der Fokus und die Blende.
Über die Fokussteuerung kann eine einzelne Tiefenebene im Bild scharf gestellt werden.
Da das Prüfobjekt und das Streifenmuster in unterschiedlichen Tiefenebenen liegen, führt das bereits dazu, dass im Kamerabild nicht beides gleichzeitig fokussiert werden kann.
Der Fokus liegt zur Prüfung auf der Oberfläche des Objekts.
Dadurch wird das Streifenmuster unscharf, wodurch die Breiten der hellen und dunklen Streifen verändert werden.
Die zweite Einstellungsmöglichkeit ist die Blende.
Öffnet man diese weiter, lässt man mehr Licht in den Kamerasensor.
Durch mehr einfallendes Licht vergrößern sich die Breiten der hellen Streifen im Bild.
Oberflächendefekte des Prüfobjekts werden gleichzeitig besser sichtbar.
Zur Kompensation der unterschiedlichen Streifenbreiten im Kamerabild müssen die Breiten der hellen und dunklen Streifen im erzeugten Muster unterschiedlich gewählt werden (siehe Abbildung \ref{img:differenceCamPat}).

\begin{figure}[H]
	\centering
	\includegraphics[width=\textwidth]{03_sichtpruefungDurchLichtstreuung/optimierungen/unterschiedeKameraUndMonitor/figures/differenceCameraPattern}
	\caption[Unterschied zwischen Muster und Kameraaufnahme]{Unterschied zwischen Muster und Kameraaufnahme. Links erzeugtes Muster mit fünf Pixel Breite der hellen und neun Pixel Breite der dunklen Streifen, rechts Kameraaufnahme}
	\label{img:differenceCamPat}
\end{figure}

\noindent
Man muss also mit Streifenmustern arbeiten, die von der Gleichung \ref{eq:rstreifenmuster} abweichen und unterschiedliche Breiten für die hellen und dunklen Streifen haben.
Es kommt dazu, dass die hellen und dunklen Streifen im Kamerabild unter Umständen nicht exakt gleich breit sein können, da die Anpassung der Streifenbreiten lediglich auf pixelgenauer Ebene durchgeführt werden kann.
Außerdem gibt es auch bei der Phase der Streifenmuster die Beschränkung, dass die maximale Genauigkeit der Verschiebung auch ein Pixel beträgt.
Das bedeutet, dass die Streifen selbst bei exakt gleicher Breite unter dem Kamerabild nicht genau versetzt zueinander liegen.
Es kommt hinzu, dass diese Genauigkeit sich auf den projizierenden Bildschirm bezieht.
Durch das Brillenglas zwischen der Kamera und dem Bildschirm kann die Phasenverschiebung im Kamerabild also mit zusätzlichen Fehlern behaftet sein.
Außerdem ist zu beachten, dass in der Kameraaufnahme stets ein Rauschen die Szene überlagert.
}

% Muster mit unterschiedlichen Breiten Streifenbreiten
{
	\FloatBarrier
    \subsection{Muster mit unterschiedlichen Streifenbreiten}
    \label{sub:musterUnterschiedlichenStreifenbreiten}
    Das erzeugte Streifenmuster in Abbildung \ref{img:differenceCamPat} hat nicht mehr den Grauwerteverlauf einer Rechteckschwingung (vgl. Abbildung \ref{tikz:abbRechteckschwingung}) entlang der Ausbreitungsrichtung.
Dadurch lässt sich das Streifenmuster nicht mehr in der Form aus Gleichung \ref{eq:rstreifenmuster} darstellen.
Der Grauwerteverlauf in einer Zeile bzw. Spalte eines solchen Streifenmusters entspricht einer Impulsschwingung (siehe Abbildung \ref{tikz:abbPulsewave}).
%
% Abbildung: Pulswave
{
	\begin{figure}[H]
		\centering
		\begin{adjustbox}{width=\textwidth}
	\begin{tikzpicture}
	
		% Koordinatensystem
		\draw[thick,-stealth,black] (0,0)--(16,0) node[below] {$x$};
		\draw[thick,-stealth,black] (0,0)--(0,4.25) node[left] {$m_1(x,y_0)$};
		\draw[thick,black] (0,0) -- (0,-0.1) node[anchor=north,fill=white] {$0$};
		\draw[thick,black] (15,0) -- (15,-0.1) node[anchor=north,fill=white] {\acrshort{lwidth}};
		\draw[thick,black] (0,0) -- (-0.1,0) node[anchor=east,fill=white] {$0$};
		\draw[thick,black] (0,3.75) -- (-0.1,3.75) node[anchor=east,fill=white] {$255$};

		% Funktion, erste Periode
		\draw[blue, thick] (0,3.75) -- (1,3.75);
		\draw[blue, thick] (1,3.75) -- (1,0);
		\draw[blue, thick] (1,0) -- (3.75,0);
		\draw[blue, thick] (3.75,0) -- (3.75,3.75);
		
		% Funktion, zweite Periode
		\draw[blue, thick] (3.75,3.75) -- (4.75,3.75);
		\draw[blue, thick] (4.75,3.75) -- (4.75,0);
		\draw[blue, thick] (4.75,0) -- (7.5,0);
		\draw[blue, thick] (7.5,0) -- (7.5,3.75);
		
		% Funktion, dritte Periode
		\draw[blue, thick] (7.5,3.75) -- (8.5,3.75);
		\draw[blue, thick] (8.5,3.75) -- (8.5,0);
		\draw[blue, thick] (8.5,0) -- (11.25,0);
		\draw[blue, thick] (11.25,0) -- (11.25,3.75);
		
		% Funktion, vierte Periode
		\draw[blue, thick] (11.25,3.75) -- (12.25,3.75);
		\draw[blue, thick] (12.25,3.75) -- (12.25,0);
		\draw[blue, thick] (12.25,0) -- (15,0);

		\end{tikzpicture}
\end{adjustbox}
\caption[Pulsschwingung bzw. \textit{eng: pulse wave, rectangular wave}]{Pulsschwingung, \textit{eng: pulse wave, rectangular wave}, eines Streifenmusters mit Ausbreitungsrichtung in $x$ bei fester, aber beliebiger Zeile $y_0$.}
		\label{tikz:abbPulsewave}
	\end{figure}
}
%
\noindent
Um eine Gleichung für eine solche Impulsschwingung (siehe Abbildung \ref{tikz:abbPulsewave}) herzuleiten, kann man die periodische Sägezahnschwingung verwenden \cite{waveGeneration}.
Eine Sägezahnschwingung lässt sich darstellen durch (vgl. \cite{sawtoothWave}):
%
\begin{equation} \label{eq:saegezahnschwingung}
	w_f(t) = 2 \left( ft - \left\lfloor ft \right\rfloor \right) - 1
\end{equation}
%
\noindent
$f$ bezeichnet die Periodenlänge der Sägezahnschwingung und steht analog zu Gleichung \ref{eq:einheitsrechteckschwingung} über den Kehrwert im Zusammenhang mit der Periodenlänge der Sägezahnschwingung $w_f(t)$:
%
\begin{equation*}
	f = \dfrac{1}{T}
\end{equation*}
%
Mit $T = 3$ erhält man folgendes Schaubild (siehe Abbildung \ref{tikz:abbsaegezahnSchwingung}).
%
% Abbildung: Sägezahnschwingung
{
	\begin{figure}[H]
		\centering
		\input{03_sichtpruefungDurchLichtstreuung/optimierungen/musterMitUnterschiedlichenStreifenbreiten/figures/abbSaegezahnSchwingung}
		\label{tikz:abbsaegezahnSchwingung}
	\end{figure}
}
%
\noindent
Bildet man die Differenz von zwei versetzten Sägezahnfunktionen, erhält man eine Impulsschwingung:
%
\begin{equation} \label{eq:pulsewave}
	p_{f,D}(t) = w_f(t - D \cdot T) - w_f(t) - w_f(-D \cdot T),
	\quad
	D \in \left(0,1\right] \subset \acrshortmath{real}
\end{equation}
%
\noindent
$D$ wird Tastgrad (\textit{engl: duty cycle}) genannt und bezeichnet die Impulsdauer der Impulsschwingung im Verhältnis zur Periodenlänge $T$.
Mit $D = \tfrac{1}{2}$ erhält man eine Rechteckschwingung wie in Abbildung \ref{tikz:abbRechteckschwingung}.
Das Schaubild der Gleichung \ref{eq:pulsewave} wird in Abbildung \ref{tikz:abbNormalPulsewave} dargestellt.
%
% Abbildung: Normale Impulsschwingung
{
	\begin{figure}[H]
		\centering
		\begin{adjustbox}{width=\textwidth}
	\begin{tikzpicture}[declare function={sawtooth(\x) = 1.5*(2*((\x/3)-0.5-floor((\x/3))));}]
	
		% Koordinatensystem
		\draw[thick,-stealth,black] (-8,0)--(8,0) node[below] {$t$};
		\draw[thick,-stealth,black] (0,-2.125)--(0,2.125) node[left] {$p_{\tfrac{1}{3},\tfrac{1}{3}}(t)$};
		
		% Funktion
		\draw[name path = func1, blue, thick, domain=-8:8, samples=600] plot (\x, {sawtooth(\x - (1/3) * 3) - sawtooth(\x) - sawtooth(-(1/3) * 3)});
		
		% Achsenbeschriftungen
		\draw[thick,black] (0,-1.5) -- (-0.1,-1.5) node[anchor=east,fill=white] {$-1$};
		\draw[thick,black] (0,1.5) -- (-0.1,1.5) node[anchor=east,fill=white] {$1$};
		\draw[thick,black] (-7,0) -- (-7,-0.1) node[anchor=north,fill=white] {$-7$};
		\draw[thick,black] (-6,0) -- (-6,-0.1) node[anchor=north,fill=white] {$-6$};
		\draw[thick,black] (-5,0) -- (-5,-0.1) node[anchor=north,fill=white] {$-5$};
		\draw[thick,black] (-4,0) -- (-4,-0.1) node[anchor=north,fill=white] {$-4$};
		\draw[thick,black] (-3,0) -- (-3,-0.1) node[anchor=north,fill=white] {$-3$};
		\draw[thick,black] (-2,0) -- (-2,-0.1) node[anchor=north,fill=white] {$-2$};
		\draw[thick,black] (-1,0) -- (-1,-0.1) node[anchor=north,fill=white] {$-1$};
		\draw[thick,black] (1,0) -- (1,-0.1) node[anchor=north,fill=white] {$1$};
		\draw[thick,black] (2,0) -- (2,-0.1) node[anchor=north,fill=white] {$2$};
		\draw[thick,black] (3,0) -- (3,-0.1) node[anchor=north,fill=white] {$3$};
		\draw[thick,black] (4,0) -- (4,-0.1) node[anchor=north,fill=white] {$4$};
		\draw[thick,black] (5,0) -- (5,-0.1) node[anchor=north,fill=white] {$5$};
		\draw[thick,black] (6,0) -- (6,-0.1) node[anchor=north,fill=white] {$6$};
		\draw[thick,black] (7,0) -- (7,-0.1) node[anchor=north,fill=white] {$7$};	
		
	\end{tikzpicture}
\end{adjustbox}
\caption[Einheits-Impulsschwingung]{Einheits-Impulsschwingung nach Gleichung \ref{eq:pulsewave} mit $f = \tfrac{1}{3}$ und $D = \tfrac{1}{3}$.}
		\label{tikz:abbNormalPulsewave}
	\end{figure}
}
%
\noindent
Aus Gleichung \ref{eq:pulsewave} lässt sich somit eine mathematische Darstellung für horizontale Streifenmuster mit unterschiedlicher Streifenbreite aufschreiben:
%
\begin{equation} \label{eq:impulsStreifenmuster}
	\begin{gathered}
		m_k(x,y) = A_m \left( 1 + p_{T,D}\left( x - \dfrac{T}{2\pi} \psi_k \right) \right),
		\\
		T = \dfrac{\acrshortmath{lwidth}}{N_p},
		\quad
		D \in \left(0,1\right] \subset \acrshortmath{real},
		\quad
		\psi_k = (k - 1)\dfrac{2\pi}{N_{shift}},
		\quad
		k \in \lbrace 1,\ldots,N_{shift}\rbrace
	\end{gathered}
\end{equation}
%
Wie auch in Gleichung \ref{eq:rstreifenmuster} gilt für dieses Muster die Periodizität.
Auch hier bezeichnet $A_m$ die Amplitude, $N_p$ die Anzahl der Perioden über die Monitorbreite \acrshort{lwidth}, $N_{shift}$ die Anzahl der Phasenverschiebungen und $\psi_k$ die Phasenverschiebung des $k$-ten Musters.
Zusätzlich zu diesen Parametern hat man den Tastgrad $D$, der in diesem Fall die Breite der hellen Streifen im Verhältnis zu der Periodenlänge $T$ angibt.
Analog zu Gleichung \ref{eq:impulsStreifenmuster} lässt sich auch ein vertikales Streifenmuster mit Ausbreitungsrichtung in $y$ über die Monitorhöhe \acrshort{lheight} aufschreiben.
Das Bild eines horizontalen Streifenmusters nach Gleichung \ref{eq:impulsStreifenmuster} wird in Abbildung \ref{img:impulsschwingungStreifenmuster} dargestellt.
%
\begin{figure}[H]
	\centering
	\includegraphics[frame,width=0.5\textwidth]{03_sichtpruefungDurchLichtstreuung/optimierungen/musterMitUnterschiedlichenStreifenbreiten/figures/impulsschwingungStreifenmuster}
	\caption[Rechteckförmiges Streifenmuster]{Streifenmuster nach Gleichung \ref{eq:rstreifenmuster}, mit $A_m = 127.5$, $N_p = 6$, $\acrshortmath{lwidth} = 384$, $D = \tfrac{1}{4}$ und $\psi = 0$. Die hellen Streifen haben damit eine Breite von 16 Pixeln und die dunklen Streifen eine Breite von 48 Pixeln.}
	\label{img:impulsschwingungStreifenmuster}
\end{figure}
}

% Verknüpfung von mehreren Kameraaufnahmen
{
	\FloatBarrier
    \subsection{Verknüpfung von mehreren Kameraaufnahmen}
    \label{sub:verknuepfungMehrererKameraaufnahmen}
    Da die Kameraeinstellungen auf die Prüfstation und Prüfbedingungen angepasst werden müssen, sind die Möglichkeiten zur Eliminierung der Streifen im Vorhinein begrenzt.
Die Nachbearbeitung kann über verschiedene Ansätze erfolgen.
Aufgrund der Periodizität und der festen Ausbreitungsrichtung der Streifen bietet es sich an, die Fourier-Analyse anzuwenden.
Man untersucht die Frequenzkomponenten der Streifen und filtert speziell diese aus dem Bild heraus, um sie zu entfernen.
Diese Methode kann allerdings nicht die fehlende Information an den dunklen Streifen wiederherstellen, sondern lediglich eine Bildverbesserung durchführen.
Eine andere Möglichkeit ist das Hinzuziehen von weiteren Bildern.
Da an den Stellen der horizontalen Streifen die Information fehlt, kann man, wie auch schon zuvor, zusätzliche Muster zur Hand nehmen, um die Informationen zusammenzufassen.
Man zieht weitere phasenverschobene Muster hinzu, sodass man eine gerade Anzahl an Phasenverschiebungen hat.
Daraus kann man die Informationen vereinen und eliminiert die horizontalen Streifen.
Im Folgenden werden als Beispiel vier Bilder von Streifenmustern miteinander verknüpft, die je eine versetzte Phase zueinander haben.
Aus den vier aufgenommenen Bildern verknüpft man je zwei Bilder, die eine Phasenverschiebung von $\pi$ zueinander haben, mit der betragsmäßigen Differenz.
Die zwei resultierenden Bilder haben zueinander versetzte Streifen, die aus den Überlappungen entstehen.
Zum Schluss kann man die beiden Bilder so verknüpfen, dass man stets den Bildpunkt mit dem höheren Helligkeitswert nimmt.
Das entspricht der Maximierung.
Analog kann man auch die horizontalen Streifen aus Abbildung \ref{img:minAndMaxLink} eliminieren, in welcher die Typen von Fehlstellen isoliert voneinander betrachtet wurden.
In Abbildung \ref{img:imageTree} werden die einzelnen Schritte dieses Verfahrens veranschaulicht.
Analog lässt sich auch die Verknüpfung von mehr als vier phasenverschobenen Streifenmustern durchführen.

\begin{figure}[H]
	\centering
	\includegraphics[width=\textwidth]{03_sichtpruefungDurchLichtstreuung/optimierungen/figures/imageTree}
	\caption[Prozess der Hervorhebung von Oberflächendefekten]{Prozess zur Hervorhebung von Oberflächendefekten mit $N_{shift} = 4$.}
	\label{img:imageTree}
\end{figure}
}

% Nachbearbeitung durch die Fourier-Analyse
{
	\FloatBarrier
    \subsection{Nachbearbeitung durch die Fourier-Analyse}
    \label{sub:nachbearbeitungFourierAnalyse}
    Im praktischen Durchlauf verbleiben auch im letzten Bild aus Abbildung \ref{img:imageTree} noch schwache horizontale Streifen.
Das liegt an den Ungenauigkeiten im Prüfprozess und der Mustererzeugung (vgl. Abschnitt \ref{sub:unterschiedeKameraUndMonitor}).
Da das Ergebnis mit weiteren phasenverschobenen Streifenmustern stetig besser wird, könnte man diese durch eine höhere Anzahl von Streifenmustern $N_{shift}$ eliminieren.
Dabei gilt zu beachten, dass man im Diskreten und Numerischen arbeitet und bei einer sehr hohen Anzahl von Phasenverschiebungen Messfehler und Gleitpunktfehler zum Tragen kommen.
Das bedeutet, man muss eine zum Prüfaufbau passende Anzahl an Mustern verwenden, um bessere Ergebnisse dokumentieren zu können.

\p
Zur Bildverbesserung hat man auch die Möglichkeit, die zuvor erwähnte Fourier-Analyse anzuwenden.
Die noch verbliebenen Strukturen lassen sich im Amplitudenspektrum des Bildes durch die Ausbreitungsrichtung finden.
Da diese Strukturen nur noch sehr schwach im Bild vorhanden ist, sind die jeweiligen Frequenzkomponenten auch dementsprechend gering gewichtet (siehe Abbildung \ref{img:amplitudeSpectrum}).

\begin{figure}[H]
	\centering
	\includegraphics[width=\textwidth]{03_sichtpruefungDurchLichtstreuung/optimierungen/figures/amplitudeSpectrum}
	\caption[Amplitudenspektrum des Gesamtbildes]{Amplitudenspektrum des Gesamtbildes. (mit Markierungen)}
	\label{img:amplitudeSpectrum}
\end{figure}

\noindent
In Rot ist im Bild die Ausbreitungsrichtung der ersten Struktur markiert.
Die entsprechenden Gewichte der Frequenzkomponenten sind in derselben Farbe im Amplitudenspektrum umrahmt.
Analog sind in Grün die für die zweite Struktur relevante Ausbreitungsrichtung und Gewichte der Frequenzkomponenten markiert.
Die grün gekennzeichnete Struktur entsteht in diesem Fall durch die Verwendung eines speziellen Polarisationsfilters auf dem Objektiv der Kamera, auf dessen Funktion nicht genauer eingegangen werden muss.
Filtert man im Amplitudenspektrum die markierten Bereiche heraus und wendet die inverse Fourier-Transformation an, erhält man ein Bild, indem die beiden störenden Strukturen nicht mehr vorhanden sind.
Das Ergebnisbild wird durch ein geeignetes Werkzeug \cite{fourierTool} berechnet und in Abbildung \ref{img:frequencyFiltered} dargestellt.

\begin{figure}[H]
	\centering
	\includegraphics[width=0.4\textwidth]{03_sichtpruefungDurchLichtstreuung/optimierungen/figures/frequencyFiltered}
	\caption[Bild mit angewandtem frequenzselektives Filter]{Angewandtes frequenzselektives Filter nach Abbildung \ref{img:amplitudeSpectrum}. Erstellt mit dem Online-Werkzeug \glqq Fourifier\grqq ~von © 2020 Ejectamenta.\cite{fourierTool}\footnotemark}
	\label{img:frequencyFiltered}
\end{figure}
\footnotetext{Anmerkungen: Das Bild musste aufgrund der Anforderungen des verwendeten Online-Werkzeugs zugeschnitten werden. Außerdem wurden die Frequenzkomponenten \glqq händisch\grqq ~entfernt, weshalb das Bild fehlerbehaftet ist. Das Bild konnte auch nicht auf Korrektheit geprüft werden und dient nur der Veranschaulichung. Die Nutzungsrechte unterliegen der Lizenz von © 2020 Ejectamenta. Für weitere Informationen zu Geschäfts- und Nutzungsbedingungen siehe: \url{https://ejectamenta.com/about/terms-of-service/}}
}

\noindent
Mit dem vorgestellten deflektometrischen Verfahren wurden Oberflächen\-de\-fekte wie z. B. Kratzer, Eingravierungen oder Partikel auf transparenten und spiegelnden Prüfobjekten sichtbar gemacht.
Das erzeugte Gesamtbild dieses Verfahrens kann nun durch anschließende Bildverarbeitung analysiert und geprüft werden.
In diesem konkreten Fall würden sich z. B. Kratzer detektieren oder die eingravierten Zeichen auslesen lassen.

\p
Geeignete Anpassungen der Parameter machen es möglich, diesen Prototyp für unterschiedliche Prüfobjekte, Aufbauten und Anforderungen anzuwenden.
Daraus ergibt sich das Ziel, eine allgemeine Lösung zu entwickeln, die mit möglichst wenig Aufwand für verschiedene Anwendungen genutzt werden kann.
}