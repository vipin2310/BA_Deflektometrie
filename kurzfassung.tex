%-PRÄAMBEL----------------------------------------------------------------
\documentclass[ngerman,11pt,twocolumn,a4paper]{article}

%-Import von Paketen------------------------------------------------------
\usepackage[utf8]{inputenc}
\usepackage{csquotes}
\usepackage{amsmath}
\usepackage{amsfonts}
\usepackage{amssymb}
\usepackage{dsfont}
\usepackage{fancyhdr}
\usepackage[greek.polutoniko,german]{babel}
\usepackage[backend=biber,style=numeric,sorting=none]{biblatex}
\usepackage{float}
\usepackage[section]{placeins} %Verhindert Fehlpositionierungen von Text / Abbildungen
\usepackage[skip=0pt, indent=17pt]{parskip}
\usepackage{sectsty}
%-------------------------------------------------------------------------

%-Seiten-Layout-----------------------------------------------------------
\usepackage[top=2.7cm, bottom=2.8cm, left=2.5cm, right=2.5cm, headheight=20pt]{geometry}
\pagestyle{fancy}
\fancyhead{}
\fancyhead[R]{\footnotesize \fancyplain{}{\textit{\leftmark}}}
\fancyfoot{}
\fancyfoot[R]{Seite \thepage}
\fancypagestyle{plain}
{%
\fancyhf{} % clear all header and footer fields
\fancyfoot[R]{Seite \thepage} % except the center
	\renewcommand{\headrulewidth}{0pt}
}
\predisplaypenalty=9900
%-------------------------------------------------------------------------

%-Definitionen von Commands / Eigenschaften-------------------------------
\AtBeginDocument{\renewcommand{\abstractname}{Abstract}}
%-------------------------------------------------------------------------

%-Definitionen von Worttrennungen-----------------------------------------
\hyphenation{De-flek-to-me-trie}
\hyphenation{de-flek-to-me-trisch}
\hyphenation{de-flek-to-me-tri-sche}
\hyphenation{De-flek-to-me-tri-sche}
\hyphenation{De-flek-to-me-trisch}
\hyphenation{Re-gis-t-rie-rung}
%-------------------------------------------------------------------------

%-Sonstiges---------------------------------------------------------------
\addbibresource{verweise.bib} %Quellenverzeichnis
\sectionfont{\raggedright} %Section-Titel linksbündig
%-------------------------------------------------------------------------

%-Dokument-Informationen--------------------------------------------------
\author{Vipin Singh}
\title{Kurzfassung der Bachelor-Thesis: Entwicklung eines deflektometrischen Prüfaufbaus für spiegelnde Prüfobjekte}
\date{\today}
%-------------------------------------------------------------------------

%-PRÄAMBEL-Ende-----------------------------------------------------------


% Anmerkungen zum Tex-File:
%	- jeder Satz beginnt in einer neuen Zeile
%-DOKUMENT-Start----------------------------------------------------------
\begin{document}
	% Titel und Abstract nicht in zwei Spalten
	\twocolumn[
	  	\begin{@twocolumnfalse}	
			\maketitle
			
			\begin{abstract}
				\noindent
				Das Ziel der vorliegenden Arbeit ist es, Verfahren zur Sichtprüfung von spiegelnden und transparenten Prüfoberflächen einzuführen und mathematisch zu erklären.
				Dabei wird erklärt, welche Methoden im wissenschaftlichen Gebiet der Deflektometrie eingesetzt werden, um Oberflächen vollständig zu erfassen.
				Zur Bearbeitung des Themas werden transparente Brillengläser und spiegelnde Keramikobjekte analysiert und mit den eingeführten Verfahren automatisiert ausgewertet.
				Die Ergebnisse der Auswertung durch die einführten Verfahren zeigten, dass es möglich ist, Anomalien der Oberflächenkrümmung spiegelnder und transparenter Prüfobjekte, wie z. B. Pickel, Dellen, Kratzer oder Gravuren, zu erfassen.
				Dadurch wird es ermöglicht, Oberflächendefekte spiegelnder und transparenter Prüfobjekte zu lokalisieren und qualitative Aussagen der Oberflächenbeschaffenheit zu formulieren.
			\end{abstract}
		\end{@twocolumnfalse}
	]
	
	\section{Einführung} \label{sec:einfuehrung}
	Glänzende Oberflächen faszinieren schon seit langer Zeit zahlreiche Menschen.
	Es gibt Studien, die besagen, dass Menschen die optischen Besonderheiten von glänzenden bzw. spiegelnden Oberflächen instinktiv mit Wasser assoziieren \cite{waterAndShininess}.
	Diese Faszination möchte auch die Industrie in uns anregen.
	So werden z. B. täglich große Karosserieflächen glänzend lackiert.
	Zur Qualitätssicherung ist es unumgänglich die Prüfung der Oberflächen durch automatisierte Prozesse umzusetzen.
	
	\par
	In dieser Ausarbeitung werden zwei Verfahren vorgeschlagen, um Oberflächendefekte von spiegelnden Prüfobjekten zu erfassen.
	
	\section{Grundlagen der Deflektometrie} \label{sec:grundlagenDerDeflektometrie}
	
	\section{Sichtprüfung durch Lichtstreuung} \label{sec:sichtpruefungDurchLichtstreuung}
	
	\section{Deflektometrische Registrierung} \label{sec:deflektometrischeRegistrierung}
	
	\section{Ergebnisse} \label{sec:ergebnisse}
	
	\section{Abschlussbemerkungen} \label{sec:abschlussbemerkungen}
	
	% Quellenverweise
	\printbibliography[title = Quellenverzeichnis]
\end{document}
%-DOKUMENT-Ende-----------------------------------------------------------
