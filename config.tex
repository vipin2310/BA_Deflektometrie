\usepackage[utf8]{inputenc}
\usepackage{csquotes}
\usepackage{amsmath}
\usepackage{amsfonts}
\usepackage{amssymb}
\usepackage{dsfont}
\usepackage{algorithm}
\usepackage{algpseudocode}
\usepackage[dvipsnames]{xcolor}
\usepackage{pdfpages}
\usepackage[greek.polutoniko,german]{babel}
\usepackage{graphicx}
\usepackage[bottom,splitrule]{footmisc}
\usepackage{float}
\usepackage[section]{placeins}
\usepackage[backend=biber,style=numeric,sorting=none]{biblatex}
\usepackage[format=hang]{caption}
\usepackage[export]{adjustbox}
\usepackage{multicol}
\usepackage{xurl}
\usepackage{fancyhdr}
\usepackage{pgfkeys}
\usepackage{tikz}
\usepackage{tkz-euclide}
\usepackage{pgfplots}
\pgfplotsset{compat=1.18}
\usetikzlibrary{shapes.callouts,shapes.geometric,calc,positioning,decorations.markings}
\usepackage{hyperref}
\usepackage[acronym,automake,nomain]{glossaries}

%---------------------------------------------------------------------------------------
\makeglossaries
\addbibresource{verweise.bib}

%Bib-Strings
\DefineBibliographyStrings{german}{%
urlseen = {Letzter Zugriff:},
}

%Abstract-Title
\AtBeginDocument{\renewcommand{\abstractname}{
\begin{flushleft}
{\Large Kurzfassung}
\end{flushleft}}}

%PageLayout
\linespread{1.2}
\usepackage[top=2.7cm, bottom=2.8cm, left=3cm, right=3cm, headheight=20pt]{geometry}
\pagestyle{fancy}
\fancyhead{}
\fancyhead[R]{\footnotesize \fancyplain{}{\textit{\leftmark}}}
\fancyfoot{}
\fancyfoot[R]{Seite \thepage}
\fancypagestyle{plain}
{%
	\fancyhf{} % clear all header and footer fields
	\fancyfoot[R]{Seite \thepage} % except the center
	\renewcommand{\headrulewidth}{0pt}
}
\predisplaypenalty=9900

%Defintions
\usepackage[framemethod=TikZ]{mdframed}
\newcounter{def}[chapter]\setcounter{def}{0}
\renewcommand{\thedef}{\arabic{chapter}.\arabic{def}}
\newenvironment{Definition}[2]
{
	\refstepcounter{def}
	\mdfsetup
	{
    	frametitle=
    	{
	        \tikz[baseline=(current bounding box.east),outer sep=0pt]
	        \node[anchor=east,rectangle,fill=Green!40]
	        {\strut Definition~\thedef:~#1};
	    },
	    innertopmargin=10pt,linecolor=Green!40,
    	linewidth=2pt,topline=true,
	    frametitleaboveskip=\dimexpr-\ht\strutbox\relax
	}
	\begin{mdframed}[backgroundcolor=Green!10]\relax
	\label{#2}
}
{
	\end{mdframed}
}

%Theorems
\newcounter{theo}[chapter]\setcounter{theo}{0}
\renewcommand{\thetheo}{\arabic{chapter}.\arabic{theo}}
\newenvironment{Satz}[2]
{
	\refstepcounter{theo}
	\mdfsetup
	{
    	frametitle=
    	{
	        \tikz[baseline=(current bounding box.east),outer sep=0pt]
	        \node[anchor=east,rectangle,fill=Green!40]
	        {\strut Satz~\thetheo:~#1};
	    },
	    innertopmargin=10pt,linecolor=Green!40,
    	linewidth=2pt,topline=true,
	    frametitleaboveskip=\dimexpr-\ht\strutbox\relax
	}
	\begin{mdframed}[backgroundcolor=Green!10]\relax
	\label{#2}
}
{
	\end{mdframed}
}

%Algorithms
\floatname{algorithm}{Algorithmus}
\renewcommand{\algorithmicrequire}{\textbf{Eingabe:}}
\renewcommand{\algorithmicensure}{\textbf{Ausgabe:}}

% redefine keywords
\algrenewcommand\algorithmicfunction{\textbf{\textcolor{violet}{function}}}
\algrenewcommand\algorithmicwhile{\textbf{\textcolor{violet}{while}}}
\algrenewcommand\algorithmicfor{\textbf{\textcolor{violet}{for}}}
\algrenewcommand\algorithmicif{\textbf{\textcolor{violet}{if}}}
\algrenewcommand\algorithmicelse{\textbf{\textcolor{violet}{else}}}
\algrenewcommand\algorithmicend{\textbf{\textcolor{violet}{end}}}
\algrenewcommand\algorithmicdo{\textbf{\textcolor{violet}{do}}}
\algrenewcommand\algorithmicthen{\textbf{\textcolor{violet}{then}}}
\algrenewcommand\algorithmicreturn{\textbf{\textcolor{violet}{return}}}
\algrenewcommand\algorithmicprocedure{\textbf{\textcolor{violet}{procedure}}}

%---------------------------------------------------------------------------------------
%New commands
\newcommand{\argmin}{\mathop{\mathrm{arg\,min}}} 		% argmin
\newcommand{\p}{\par\bigskip\noindent}					% Für Absätze
\newcommand{\acrshortmath}[1]{\text{\acrshort{#1}}}		% Für Zugriff auf Glossary in Mathe-Umgebungen

%Math-Operator
\DeclareMathOperator{\sign}{sgn}

%Tikz-Set
\tikzset{midarrow/.style={
        decoration={markings,
            mark= at position 0.5 with {\arrow{Latex[width=3mm]}} ,
        },
        postaction={decorate}
    }
}
