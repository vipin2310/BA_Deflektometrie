Die Bestimmung der deflektometrischen Registrierung nach Definition \ref{def:deflektometrischeRegistrierung} bedeutet die Zuordnung von Kamerapixeln zu Monitorpixeln.
Grundsätzlich soll jeder Lichtstrahl, der über das Prüfobjekt in die Kamera reflektiert wird im Kamerabild eindeutig zugeordnet werden können.
Eine einfache Möglichkeit eine solche Zuordnung zu erreichen bekommt man, indem man die Pixel des Monitors einzeln einschaltet und dabei die Veränderung im Kamerabild betrachtet.
Bei kleinen Bildern ist dies noch umsetzbar, allerdings wird die Anzahl von Pixeln mit zunehmender Auflösung schnell sehr groß und unübersichtlich, sodass dieser Ansatz nicht praktikabel ist.
Aus dem Grund ist es effektiver Bilder auf dem Monitor anzuzeigen, die Positionen visuell codieren können.
Durch die visuelle Erfassung des Spiegelbilds können somit die zugehörigen Positionen auf dem Monitor berechnet werden.

\p
Ein herkömmliches Kodierverfahren für solche Prozesse ist das Phasenschiebeverfahren.
Der Ansatz bei dem Verfahren ist es die Zeilen- und Spaltenpositionen des Monitors durch periodische Muster zu kodieren.
Die Grauwerte von periodischen Mustern nehmen dabei Werte an, die über die periodische Funktion bestimmt wurden.
Die hier verwendeten Funktionen sind Kosinusfunktionen.
Damit können die Pixel des Musters Phaseninformationen eines lokalen Orts übertragen.
Durch die Periodizität der Funktionen kann aus der reinen Phaseninformation noch nicht die genaue Monitorposition bestimmt werden.
Die Lösung dieses Problems wird als Phasenentfaltung bezeichnet (siehe Definition \ref{def:phasenentfaltung}).
Hierzu stellt Werling in seiner Arbeit \cite{kit_werling} ein mehrstufiges Phasenschiebeverfahren vor, das Muster mit unterschiedlichen Perioden verwendet.

\begin{Definition}{Phasenentfaltung}{def:phasenentfaltung}
	Die \textit{Phasenentfaltung} bezeichnet den Vorgang zur Auflösung der mehrdeutigen Zuordnung der Phaseninformation zu unterschiedlichen Perioden.
\end{Definition}

%Deflektometrische Registrierung ohne Phasenentfaltung
{
	\FloatBarrier
    \subsection{Deflektometrische Registrierung ohne Phasenentfaltung}
    \label{sub:registrierungOhnePhasenentfaltung}
    Ohne eine Phasenentfaltung bekommt man eventuell eine mehrdeutige Zuordnung aufgrund der Periodizität der Kosinusfunktion.
Soll eine eindeutige Zuordnung über das Phasenschiebeverfahren ohne Phasenentfaltung erfolgen, darf es also nur eine einzige Musterperiode auf der Monitorbreite bzw. Monitorhöhe geben.
Damit lässt sich die Kodierung der Monitorkoordinaten $(x_{L}, y_{L})$ durch die Phasen $(\phi_{x}, \phi_{y})$ folgendermaßen aufstellen:
%
\begin{equation}
	x_{L} = \dfrac{\text{\acrshort{lwidth}}}{2\pi}\phi_{x},
	\qquad
	y_{L} = \dfrac{\text{\acrshort{lheight}}}{2\pi}\phi_{y}
\end{equation}
%
\noindent
Die Monitorbreite wird dabei mit \acrshort{lwidth} und die Monitorhöhe mit \acrshort{lheight} angegeben.

\p
O.B.d.A. wird nachfolgend nur die deflektometrische Registrierung der Spaltenpositionen \acrshort{lrx} ($x$-Richtung) betrachtet.
Die deflektometrische Registrierung der Zeilenpositionen \acrshort{lry} ($y$-Richtung) kann analog bestimmt werden.
Das $k$-te Muster $m_k$ zur Kodierung der Monitorpunkte wird durch eine Kosinusfunktion aufgebaut und hat die Form:
%
\begin{equation}
	\begin{split}	
		m_k(x_L,y_L) = A_m \left(1 + C_m \cos \left(\dfrac{2\pi}{\text{\acrshort{lwidth}}} x_L + \psi_k\right)\right),
		\qquad
		k \in \lbrace 1,\ldots,N_{shift}\rbrace, \\
		\psi_k = (k - 1)\left(\dfrac{2\pi}{N_{shift}}\right)
	\end{split}
\end{equation}
%
$A_m$ bezeichnet die Amplitude, $C_m$ den Kontrast, $\psi_k$ die Phasenverschiebung des $k$-ten Musters und $N_{shift}$ die Anzahl an Mustern für das Phasenschiebeverfahren.
}

%Deflektometrische Registrierung mit Phasenentfaltung
{
	\FloatBarrier
    \subsection{Deflektometrische Registrierung mit Phasenentfaltung}
    \label{sub:registrierungMitPhasenentfaltung}
    Die hier beschriebene Methodik zur Bestimmung der deflektometrischen Registrierung ist ein mehrstufiges Phasenschiebeverfahren.
Ein solches Verfahren wird von Kammel in seiner Dissertation \cite{kit_kammel} vorgestellt.
Das Verfahren von Kammel zeigt jedoch in der Praxis Kodierungsartefakte bzw. Phasensprünge insbesondere an den Periodengrenzen.
Aus dem Grund stellt Werling darauf aufbauend in seiner Dissertation \cite{kit_werling} einen anderen Ansatz eines mehrstufigen Phasenschiebeverfahrens vor, der das Problem mit den Phasensprünge minimiert.
Die Idee hinter dem Ansatz ist dabei, dass man sich zunächst, analog zum Verfahren aus Kapitel \ref{sub:registrierungOhnePhasenentfaltung}, die $x$- bzw. $y$-Koordinaten in Relation zu den Perioden der Muster bestimmt.
Durch mehrere Muster mit unterschiedlichen Perioden erhält man schließlich mehrere relative unterschiedliche $x$- bzw. $y$-Koordinaten.
Diese müssen im finalen Ergebnis der richtigen Periode zugeordnet werden, indem der Abstand der Koordinaten minimiert wird.

\p
Analog zum Verfahren aus Kapitel \ref{sub:registrierungOhnePhasenentfaltung} benötigt man $N_{shift}$-viele Muster um die Phase eines Bildpunkts $(x_B, y_B)^\top$ zu bestimmen.
Zusätzlich betrachtet man mehrere Stufen des Verfahrens.
Das Musters auf der Stufe $i$ hat $N_\lambda^i$-viele Perioden über die Monitorbreite \acrshort{lwidth} bzw. -höhe \acrshort{lheight}.
Dabei sollen sich für unterschiedliche Stufen auch die Perioden der verwendeten Muster unterscheiden.
Die Phasen $\phi_x^i$ und $\phi_y^i$ der kodierten Monitorpunkte $(x_L, y_L)^\top$ auf der $i$-ten Stufe sehen dann folgendermaßen aus:
%
\begin{equation}
	\phi_x^i = \dfrac{2\pi N_\lambda^i}{\text{\acrshort{lwidth}}} x_L
	\qquad
	\phi_y^i = \dfrac{2\pi N_\lambda^i}{\text{\acrshort{lheight}}} y_L
\end{equation}

\p
O.B.d.A. wird nachfolgend nur die deflektometrische Registrierung der Spaltenpositionen \acrshort{lrx} ($x$-Richtung) betrachtet.
Die deflektometrische Registrierung der Zeilenpositionen \acrshort{lry} ($y$-Richtung) kann analog bestimmt werden.
Auf der Stufe $i$ hat das $k$-te Muster $m_k^i$ zur Kodierung der Monitorpunkte $(x_L, y_L)^\top$ somit die Form:
%
\begin{equation}\label{eq:monitormuster_mehrstufig}
	\begin{gathered}	
		m_k^i(x_L,y_L) = A_m^i \left(1 + C_m^i \cos \left(\phi_x^i + \psi_k\right)\right),\\
		k \in \lbrace 1,\ldots,N_{shift}\rbrace,
		\quad
		\psi_k = (k - 1)\dfrac{2\pi}{N_{shift}}
	\end{gathered}
\end{equation}
%
Es bezeichnet $A_m^i$ die Amplitude und $C_m^i$ den Kontrast des Musters der $i$-ten Stufe.
Wie auch in Kapitel \ref{sub:registrierungOhnePhasenentfaltung} entspricht $\psi_k$ der Phasenverschiebung des $k$-ten Musters der Stufen.
Analog zu Kapitel \ref{sub:registrierungOhnePhasenentfaltung} nimmt die Kamera das Signal $g_k^i$ auf:
%
\begin{equation}\label{eq:kamerabild}
	g_k^i(x_B, y_B) = A_g^i(x_B, y_B) \left(1 + C_g^i(x_B, y_B) \cos \left(\dfrac{2\pi N_\lambda^i}{\text{\acrshort{lwidth}}}\text{\acrshort{lrx}}(x_B, y_B) + \psi_k\right)\right)
\end{equation}
}