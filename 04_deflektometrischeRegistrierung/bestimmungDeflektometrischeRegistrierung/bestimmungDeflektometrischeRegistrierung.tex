Die Bestimmung der deflektometrischen Registrierung nach Definition \ref{def:deflektometrischeRegistrierung} bedeutet eine Zuordnung von Kamerapixeln zu Monitorpixeln.
Grundsätzlich soll jeder Lichtstrahl, der über das Prüfobjekt in die Kamera reflektiert wird im Kamerabild identifizierbar sein.
Eine einfache Möglichkeit eine solche Zuordnung zu erreichen bekommt man, indem man die Pixel des Monitors einzeln einschaltet und dabei die Veränderung im Kamerabild betrachtet.
Bei kleinen Bildern ist dies noch umsetzbar, allerdings wird die Anzahl von Pixeln mit zunehmender Auflösung schnell sehr groß und unübersichtlich, sodass dieser Ansatz nicht praktikabel ist.
Aus dem Grund ist es effektiver Bilder auf dem Monitor anzuzeigen, die Positionen visuell codieren können.
Durch die visuelle Erfassung des Spiegelbilds können somit die zugehörigen Positionen auf dem Monitor berechnet werden.

\p
Ein herkömmliches Kodierverfahren für solche Prozesse ist das Phasenschiebeverfahren.
Der Ansatz bei dem Verfahren ist es die Zeilen- und Spaltenpositionen des Monitors durch periodische Muster zu kodieren.
Die Grauwerte von periodischen Mustern nehmen dabei Werte an, die über die periodische Funktion bestimmt wurden.
Die hier verwendeten Funktionen sind Kosinusfunktionen.
Damit können die Pixel des Musters Phaseninformationen eines lokalen Orts übertragen.
Durch die Periodizität der Funktionen kann aus der reinen Phaseninformation noch nicht die genaue Monitorposition bestimmt werden.
Die Lösung dieses Problems wird als Phasenentfaltung bezeichnet (siehe auch Definition \ref{def:phasenentfaltung}).
Hierzu stellt Werling in seiner Arbeit \cite{kit_werling} ein mehrstufiges Phasenschiebeverfahren vor, das Muster mit unterschiedlichen Perioden verwendet.

\begin{Definition}{Phasenentfaltung}{def:phasenentfaltung}
	Die \textit{Phasenentfaltung} bezeichnet den Vorgang zur Auflösung der mehrdeutigen Zuordnung der Phaseninformation zu unterschiedlichen Perioden.
\end{Definition}

%Deflektometrische Registrierung ohne Phasenentfaltung
{
	\FloatBarrier
    \subsection{Deflektometrische Registrierung ohne Phasenentfaltung}
    \label{sub:registrierungOhnePhasenentfaltung}
    Ohne eine Phasenentfaltung bekommt man eventuell eine mehrdeutige Zuordnung aufgrund der Periodizität der Kosinusfunktion.
Soll eine eindeutige Zuordnung über das Phasenschiebeverfahren ohne Phasenentfaltung erfolgen, darf es also nur eine einzige Musterperiode auf der Monitorbreite bzw. Monitorhöhe geben.
Damit lässt sich die Kodierung der Monitorkoordinaten $(x_{L}, y_{L})$ durch die Phasen $(\phi_{x}, \phi_{y})$ folgendermaßen aufstellen:
%
\begin{equation}\label{eq:phasenkodierung}
	x_{L} = \dfrac{\text{\acrshort{lwidth}}}{2\pi}\phi_{x},
	\qquad
	y_{L} = \dfrac{\text{\acrshort{lheight}}}{2\pi}\phi_{y}
\end{equation}
%
\noindent
Die Monitorbreite wird dabei mit \acrshort{lwidth} und die Monitorhöhe mit \acrshort{lheight} angegeben.

\p
O.B.d.A. wird nachfolgend nur die deflektometrische Registrierung der Spaltenpositionen \acrshort{lrx} ($x$-Richtung) betrachtet.
Die deflektometrische Registrierung der Zeilenpositionen \acrshort{lry} ($y$-Richtung) kann analog bestimmt werden.
Das $k$-te Muster $m_k$ zur Kodierung der Monitorpunkte wird durch eine Kosinusfunktion aufgebaut und hat die Form:
%
\begin{equation}\label{eq:monitormuster}
	\begin{split}	
		m_k(x_L,y_L) = A_m \left(1 + C_m \cos \left(\dfrac{2\pi}{\text{\acrshort{lwidth}}} x_L + \psi_k\right)\right),
		\qquad
		k \in \lbrace 1,\ldots,N_{shift}\rbrace, \\
		\psi_k = (k - 1)\dfrac{2\pi}{N_{shift}}
	\end{split}
\end{equation}
%
$A_m$ bezeichnet die Amplitude, $C_m$ den Kontrast, $\psi_k$ die Phasenverschiebung des $k$-ten Musters und $N_{shift}$ die Anzahl an Mustern für das Phasenschiebeverfahren.
Das Muster wird auf dem Monitor angezeigt und über eine spiegelnde Oberfläche durch die Kamera beobachtet.
Über die Reflexion an der Oberfläche trifft der Sichtstrahl ausgehend vom Bildpunkt $(x_B, y_B)^\top \in A_{Cam}$ den Monitor im Punkt $(x_L, y_L)^\top \in L \cup \varnothing$ bestimmt durch die Abbildung aus der Gleichung \ref{eq:deflektometrischeRegistrierungAbbildung}:
%
\begin{equation}
	\begin{pmatrix}
		x_L \\ 
		y_L
	\end{pmatrix}
	= \text{\acrshort{lr}}(x_B, y_B) = 
	\begin{pmatrix}
		\text{\acrshort{lrx}}(x_B, y_B) \\ 
		\text{\acrshort{lry}}(x_B, y_B)
	\end{pmatrix} 
\end{equation}
%
Dementsprechend lässt sich das $k$-te Kamerabild $g$ am Bildpunkt $(x_B, y_B)^\top$ beschreiben durch das $k$-te Muster am Punkt $\text{\acrshort{lrx}}(x_B, y_B), \text{\acrshort{lry}}(x_B, y_B))^\top$:
%
\begin{equation}
	g_k(x_B, y_B) = m_k(\text{\acrshort{lrx}}(x_B, y_B), \text{\acrshort{lry}}(x_B, y_B))
\end{equation}
%
Setzt man für $m_k$ die Gleichung \ref{eq:monitormuster} unter Berücksichtigung der Veränderung durch die Oberfläche erhält man:
%
\begin{equation}\label{eq:kamerabild}
	g_k(x_B, y_B) = A_g(x_B, y_B) \left(1 + C_g(x_B, y_B) \cos \left(\dfrac{2\pi}{\text{\acrshort{lwidth}}}\text{\acrshort{lrx}}(x_B, y_B) + \psi_k\right)\right)
\end{equation}
%
Dabei sind die Amplitude $A_g$ und der Kontrast $C_g$ des Musters im Kamerabild unter Umständen unterschiedlich zu dem angezeigten Muster im Monitorbild.
Der Unterschied ist abhängig von dem Oberflächenpunkt an dem der Sichtstrahl reflektiert wird.
Aus dem Grund werden $A_g$ und $C_g$ in Abhängigkeit vom Kamerabildpunkt $(x_B, y_B)^\top$ angegeben.

\p
Durch Umstellung der Gleichung \ref{eq:phasenkodierung} nach der Phase und Einsetzen der deflektometrischen Registrierung \acrshort{lr}, erhält man die Phase $\phi_x$ des $k$-ten Musters in Abhängigkeit von den Kamerabildpunkten:
%
\begin{equation}\label{eq:phaseBildkoordinaten}
	\phi_x = \dfrac{2\pi}{\text{\acrshort{lwidth}}}\text{\acrshort{lrx}}(x_B, y_B)
\end{equation}
%
Durch Einsetzen der Phase aus Gleichung \ref{eq:phaseBildkoordinaten} in die Gleichung \ref{eq:kamerabild} des Kamerabilds erhält man den Zusammenhang zwischen der Phase $\phi_x$ des $k$-ten Musters und dem Kamerabild   $g_k$:
%
\begin{equation}\label{eq:kamerabildMitPhase}
	g_k(x_B, y_B) = A_g(x_B, y_B) \left(1 + C_g(x_B, y_B) \cos \left(\phi_x(x_B, y_B) + \psi_k\right)\right)
\end{equation}
%
Unter Einbeziehung der $N_{shift}$-vielen phasenverschobenen Streifenmuster kann man die Phase $\phi_x$ berechnen:
%
\begin{equation}\label{eq:tanPhase}
	\tan (\phi_x) = -\dfrac{\sum\limits_{k=1}^{N_{shift}} g_k(x_B, y_B) sin(\psi_k)}{\sum\limits_{k=1}^{N_{shift}} g_k(x_B, y_B) cos(\psi_k)}
\end{equation}
%
Aus den Gleichungen \ref{eq:phasenkodierung} und \ref{eq:tanPhase} folgt schließlich:
%
\begin{equation}\label{eq:registrierungX}
	x_L = \text{\acrshort{lrx}}(x_B, y_B) = 
	\dfrac{\text{\acrshort{lwidth}}}{2\pi}
	\arctan 
	\left( 
		-\dfrac
		{\sum\limits_{k=1}^{N_{shift}} g_k(x_B, y_B) sin\left((k - 1)\dfrac{2\pi}{N_{shift}}\right)}
		{\sum\limits_{k=1}^{N_{shift}} g_k(x_B, y_B) cos\left((k - 1)\dfrac{2\pi}{N_{shift}}\right)}
	\right)
\end{equation}
%
Mittels der Gleichung \ref{eq:registrierungX} ist somit die deflektometrische Registrierung in $x$-Richtung angegeben und die Monitorpositionen eindeutig bestimmt.
Die deflektometrische Registrierung in $y$-Richtung lässt sich analog dazu bestimmen.
Die Anzahl an Mustern bzw. Phasenverschiebungen $N_{shift}$ ist noch als Parameter übrig geblieben.
Um eine eindeutige Zuordnung zu erhalten, benötigt man aufgrund der drei unbekannten Größen Amplitude $A_g$, Kontrast $C_g$ und Phase $\phi_x$ in Gleichung \ref{eq:kamerabildMitPhase} eine Mindestanzahl von drei Phasenverschiebungen.

\p
Durch die Eindeutigkeit der Phase in dem verwendeten Muster spart man sich die Phasenentfaltung.
Dennoch ist das Phasenschiebeverfahren nach diesem Ansatz nicht sehr präzise und daher eher ungeeignet für Praxisanwendungen.
Der Grund liegt darin, dass Monitore lediglich eine vergleichsweise kleine Anzahl an Helligkeitsstufen darstellen können.
Die Anzahl an zu kodierenden Pixelpositionen sind in der Regel deutlich größer.
Eine ähnliche Beschränkung gibt es auch bei der Bildaufnahme, denn auch die Kamera nimmt eine Quantisierung vor und kann nicht sämtliche Helligkeitsstufen aufnehmen.
Zusammengenommen ist die Auflösung der Monitorpunkte damit stark begrenzt.
Eine höhere Ortsauflösung lässt sich erreichen indem man Muster mit mehreren Perioden über die Monitorbreite bzw. Monitorhöhe verwendet.
Damit wird schließlich eine Phasenentfaltung als zusätzliche Aufgabe erforderlich.
}

%Deflektometrische Registrierung mit Phasenentfaltung
{
	\FloatBarrier
    \subsection{Deflektometrische Registrierung mit Phasenentfaltung}
    \label{sub:registrierungMitPhasenentfaltung}
    Die hier beschriebene Methodik zur Bestimmung der deflektometrischen Registrierung ist ein mehrstufiges Phasenschiebeverfahren.
Ein solches Verfahren wird von Kammel in seiner Dissertation \cite{kit_kammel} vorgestellt.
Das Verfahren von Kammel zeigt jedoch in der Praxis Kodierungsartefakte bzw. Phasensprünge insbesondere an den Periodengrenzen.
Aus dem Grund stellt Werling darauf aufbauend in seiner Dissertation \cite{kit_werling} einen neuen Ansatz eines mehrstufigen Phasenschiebeverfahrens vor, der das Problem mit den Phasensprüngen minimiert.
Die Idee hinter dem Ansatz ist dabei, dass man sich zunächst, analog zum Verfahren aus Kapitel \ref{sub:registrierungOhnePhasenentfaltung}, die $x$- bzw. $y$-Koordinaten in Relation zu den Perioden der Muster bestimmt.
Durch mehrere Muster mit unterschiedlichen Perioden erhält man schließlich mehrere relative unterschiedliche $x$- bzw. $y$-Koordinaten.
Diese müssen im finalen Ergebnis der richtigen Periode zugeordnet werden, indem ein Optimierungsproblem gelöst wird.

\p
Analog zum Verfahren aus Kapitel \ref{sub:registrierungOhnePhasenentfaltung} benötigt man $N_{shift}$-viele Muster um die Phase eines Bildpunkts $(x_B, y_B)^\top$ zu bestimmen.
Zusätzlich betrachtet man mehrere Stufen des Verfahrens.
Das Musters auf der Stufe $i$ hat $N_\lambda^i$-viele Perioden über die Monitorbreite \acrshort{lwidth} bzw. -höhe \acrshort{lheight}.
Dabei sollen sich für unterschiedliche Stufen auch die Perioden der verwendeten Muster unterscheiden.
Die Phasen $\phi_x^i$ und $\phi_y^i$ der kodierten Monitorpunkte $(x_L, y_L)^\top$ auf der $i$-ten Stufe sehen dann folgendermaßen aus:
%
\begin{equation}
	\phi_x^i = \dfrac{2\pi N_\lambda^i}{\text{\acrshort{lwidth}}} x_L
	\qquad
	\phi_y^i = \dfrac{2\pi N_\lambda^i}{\text{\acrshort{lheight}}} y_L
\end{equation}

\p
O.B.d.A. wird nachfolgend nur die deflektometrische Registrierung der Spaltenpositionen \acrshort{lrx} ($x$-Richtung) betrachtet.
Die deflektometrische Registrierung der Zeilenpositionen \acrshort{lry} ($y$-Richtung) kann analog bestimmt werden.
Auf der Stufe $i$ hat das $k$-te Muster $m_k^i$ zur Kodierung der Monitorpunkte $(x_L, y_L)^\top$ somit die Form:
%
\begin{equation}\label{eq:monitormuster_mehrstufig}
	\begin{gathered}	
		m_k^i(x_L,y_L) = A_m^i \left(1 + C_m^i \cos \left(\phi_x^i + \psi_k\right)\right),\\
		k \in \lbrace 1,\ldots,N_{shift}\rbrace,
		\quad
		\psi_k = (k - 1)\dfrac{2\pi}{N_{shift}}
	\end{gathered}
\end{equation}
%
Es bezeichnet $A_m^i$ die Amplitude und $C_m^i$ den Kontrast des Musters der $i$-ten Stufe.
Wie auch in Kapitel \ref{sub:registrierungOhnePhasenentfaltung} entspricht $\psi_k$ der Phasenverschiebung des $k$-ten Musters der Stufen.
Analog zu Kapitel \ref{sub:registrierungOhnePhasenentfaltung} nimmt die Kamera das Signal $g_k^i$ auf:
%
\begin{equation}\label{eq:kamerabild_mehrstufig}
	g_k^i(x_B, y_B) = A_g^i(x_B, y_B) \left(1 + C_g^i(x_B, y_B) \cos \left(\dfrac{2\pi N_\lambda^i}{\text{\acrshort{lwidth}}}\text{\acrshort{lrx}}(x_B, y_B) + \psi_k\right)\right)
\end{equation}
%
Analog zu Kapitel \ref{sub:registrierungOhnePhasenentfaltung} lässt sich aus den Bildern $g_k^i$ die Phase der $i$-ten Stufe $\phi_x^i$ berechnen.
Vergleichbar zu Gleichung \ref{eq:registrierungX}, kann man in diesem Verfahren aus der Phase die Monitorkoordinaten relativ zum Intervall $[0,\text{\acrshort{lwidth}}/N_\lambda^i)$ bestimmen:
%
\begin{equation}\label{eq:registrierungX_relativ}
	x_{L,relativ}^i =
	\dfrac{\text{\acrshort{lwidth}}}{2\pi N_\lambda^i}
	\arctan 
	\left( 
		-\dfrac
		{\sum\limits_{k=1}^{N_{shift}} g_k^i(x_B, y_B) sin\left((k - 1)\dfrac{2\pi}{N_{shift}}\right)}
		{\sum\limits_{k=1}^{N_{shift}} g_k^i(x_B, y_B) cos\left((k - 1)\dfrac{2\pi}{N_{shift}}\right)}
	\right)
\end{equation}
%
Die absolute Monitorkoordinate $x_L^i$ der $i$-ten Stufe lässt sich bestimmen indem man zur Phase $\phi_x$ ein unbekanntes ganzzahliges Vielfaches von $2\pi$ addiert.
In Abhängigkeit von $x_{L,relativ}^i$ bedeutet das für $x_L^i$:
%
\begin{equation}\label{registrierungX_absolut}
	x_L^i = x_{L,relativ}^i + n^i \dfrac{\text{\acrshort{lwidth}}}{N_\lambda^i},
	\qquad
	(n^i \in \mathbb{N}_0)
\end{equation}
%
Dabei ist $n^i$ ein unbekannter ganzzahliger Faktor, der die absolute Auswertung bestimmt.
Zur Bestimmung des Faktors $n_i$ sollen zwei verschiedene Stufen des Verfahrens $i$,$j$ mit $i \neq j$ betrachtet werden.
Zur eindeutigen Lösbarkeit des nachfolgenden Gleichungssystems müssen $N_\lambda^i$ und $N_\lambda^j$ teilerfremd sein.
Das heißt es gilt:
%
\begin{equation*}
	ggT(N_\lambda^i, N_\lambda^j) = 1
	\quad
	\forall i \neq j
\end{equation*}
%
Dadurch erhält man zwei unterschiedliche Muster $m_k^i$ und $m_k^j$ und das eindeutig lösbare Gleichungssystem:
%
\begin{equation}\label{eq:gleichungssystemRegistrierung}
	\begin{split}
		x_L & = x_{L,relativ}^i + n^i \dfrac{\text{\acrshort{lwidth}}}{N_\lambda^i},\\
		x_L & = x_{L,relativ}^j + n^j \dfrac{\text{\acrshort{lwidth}}}{N_\lambda^j},
		\quad i \neq j,
		\quad n^i, n^j \in \mathbb{N}_0
	\end{split}
\end{equation}
%
In diesem Zusammenhang muss für die Monitorkoordinate $x_L = x_L^i = x_L^j$ gelten.
Die Gleichheit ist aufgrund von Ungenauigkeiten in der Kameraaufnahme nur schwierig zu erreichen, weshalb man aus dem Gleichungssystem ein Optimierungsproblem bildet.
Das heißt, gesucht ist folgende Näherungslösung $(n^i, n^j)$:
%
\begin{equation}\label{eq:optimierungsProblem_zweiStufen}
	\begin{gathered}	
		(n^i, n^j) = \argmin_{\alpha, \beta \in \mathbb{N}_0}
		\left\lvert
			\Bigg(
				x_{L,relativ}^i + \alpha \dfrac{\text{\acrshort{lwidth}}}{N_\lambda^i}
			\Bigg)
			-
			\Bigg(		
				x_{L,relativ}^j + \beta \dfrac{\text{\acrshort{lwidth}}}{N_\lambda^j}
			\Bigg)
		\right\rvert,\\
		\alpha \in \lbrace 0,\ldots, N_\lambda^i - 1\rbrace,
		\qquad
		\beta \in \lbrace 0,\ldots, N_\lambda^j - 1\rbrace
	\end{gathered}
\end{equation}
%
Durch weitere Stufen dieses Verfahrens erhöht sich zunehmend die Genauigkeit der Phasenentfaltung.
Das zu betrachtende Optimierungsproblem ergibt schließlich das Tupel $(n^1,\ldots, n^{N_{step}})$ zur Bestimmung der deflektometrischen Registrierung in $x$-Richtung:
%
\begin{equation}\label{eq:optimierungsProblem_mehrstufig}
	\begin{gathered}	
		(n^1,\ldots, n^{N_{step}}) = \argmin_{\mathrm{\textit{A}}}
		\sum\limits_{i = 1}^{N_{step}}
		\sum\limits_{j = i + 1}^{N_{step}}
		\left\lvert
			\Bigg(
				x_{L,relativ}^i + \alpha^i \dfrac{\text{\acrshort{lwidth}}}{N_\lambda^i}
			\Bigg)
			-
			\Bigg(		
				x_{L,relativ}^j + \alpha^j \dfrac{\text{\acrshort{lwidth}}}{N_\lambda^j}
			\Bigg)
		\right\rvert,\\
		\text{mit} ~\mathrm{\textit{A}} = (\alpha^1,\ldots,\alpha^{N{step}}) ~\text{und} ~\alpha^i = \lbrace 0,\ldots, N_\lambda^i - 1 \rbrace \subset \mathbb{N}_0
	\end{gathered}
\end{equation}
%
Mit dem Ergebnis aus Gleichung \ref{eq:optimierungsProblem_mehrstufig} lässt sich die absolute Monitorkoordinate in der Spalten $x_L^i$ berechnen.
Durch die Berechnung eines Mittelwerts der $x_L^i$ erhält man die beste Annäherung an den tatsächlichen Wert.
%
\begin{equation}\label{registrierungX_mehrstufig}
	\begin{split}
		x_L
		& =
			\text{\acrshort{lrx}}(x_B, y_B)\\
		& =
			\dfrac{1}{N_{step}}
			\sum\limits_{i = 1}^{N_{step}}
			x_{L,relativ}^i + n^i \dfrac{\text{\acrshort{lwidth}}}{N_\lambda^i}\\
		& =
			\dfrac{1}{N_{step}}
			\sum\limits_{i = 1}^{N_{step}}
			\dfrac{\text{\acrshort{lwidth}}}{2\pi N_\lambda^i}
			\arctan
			\left(
				-\dfrac
				{\sum\limits_{k=1}^{N_{shift}} g_k^i(x_B, y_B) sin\left((k - 1)\dfrac{2\pi}{N_{shift}}\right)}
				{\sum\limits_{k=1}^{N_{shift}} g_k^i(x_B, y_B) cos\left((k - 1)\dfrac{2\pi}{N_{shift}}\right)}
			\right)
			+ n^i \dfrac{\text{\acrshort{lwidth}}}{N_\lambda^i}
	\end{split}
\end{equation}
%
Durch Gleichung \ref{registrierungX_mehrstufig} ist somit die deflektometrische Registrierung der Spaltenpositionen \acrshort{lrx} nach der Phasenentfaltung angegeben.
Analog lässt sich dieses Verfahren auch für die deflektometrische Registrierung der Zeilenpositionen \acrshort{lry} anwenden.

\p
Für die eindeutige Lösbarkeit des Gleichungssystems aus \ref{eq:gleichungssystemRegistrierung} ist die Wahl der Anzahl an Perioden im Muster $N_\lambda^i$ entscheidend.
Da der Zusammenhang
%
\begin{equation*}
	ggT(N_\lambda^i, N_\lambda^j) = 1
	\quad
	\forall i \neq j
\end{equation*}
%
gelten soll, sind geeignete Wahlen für die $N_\lambda^i$ die Elemente aus der Menge der Primzahlen $\mathbb{P} = \lbrace 2, 3, 5, 7, 11,\ldots\rbrace$.
Die festzulegenden Parameter $N_{shift}$ und $N_{step}$ haben zunächst keine großen Auswirkungen auf die Lösbarkeit des Verfahrens, dennoch empfiehlt sich eine genügend hohe Anzahl an Phasenverschiebungen und Mustern mit unterschiedlichen Perioden zu wählen um die Genauigkeit des Verfahrens anzuheben.
}