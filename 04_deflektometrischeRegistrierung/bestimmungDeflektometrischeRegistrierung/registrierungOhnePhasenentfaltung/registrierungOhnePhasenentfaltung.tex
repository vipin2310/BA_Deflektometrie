Ohne eine Phasenentfaltung bekommt man eventuell eine mehrdeutige Zuordnung aufgrund der Periodizität der Kosinusfunktion.
Soll eine eindeutige Zuordnung über das Phasenschiebeverfahren ohne Phasenentfaltung erfolgen, darf es demnach nur eine einzige Musterperiode auf der Monitorbreite bzw. Monitorhöhe geben.
Damit lässt sich die Kodierung der Monitorkoordinaten $(x_{L}, y_{L})$ durch die Phasen $(\phi_{x}, \phi_{y})$ folgendermaßen aufstellen:
\begin{equation}\label{eq:phasenkodierung}
	x_{L} = \dfrac{\acrshortmath{lwidth}}{2\pi}\phi_{x},
	\qquad
	y_{L} = \dfrac{\acrshortmath{lheight}}{2\pi}\phi_{y}
\end{equation}
%
\noindent
Die Monitorbreite wird dabei mit \acrshort{lwidth} und die Monitorhöhe mit \acrshort{lheight} angegeben.

\p
O.B.d.A. wird unter Verwendung von Satz \ref{theo:separierbarkeitDeflektometrischeRegistrierung} nachfolgend nur die deflektometrische Registrierung der Spaltenpositionen \acrshort{lrx} ($x$-Richtung) betrachtet.
Die deflektometrische Registrierung der Zeilenpositionen \acrshort{lry} ($y$-Richtung) kann analog bestimmt werden.
Das $k$-te Muster $m_k$ zur Kodierung der Monitorpunkte wird durch eine Kosinusfunktion aufgebaut und hat die Form:
%
\begin{equation}\label{eq:monitormuster}
	\begin{gathered}	
		m_k(x_L,y_L) = A_m \left(1 + C_m \cos \left(\dfrac{2\pi}{\acrshortmath{lwidth}} x_L + \psi_k\right)\right),\\
		k \in \lbrace 1,\ldots,N_{shift}\rbrace,
		\quad
		\psi_k = (k - 1)\dfrac{2\pi}{N_{shift}}
	\end{gathered}
\end{equation}
%
$A_m$ bezeichnet die Amplitude, $C_m$ den Kontrast, $\psi_k$ die Phasenverschiebung des $k$-ten Musters und $N_{shift}$ die Anzahl der Muster für das Phasenschiebeverfahren.
Das Muster wird auf dem Monitor angezeigt und über eine spiegelnde Oberfläche durch die Kamera beobachtet.
Durch die Reflexion an der Oberfläche trifft der Sichtstrahl ausgehend vom Bildpunkt $(x_B, y_B)^\top \in A_{Cam}$ den Monitor im Punkt $(x_L, y_L)^\top \in L \cup \varnothing$ bestimmt durch die Abbildung aus der Gleichung \ref{eq:deflektometrischeRegistrierungAbbildung}:
%
\begin{equation}
	\begin{pmatrix}
		x_L \\ 
		y_L
	\end{pmatrix}
	= \acrshortmath{lr}(x_B, y_B) = 
	\begin{pmatrix}
		\acrshortmath{lrx}(x_B, y_B) \\ 
		\acrshortmath{lry}(x_B, y_B)
	\end{pmatrix} 
\end{equation}
%
Dementsprechend lässt sich das $k$-te Kamerabild $g$ am Bildpunkt $(x_B, y_B)^\top$ beschreiben durch das $k$-te Muster am Punkt $(\acrshortmath{lrx}(x_B, y_B), \acrshortmath{lry}(x_B, y_B))^\top$:
%
\begin{equation}
	g_k(x_B, y_B) = m_k(\acrshortmath{lrx}(x_B, y_B), \acrshortmath{lry}(x_B, y_B))
\end{equation}
%
Setzt man für $m_k$ die Gleichung \ref{eq:monitormuster} unter Berücksichtigung der Veränderung durch die Oberfläche erhält man:
%
\begin{equation}\label{eq:kamerabild}
	g_k(x_B, y_B) = A_g(x_B, y_B) \left(1 + C_g(x_B, y_B) \cos \left(\dfrac{2\pi}{\acrshortmath{lwidth}}\acrshortmath{lrx}(x_B, y_B) + \psi_k\right)\right)
\end{equation}
%
Dabei sind die Amplitude $A_g$ und der Kontrast $C_g$ des Musters im Kamerabild unter Umständen unterschiedlich zu dem angezeigten Muster im Monitorbild.
Der Unterschied ist abhängig von dem Oberflächenpunkt an dem der Sichtstrahl reflektiert wird.
Aus dem Grund werden $A_g$ und $C_g$ in Abhängigkeit vom Kamerabildpunkt $(x_B, y_B)^\top$ angegeben.

\p
Durch Umstellung der Gleichung \ref{eq:phasenkodierung} nach der Phase und Einsetzen der deflektometrischen Registrierung \acrshort{lr}, erhält man die Phase $\phi_x$ des $k$-ten Musters in Abhängigkeit von den Kamerabildpunkten:
%
\begin{equation}\label{eq:phaseBildkoordinaten}
	\phi_x = \dfrac{2\pi}{\acrshortmath{lwidth}}\acrshortmath{lrx}(x_B, y_B)
\end{equation}
%
Durch Einsetzen der Phase aus Gleichung \ref{eq:phaseBildkoordinaten} in die Gleichung \ref{eq:kamerabild} des Kamerabilds erhält man den Zusammenhang zwischen der Phase $\phi_x$ des $k$-ten Musters und dem Kamerabild   $g_k$:
%
\begin{equation}\label{eq:kamerabildMitPhase}
	g_k(x_B, y_B) = A_g(x_B, y_B) \left(1 + C_g(x_B, y_B) \cos \left(\phi_x(x_B, y_B) + \psi_k\right)\right)
\end{equation}
%
Unter Einbeziehung der $N_{shift}$-vielen phasenverschobenen Streifenmuster kann man die Phase $\phi_x$ berechnen \cite{kit_werling}:
%
\begin{equation}\label{eq:tanPhase}
	\tan (\phi_x) = -\dfrac{\sum\limits_{k=1}^{N_{shift}} g_k(x_B, y_B) sin(\psi_k)}{\sum\limits_{k=1}^{N_{shift}} g_k(x_B, y_B) cos(\psi_k)}
\end{equation}
%
Aus den Gleichungen \ref{eq:phasenkodierung} und \ref{eq:tanPhase} folgt schließlich:
%
\begin{equation}\label{eq:registrierungX}
	x_L = \acrshortmath{lrx}(x_B, y_B) = 
	\dfrac{\acrshortmath{lwidth}}{2\pi}
	\arctan 
	\left( 
		-\dfrac
		{\sum\limits_{k=1}^{N_{shift}} g_k(x_B, y_B) sin\left((k - 1)\dfrac{2\pi}{N_{shift}}\right)}
		{\sum\limits_{k=1}^{N_{shift}} g_k(x_B, y_B) cos\left((k - 1)\dfrac{2\pi}{N_{shift}}\right)}
	\right)
\end{equation}
%
Mit der Gleichung \ref{eq:registrierungX} ist die deflektometrische Registrierung in $x$-Richtung angegeben und die Monitorpositionen eindeutig bestimmt.
Die deflektometrische Registrierung in $y$-Richtung lässt sich analog dazu bestimmen.
Die Anzahl der Muster bzw. Phasenverschiebungen $N_{shift}$ bleibt noch als Parameter festzulegen.
Um eine eindeutige Zuordnung zu erhalten, benötigt man aufgrund der drei unbekannten Größen Amplitude $A_g$, Kontrast $C_g$ und Phase $\phi_x$ in Gleichung \ref{eq:kamerabildMitPhase} eine Mindestanzahl von drei Phasenverschiebungen.

\p
Durch die Eindeutigkeit der Zuordnung einer Spaltenposition aus Phase des verwendeten Musters spart man sich die Phasenentfaltung.
Dennoch ist das Phasenschiebeverfahren nach diesem Ansatz begrenzt präzise und daher eher ungeeignet für Praxisanwendungen.
Der Grund liegt darin, dass Bildschirme lediglich eine vergleichsweise kleine Anzahl von Helligkeitsstufen darstellen können.
Die Anzahl zu kodierender Pixelpositionen sind in der Regel deutlich größer.
Hinzu kommt die Limitierung durch die Kamera, welche eine maximale Auflösung und Anzahl Helligkeitsstufen aufnehmen kann.
Zusammengenommen ist die Auflösung der Monitorpunkte stark begrenzt durch die Abhängigkeit von der Monitorbreite bzw. -höhe.
Eine höhere Ortsauflösung lässt sich erreichen indem man Muster mit mehreren Perioden über die Monitorbreite bzw. Monitorhöhe verwendet.
Damit wird eine Phasenentfaltung als zusätzliche Aufgabe erforderlich.