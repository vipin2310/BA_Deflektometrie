Die hier beschriebene Methodik zur Bestimmung der deflektometrischen Registrierung ist ein mehrstufiges Phasenschiebeverfahren.
Ein solches Verfahren wird von Kammel in seiner Dissertation \cite{kit_kammel} vorgestellt.
Das Verfahren von Kammel zeigt jedoch in der Praxis Kodierungsartefakte bzw. Phasensprünge insbesondere an den Periodengrenzen.
Aus dem Grund stellt Werling darauf aufbauend in seiner Dissertation \cite{kit_werling} einen anderen Ansatz eines mehrstufigen Phasenschiebeverfahrens vor, der das Problem mit den Phasensprünge minimiert.
Die Idee hinter dem Ansatz ist dabei, dass man sich zunächst, analog zum Verfahren aus Kapitel \ref{sub:registrierungOhnePhasenentfaltung}, die $x$- bzw. $y$-Koordinaten in Relation zu den Perioden der Muster bestimmt.
Durch mehrere Muster mit unterschiedlichen Perioden erhält man schließlich mehrere relative unterschiedliche $x$- bzw. $y$-Koordinaten.
Diese müssen im finalen Ergebnis der richtigen Periode zugeordnet werden, indem der Abstand der Koordinaten minimiert wird.

\p
Analog zum Verfahren aus Kapitel \ref{sub:registrierungOhnePhasenentfaltung} benötigt man $N_{shift}$-viele Muster um die Phase eines Bildpunkts $(x_B, y_B)^\top$ zu bestimmen.
Zusätzlich betrachtet man mehrere Stufen des Verfahrens.
Das Musters auf der Stufe $i$ hat $N_\lambda^i$-viele Perioden über die Monitorbreite \acrshort{lwidth} bzw. -höhe \acrshort{lheight}.
Dabei sollen sich für unterschiedliche Stufen auch die Perioden der verwendeten Muster unterscheiden.
Die Phasen $\phi_x^i$ und $\phi_y^i$ der kodierten Monitorpunkte $(x_L, y_L)^\top$ auf der $i$-ten Stufe sehen dann folgendermaßen aus:
%
\begin{equation}
	\phi_x^i = \dfrac{2\pi N_\lambda^i}{\text{\acrshort{lwidth}}} x_L
	\qquad
	\phi_y^i = \dfrac{2\pi N_\lambda^i}{\text{\acrshort{lheight}}} y_L
\end{equation}

\p
O.B.d.A. wird nachfolgend nur die deflektometrische Registrierung der Spaltenpositionen \acrshort{lrx} ($x$-Richtung) betrachtet.
Die deflektometrische Registrierung der Zeilenpositionen \acrshort{lry} ($y$-Richtung) kann analog bestimmt werden.
Auf der Stufe $i$ hat das $k$-te Muster $m_k^i$ zur Kodierung der Monitorpunkte $(x_L, y_L)^\top$ somit die Form:
%
\begin{equation}\label{eq:monitormuster_mehrstufig}
	\begin{gathered}	
		m_k^i(x_L,y_L) = A_m^i \left(1 + C_m^i \cos \left(\phi_x^i + \psi_k\right)\right),\\
		k \in \lbrace 1,\ldots,N_{shift}\rbrace,
		\quad
		\psi_k = (k - 1)\dfrac{2\pi}{N_{shift}}
	\end{gathered}
\end{equation}
%
Es bezeichnet $A_m^i$ die Amplitude und $C_m^i$ den Kontrast des Musters der $i$-ten Stufe.
Wie auch in Kapitel \ref{sub:registrierungOhnePhasenentfaltung} entspricht $\psi_k$ der Phasenverschiebung des $k$-ten Musters der Stufen.
Analog zu Kapitel \ref{sub:registrierungOhnePhasenentfaltung} nimmt die Kamera das Signal $g_k^i$ auf:
%
\begin{equation}\label{eq:kamerabild_mehrstufig}
	g_k^i(x_B, y_B) = A_g^i(x_B, y_B) \left(1 + C_g^i(x_B, y_B) \cos \left(\dfrac{2\pi N_\lambda^i}{\text{\acrshort{lwidth}}}\text{\acrshort{lrx}}(x_B, y_B) + \psi_k\right)\right)
\end{equation}
%
Analog zu Kapitel \ref{sub:registrierungOhnePhasenentfaltung} lässt sich aus den Bildern $g_k^i$ die Phase der $i$-ten Stufe $\phi_x^i$ berechnen.
Vergleichbar zu Gleichung \ref{eq:registrierungX}, kann man in diesem Verfahren aus der Phase die Monitorkoordinaten relativ zum Intervall $[0,\text{\acrshort{lwidth}}/N_\lambda^i)$ bestimmen:
%
\begin{equation}\label{eq:registrierungX_relativ}
	x_{L,relativ}^i =
	\dfrac{\text{\acrshort{lwidth}}}{2\pi N_\lambda^i}
	\arctan 
	\left( 
		-\dfrac
		{\sum\limits_{k=1}^{N_{shift}} g_k^i(x_B, y_B) sin\left((k - 1)\dfrac{2\pi}{N_{shift}}\right)}
		{\sum\limits_{k=1}^{N_{shift}} g_k^i(x_B, y_B) cos\left((k - 1)\dfrac{2\pi}{N_{shift}}\right)}
	\right)
\end{equation}
%
Die absolute Monitorkoordinate $x_L^i$ der $i$-ten Stufe lässt sich bestimmen indem man zur Phase $\phi_x$ ein unbekanntes ganzzahliges Vielfaches von $2\pi$ addiert.
In Abhängigkeit von $x_{L,relativ}^i$ bedeutet das für $x_L^i$:
%
\begin{equation}\label{registrierungX_absolut}
	x_L^i = x_{L,relativ}^i + n^i \dfrac{\text{\acrshort{lwidth}}}{N_\lambda^i},
	\qquad
	(n^i \in \mathbb{N}_0)
\end{equation}
%
Dabei ist $n^i$ ein unbekannter ganzzahliger Faktor, der die absolute Auswertung bestimmt.
Zur Bestimmung des Faktors $n_i$ sollen zwei verschiedene Stufen des Verfahrens $i$,$j$ mit $i \neq j$ betrachtet werden.
Zur eindeutigen Lösbarkeit des nachfolgenden Gleichungssystems müssen $N_\lambda^i$ und $N_\lambda^j$ teilerfremd sein.
Das heißt es gilt:
%
\begin{equation*}
	ggT(N_\lambda^i, N_\lambda^j) = 1
	\quad
	\forall i \neq j
\end{equation*}
%
Dadurch erhält man zwei unterschiedliche Muster $m_k^i$ und $m_k^j$ und das eindeutig lösbare Gleichungssystem:
%
\begin{equation}\label{eq:gleichungssystemRegistrierung}
	\begin{split}
		x_L & = x_{L,relativ}^i + n^i \dfrac{\text{\acrshort{lwidth}}}{N_\lambda^i},\\
		x_L & = x_{L,relativ}^j + n^j \dfrac{\text{\acrshort{lwidth}}}{N_\lambda^j},
		\quad i \neq j,
		\quad n^i, n^j \in \mathbb{N}_0
	\end{split}
\end{equation}
%
In diesem Zusammenhang muss für die Monitorkoordinate $x_L = x_L^i = x_L^j$ gelten.
Die Gleichheit ist aufgrund von Ungenauigkeiten in der Kameraaufnahme nur schwierig zu erreichen, weshalb man aus dem Gleichungssystem ein Optimierungsproblem bildet.
Das heißt, gesucht ist folgende Näherungslösung $(n^i, n^j)$:
%
\begin{equation}\label{eq:optimierungsProblem_zweiStufen}
	\begin{gathered}	
		(n^i, n^j) = \argmin_{\alpha, \beta \in \mathbb{N}_0}
		\left\lvert
			\Bigg(
				x_{L,relativ}^i + \alpha \dfrac{\text{\acrshort{lwidth}}}{N_\lambda^i}
			\Bigg)
			-
			\Bigg(		
				x_{L,relativ}^j + \beta \dfrac{\text{\acrshort{lwidth}}}{N_\lambda^j}
			\Bigg)
		\right\rvert,\\
		\alpha \in \lbrace 0,\ldots, N_\lambda^i - 1\rbrace,
		\qquad
		\beta \in \lbrace 0,\ldots, N_\lambda^j - 1\rbrace
	\end{gathered}
\end{equation}
%
Durch weitere Stufen dieses Verfahrens erhöht sich zunehmend die Genauigkeit der Phasenentfaltung.
Das zu betrachtende Optimierungsproblem ergibt schließlich den Vektor $(n^1,\ldots, n^{N_{step}})$ zur Bestimmung der deflektometrischen Registrierung in $x$-Richtung:
%
\begin{equation}\label{eq:optimierungsProblem_mehrstufig}
	\begin{gathered}	
		(n^1,\ldots, n^{N_{step}}) = \argmin_{\mathrm{\textit{A}}}
		\sum\limits_{i = 1}^{N_{step}}
		\sum\limits_{j = i + 1}^{N_{step}}
		\left\lvert
			\Bigg(
				x_{L,relativ}^i + \alpha^i \dfrac{\text{\acrshort{lwidth}}}{N_\lambda^i}
			\Bigg)
			-
			\Bigg(		
				x_{L,relativ}^j + \alpha^j \dfrac{\text{\acrshort{lwidth}}}{N_\lambda^j}
			\Bigg)
		\right\rvert,\\
		\text{mit} ~\mathrm{\textit{A}} = (\alpha^1,\ldots,\alpha^{N{step}}) ~\text{und} ~\alpha^i = \lbrace 0,\ldots, N_\lambda^i - 1 \rbrace \subset \mathbb{N}_0
	\end{gathered}
\end{equation}
%TODO
Mit dem Vektor aus Gleichung \ref{eq:optimierungsProblem_mehrstufig}....