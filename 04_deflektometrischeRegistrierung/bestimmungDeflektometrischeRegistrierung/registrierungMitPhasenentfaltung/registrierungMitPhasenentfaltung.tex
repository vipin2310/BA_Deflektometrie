Die hier beschriebene Methodik zur Bestimmung der deflektometrischen Registrierung ist ein mehrstufiges Phasenschiebeverfahren.
Ein solches Verfahren wird von Kammel in seiner Dissertation \cite{kit_kammel} vorgestellt.
Das Verfahren von Kammel zeigt jedoch in der Praxis Kodierungsartefakte bzw. Phasensprünge insbesondere an den Periodengrenzen.
Aus dem Grund stellt Werling darauf aufbauend in seiner Dissertation \cite{kit_werling} einen neuen Ansatz eines mehrstufigen Phasenschiebeverfahrens vor, der das Problem mit den Phasensprüngen minimiert.
Die Idee hinter dem Ansatz ist dabei, dass man sich zunächst, analog zum Verfahren aus Kapitel \ref{sub:registrierungOhnePhasenentfaltung}, die $x$- bzw. $y$-Koordinaten in Relation zu den Perioden der Muster bestimmt.
Durch mehrere Muster mit unterschiedlichen Perioden erhält man schließlich mehrere relative unterschiedliche $x$- bzw. $y$-Koordinaten.
Diese müssen im finalen Ergebnis der richtigen Periode zugeordnet werden, indem ein Optimierungsproblem gelöst wird.

\p
Analog zum Verfahren aus Kapitel \ref{sub:registrierungOhnePhasenentfaltung} benötigt man $N_{shift}$-viele Muster um die Phase eines Bildpunkts $(x_B, y_B)^\top$ zu bestimmen.
Zusätzlich betrachtet man mehrere Stufen des Verfahrens.
Das Musters auf der Stufe $i$ hat $N_\lambda^i$-viele Perioden über die Monitorbreite \acrshort{lwidth} bzw. -höhe \acrshort{lheight}.
Dabei sollen sich für unterschiedliche Stufen auch die Perioden der verwendeten Muster unterscheiden.
Die Phasen $\phi_x^i$ und $\phi_y^i$ der kodierten Monitorpunkte $(x_L, y_L)^\top$ auf der $i$-ten Stufe sehen dann folgendermaßen aus:
%
\begin{equation}
	\phi_x^i = \dfrac{2\pi N_\lambda^i}{\acrshortmath{lwidth}} x_L
	\qquad
	\phi_y^i = \dfrac{2\pi N_\lambda^i}{\acrshortmath{lheight}} y_L
\end{equation}

\p
O.B.d.A. wird nachfolgend nur die deflektometrische Registrierung der Spaltenpositionen \acrshort{lrx} ($x$-Richtung) betrachtet.
Die deflektometrische Registrierung der Zeilenpositionen \acrshort{lry} ($y$-Richtung) kann analog bestimmt werden.
Auf der Stufe $i$ hat das $k$-te Muster $m_k^i$ zur Kodierung der Monitorpunkte $(x_L, y_L)^\top$ somit die Form:
%
\begin{equation}\label{eq:monitormuster_mehrstufig}
	\begin{gathered}	
		m_k^i(x_L,y_L) = A_m^i \left(1 + C_m^i \cos \left(\phi_x^i + \psi_k\right)\right),\\
		k \in \lbrace 1,\ldots,N_{shift}\rbrace,
		\quad
		\psi_k = (k - 1)\dfrac{2\pi}{N_{shift}}
	\end{gathered}
\end{equation}
%
Es bezeichnet $A_m^i$ die Amplitude und $C_m^i$ den Kontrast des Musters der $i$-ten Stufe.
Wie auch in Kapitel \ref{sub:registrierungOhnePhasenentfaltung} entspricht $\psi_k$ der Phasenverschiebung des $k$-ten Musters der Stufen.
Analog zu Kapitel \ref{sub:registrierungOhnePhasenentfaltung} nimmt die Kamera das Signal $g_k^i$ auf:
%
\begin{equation}\label{eq:kamerabild_mehrstufig}
	g_k^i(x_B, y_B) = A_g^i(x_B, y_B) \left(1 + C_g^i(x_B, y_B) \cos \left(\dfrac{2\pi N_\lambda^i}{\acrshortmath{lwidth}}\acrshortmath{lrx}(x_B, y_B) + \psi_k\right)\right)
\end{equation}
%
Analog zu Kapitel \ref{sub:registrierungOhnePhasenentfaltung} lässt sich aus den Bildern $g_k^i$ die Phase der $i$-ten Stufe $\phi_x^i$ berechnen.
Vergleichbar zu Gleichung \ref{eq:registrierungX}, kann man in diesem Verfahren aus der Phase die Monitorkoordinaten relativ zum Intervall $[0,\acrshortmath{lwidth}/N_\lambda^i)$ bestimmen:
%
\begin{equation}\label{eq:registrierungX_relativ}
	x_{L,relativ}^i =
	\dfrac{\acrshortmath{lwidth}}{2\pi N_\lambda^i}
	\arctan 
	\left( 
		-\dfrac
		{\sum\limits_{k=1}^{N_{shift}} g_k^i(x_B, y_B) sin\left((k - 1)\dfrac{2\pi}{N_{shift}}\right)}
		{\sum\limits_{k=1}^{N_{shift}} g_k^i(x_B, y_B) cos\left((k - 1)\dfrac{2\pi}{N_{shift}}\right)}
	\right)
\end{equation}
%
Die absolute Monitorkoordinate $x_L^i$ der $i$-ten Stufe lässt sich bestimmen indem man zur Phase $\phi_x$ ein unbekanntes ganzzahliges Vielfaches von $2\pi$ addiert.
In Abhängigkeit von $x_{L,relativ}^i$ bedeutet das für $x_L^i$:
%
\begin{equation}\label{eq:registrierungX_absolut}
	x_L^i = x_{L,relativ}^i + n^i \dfrac{\acrshortmath{lwidth}}{N_\lambda^i},
	\qquad
	(n^i \in \mathbb{N}_0)
\end{equation}
%
Dabei ist $n^i$ ein unbekannter ganzzahliger Faktor, der die absolute Auswertung bestimmt.
Zur Bestimmung des Faktors $n_i$ sollen zwei verschiedene Stufen des Verfahrens $i$,$j$ mit $i \neq j$ betrachtet werden.
Zur eindeutigen Lösbarkeit des nachfolgenden Gleichungssystems müssen $N_\lambda^i$ und $N_\lambda^j$ teilerfremd sein.
Das heißt es gilt:
%
\begin{equation*}
	ggT(N_\lambda^i, N_\lambda^j) = 1
	\quad
	\forall i \neq j
\end{equation*}
%
Dadurch erhält man zwei unterschiedliche Muster $m_k^i$ und $m_k^j$ und das eindeutig lösbare Gleichungssystem:
%
\begin{equation}\label{eq:gleichungssystemRegistrierung}
	\begin{split}
		x_L & = x_{L,relativ}^i + n^i \dfrac{\acrshortmath{lwidth}}{N_\lambda^i},\\
		x_L & = x_{L,relativ}^j + n^j \dfrac{\acrshortmath{lwidth}}{N_\lambda^j},
		\quad i \neq j,
		\quad n^i, n^j \in \mathbb{N}_0
	\end{split}
\end{equation}
%
In diesem Zusammenhang muss für die Monitorkoordinate $x_L = x_L^i = x_L^j$ gelten.
Die Gleichheit ist aufgrund von Ungenauigkeiten in der Kameraaufnahme nur schwierig zu erreichen, weshalb man aus dem Gleichungssystem ein Optimierungsproblem bildet.
Das heißt, gesucht ist folgende Näherungslösung $(n^i, n^j)$:
%
\begin{equation}\label{eq:optimierungsProblem_zweiStufen}
	\begin{gathered}	
		(n^i, n^j) = \argmin_{\alpha, \beta \in \mathbb{N}_0}
		\left\lvert
			\Bigg(
				x_{L,relativ}^i + \alpha \dfrac{\acrshortmath{lwidth}}{N_\lambda^i}
			\Bigg)
			-
			\Bigg(		
				x_{L,relativ}^j + \beta \dfrac{\acrshortmath{lwidth}}{N_\lambda^j}
			\Bigg)
		\right\rvert,\\
		\alpha \in \lbrace 0,\ldots, N_\lambda^i - 1\rbrace,
		\qquad
		\beta \in \lbrace 0,\ldots, N_\lambda^j - 1\rbrace
	\end{gathered}
\end{equation}
%
Durch weitere Stufen dieses Verfahrens erhöht sich zunehmend die Genauigkeit der Phasenentfaltung.
Das zu betrachtende Optimierungsproblem ergibt schließlich das Tupel $(n^1,\ldots, n^{N_{step}})$ zur Bestimmung der deflektometrischen Registrierung in $x$-Richtung:
%
\begin{equation}\label{eq:optimierungsProblem_mehrstufig}
	\begin{gathered}	
		(n^1,\ldots, n^{N_{step}}) = \argmin_{\mathrm{\textit{A}}}
		\sum\limits_{i = 1}^{N_{step}}
		\sum\limits_{j = i + 1}^{N_{step}}
		\left\lvert
			\Bigg(
				x_{L,relativ}^i + \alpha^i \dfrac{\acrshortmath{lwidth}}{N_\lambda^i}
			\Bigg)
			-
			\Bigg(		
				x_{L,relativ}^j + \alpha^j \dfrac{\acrshortmath{lwidth}}{N_\lambda^j}
			\Bigg)
		\right\rvert,\\
		\text{mit} ~\mathrm{\textit{A}} = (\alpha^1,\ldots,\alpha^{N{step}}) ~\text{und} ~\alpha^i = \lbrace 0,\ldots, N_\lambda^i - 1 \rbrace \subset \mathbb{N}_0
	\end{gathered}
\end{equation}
%
% Tikz-Picture: Bestimmung eindeutiger Position
{
	\FloatBarrier
	\begin{figure}[H]
	\centering
	
	\begin{adjustbox}{width=\textwidth}
		\begin{tikzpicture}
		
			% Koordinatensystem 1
			\draw[thick,-stealth,black] (0,0)--(16,0) node[below] {$x_L$};
			\draw[thick,-stealth,black] (0,0)--(0,4.25) node[left] {$m_0^1(x_L,y_0)$};
			\draw[thick,black] (0,0) -- (0,-0.1) node[anchor=north,fill=white] {$0$};
			\draw[thick,black] (15,0) -- (15,-0.1) node[anchor=north,fill=white] {$L_{width}$};
			\draw[thick,black] (0,0) -- (-0.1,0) node[anchor=east,fill=white] {$0$};
			\draw[thick,black] (0,3.75) -- (-0.1,3.75) node[anchor=east,fill=white] {$255$};
			
			% Koordinatensystem 2
			\draw[thick,-stealth,black] (0,-5.5)--(16,-5.5) node[below] {$x_L$};
			\draw[thick,-stealth,black] (0,-5.5)--(0,-1.25) node[left] {$m_0^2(x_L,y_0)$};
			\draw[thick,black] (0,-5.5) -- (0,-5.6) node[anchor=north,fill=white] {$0$};
			\draw[thick,black] (15,-5.5) -- (15,-5.6) node[anchor=north,fill=white] {$L_{width}$};
			\draw[thick,black] (0,-5.5) -- (-0.1,-5.5) node[anchor=east,fill=white] {$0$};
			\draw[thick,black] (0,-1.75) -- (-0.1,-1.75) node[anchor=east,fill=white] {$255$};
			
			% Koordinatensystem 3
			\draw[thick,-stealth,black] (0,-11)--(16,-11) node[below] {$x_L$};
			\draw[thick,-stealth,black] (0,-11)--(0,-6.75) node[left] {$m_0^3(x_L,y_0)$};
			\draw[thick,black] (0,-11) -- (0,-11.1) node[anchor=north,fill=white] {$0$};
			\draw[thick,black] (15,-11) -- (15,-11.1) node[anchor=north,fill=white] {$L_{width}$};
			\draw[thick,black] (0,-11) -- (-0.1,-11) node[anchor=east,fill=white] {$0$};
			\draw[thick,black] (0,-7.25) -- (-0.1,-7.25) node[anchor=east,fill=white] {$255$};
			
			% Funktion 1
			\draw[name path = func1, blue, thick, domain=0:15, samples=600] plot (\x,{1.875*cos(((2*pi*3)/15)*\x r) + 1.875});
			
			% Funktion 2
			\draw[name path = func2, blue, thick, domain=0:15, samples=600] plot (\x,{1.875*cos(((2*pi*4)/15)*\x r) - 3.625});
			
			% Funktion 3
			\draw[name path = func3, blue, thick, domain=0:15, samples=600] plot (\x,{1.875*cos(((2*pi*6)/15)*\x r) - 9.125});
		
			% Perioden Funktion 1
			\draw[dashed,blue] (5,-0.5) -- (5,4.25);
			\draw[dashed,blue] (10,-0.5) -- (10,4.25);
			\node[anchor=north] at (2.5,4.25) {$\color{blue} Per. ~1$};
			\node[anchor=north] at (7.5,4.25) {$\color{blue} Per. ~2$};
			\node[anchor=north] at (12.5,4.25) {$\color{blue} Per. ~3$};
			
			% Perioden Funktion 2
			\draw[dashed,blue] (3.75,-6) -- (3.75,-1.25);
			\draw[dashed,blue] (7.5,-6) -- (7.5,-1.25);
			\draw[dashed,blue] (11.25,-6) -- (11.25,-1.25);
			\node[anchor=north] at (1.875,-1.25) {$\color{blue} Per. ~1$};
			\node[anchor=north] at (5.625,-1.25) {$\color{blue} Per. ~2$};
			\node[anchor=north] at (9.375,-1.25) {$\color{blue} Per. ~3$};
			\node[anchor=north] at (13.125,-1.25) {$\color{blue} Per. ~4$};
			
			% Perioden Funktion 3
			\draw[dashed,blue] (2.5,-11.5) -- (2.5,-6.75);
			\draw[dashed,blue] (5,-11.5) -- (5,-6.75);
			\draw[dashed,blue] (7.5,-11.5) -- (7.5,-6.75);
			\draw[dashed,blue] (10,-11.5) -- (10,-6.75);
			\draw[dashed,blue] (12.5,-11.5) -- (12.5,-6.75);
			\node[anchor=north] at (1.25,-6.75) {$\color{blue} Per. ~1$};
			\node[anchor=north] at (3.75,-6.75) {$\color{blue} Per. ~2$};
			\node[anchor=north] at (6.25,-6.75) {$\color{blue} Per. ~3$};
			\node[anchor=north] at (8.75,-6.75) {$\color{blue} Per. ~4$};
			\node[anchor=north] at (11.25,-6.75) {$\color{blue} Per. ~5$};
			\node[anchor=north] at (13.75,-6.75) {$\color{blue} Per. ~6$};
			
			% Absolute x_L Linie
			\draw[name path = absoluteLine, ultra thick, Green] (6.9,-11.5) -- (6.9,4.25) node[left] {$x_{L,absolut}$};
			
			% Punkte x_L absolut
			\path [name intersections={of=func1 and absoluteLine,by=a1}];
			\path [name intersections={of=func2 and absoluteLine,by=a2}];
			\path [name intersections={of=func3 and absoluteLine,by=a3}];
			
			% Punkte x_L relativ
			\coordinate (r1) at (1.9,{1.875*cos(((2*pi*3)/15)*1.9 r) + 1.875});
			\coordinate (r2) at (3.15,{1.875*cos(((2*pi*4)/15)*3.15 r) - 3.625});
			\coordinate (r3) at (1.9,{1.875*cos(((2*pi*6)/15)*1.9 r) - 9.125});
			
			% Horizontale Linien
			\draw[violet, thick, shorten <= -2cm, shorten >= -9.1cm] (r1) -- (a1);
			\draw[violet, thick, shorten <= -3.25cm, shorten >= -9.1cm] (r2) -- (a2);
			\draw[violet, thick, shorten <= -2cm, shorten >= -9.1cm] (r3) -- (a3);
			
			% Die Punkte für Funktion 1 zeichnen
			\draw[violet,fill=violet] (r1) circle (2.5pt);
			\draw[red,fill=red] (a1) circle (2.5pt);
			\draw[yellow,fill=yellow] (11.9,{1.875*cos(((2*pi*3)/15)*11.9 r) + 1.875}) circle (2.5pt);
			
			% Die Punkte für Funktion 2 zeichnen
			\draw[violet,fill=violet] (r2) circle (2.5pt);
			\draw[red,fill=red] (a2) circle (2.5pt);
			\draw[yellow,fill=yellow] (10.65,{1.875*cos(((2*pi*4)/15)*10.65 r) - 3.625}) circle (2.5pt);
			\draw[yellow,fill=yellow] (14.4,{1.875*cos(((2*pi*4)/15)*14.4 r) - 3.625}) circle (2.5pt);
			
			% Die Punkte für Funktion 3 zeichnen
			\draw[violet,fill=violet] (r3) circle (2.5pt);
			\draw[yellow,fill=yellow] ((4.4,{1.875*cos(((2*pi*6)/15)*4.4 r) - 9.125}) circle (2.5pt);
			\draw[red,fill=red] (a3) circle (2.5pt);
			\draw[yellow,fill=yellow] ((9.4,{1.875*cos(((2*pi*6)/15)*9.4 r) - 9.125}) circle (2.5pt);
			\draw[yellow,fill=yellow] ((11.9,{1.875*cos(((2*pi*6)/15)*11.9 r) - 9.125}) circle (2.5pt);
			\draw[yellow,fill=yellow] ((14.4,{1.875*cos(((2*pi*6)/15)*14.4 r) - 9.125}) circle (2.5pt);
			
			% x_L relativ Beschriftung an x-Achse
			\draw[thick,dashed,violet] (r1) -- (1.9,-0.1) node[anchor=north,fill=white] {$x_{L,relativ}^1$};
			\draw[thick,dashed,violet] (r2) -- (3.15,-5.6) node[anchor=north,fill=white] {$x_{L,relativ}^2$};
			\draw[thick,dashed,violet] (r3) -- (1.9,-11.1) node[anchor=north,fill=white] {$x_{L,relativ}^3$};
			
		\end{tikzpicture}
	\end{adjustbox}
	
	\caption[Bestimmung eindeutiger Position]{Bestimmung eindeutiger Position durch Verwendung von unterschiedlichen Perioden.}
	\label{tikz:bestimmungEindeutigerPosition}
\end{figure}


}
%
Mit dem Ergebnis aus Gleichung \ref{eq:optimierungsProblem_mehrstufig} lässt sich die absolute Monitorkoordinate in den Spalten $x_L^i$ berechnen.
Durch die Berechnung eines Mittelwerts der $x_L^i$ erhält man die beste Annäherung an den tatsächlichen Wert.
%
\begin{equation}\label{eq:registrierungX_mehrstufig}
	\begin{split}
		x_L
		& =
			\acrshortmath{lrx}(x_B, y_B)\\
		& =
			\dfrac{1}{N_{step}}
			\sum\limits_{i = 1}^{N_{step}}
			x_{L,relativ}^i + n^i \dfrac{\acrshortmath{lwidth}}{N_\lambda^i}\\
		& =
			\dfrac{1}{N_{step}}
			\sum\limits_{i = 1}^{N_{step}}
			\dfrac{\acrshortmath{lwidth}}{2\pi N_\lambda^i}
			\arctan
			\left(
				-\dfrac
				{\sum\limits_{k=1}^{N_{shift}} g_k^i(x_B, y_B) sin\left((k - 1)\dfrac{2\pi}{N_{shift}}\right)}
				{\sum\limits_{k=1}^{N_{shift}} g_k^i(x_B, y_B) cos\left((k - 1)\dfrac{2\pi}{N_{shift}}\right)}
			\right)
			+ n^i \dfrac{\acrshortmath{lwidth}}{N_\lambda^i}
	\end{split}
\end{equation}
%
Durch Gleichung \ref{eq:registrierungX_mehrstufig} ist somit die deflektometrische Registrierung der Spaltenpositionen \acrshort{lrx} nach der Phasenentfaltung angegeben.
Analog lässt sich dieses Verfahren auch für die deflektometrische Registrierung der Zeilenpositionen \acrshort{lry} anwenden.

\p
Für die eindeutige Lösbarkeit des Gleichungssystems aus \ref{eq:gleichungssystemRegistrierung} ist die Wahl der Anzahl an Perioden im Muster $N_\lambda^i$ entscheidend.
Da der Zusammenhang
%
\begin{equation*}
	ggT(N_\lambda^1, \ldots, N_\lambda^{N_{step}}) = 1
\end{equation*}
%
gelten soll, sind geeignete Wahlen für die Anzahl an Perioden $N_\lambda^i$ die Elemente aus der Menge der Primzahlen $\mathbb{P} = \lbrace 2, 3, 5, 7, 11,\ldots\rbrace$.
Die festzulegenden Parameter $N_{shift}$ und $N_{step}$ haben zunächst keine großen Auswirkungen auf die Lösbarkeit des Verfahrens, dennoch empfiehlt sich eine genügend hohe Anzahl an Phasenverschiebungen und Mustern mit unterschiedlichen Perioden zu wählen um die Genauigkeit des Verfahrens anzuheben.

\p
Das beschriebene Verfahren zur Bestimmung der deflektometrischen Registrierung \acrshort{lr} wird nachfolgend im Algorithmus \ref{alg:bestimmungDeflektometrischeRegistrierung} zusammengefasst.

%Algorithmus zur Bestimmung der deflektometrischen Registrierung mit Phasenentfaltung
{
	\FloatBarrier
	\begin{algorithm}[H]
\caption{Bestimmung der deflektometrischen Registrierung mit Phasenentfaltung}
	\label{alg:bestimmungDeflektometrischeRegistrierung}
	\begin{algorithmic}[1]
		%Input
		\Require $N_{shift} \geq 3, ~N_{step} \geq 1, ~N_\lambda^i \geq 1 ~\forall ~i \in \lbrace 1,\ldots,N_{step} \rbrace$
		%Output
		\Ensure Deflektometrische Registrierung der Spalten \acrshort{lrx}
		
		\Statex		
		
		\Procedure {Bildaufnahme}{$N_{step}$, $N_{shift}$, $N_\lambda^i$, \acrshort{lwidth}}
			\For {Alle Stufen $i \gets 1$ \textbf{to} $N_{step}$}
				\For {Alle Phasenverschiebungen $k \gets 1$ \textbf{to} $N_{shift}$}
					\State Erzeugung des Musters $m_k^i$ nach Gleichung \ref{eq:monitormuster_mehrstufig}
					\State Anzeige des Musters $m_k^i$ auf geeignetem Bildschirm
					\State Aufnahme des Bildes mit Kamera $\rightarrow g_k^i$ nach Gleichung \ref{eq:kamerabild_mehrstufig}
				\EndFor
			\EndFor
			\State \Return Musterbilder $g_k^i$
		\EndProcedure
		
		\Statex
		
		\Procedure {DeflektometrischeRegistrierung}{$N_{step}$, $N_{shift}$, $N_\lambda^i$, \acrshort{lwidth}, $g_k^i$}
			\State // Überprüfung ob eindeutige Berechnung möglich ist:
			\If {$ggT(N_\lambda^1, \ldots, N_\lambda^{N_{step}}) = 1$}
				\State // Deflektometrische Registrierung ist eindeutig zu berechnen
				\For {Alle Kamerapixel $(x_B, y_B) \in \text{\acrshort{d}}(g_k^i$)}
					\For {Alle Stufen $i \gets 1$ \textbf{to} $N_{step}$}
						\State Berechnung von $x_{L,relativ}^i$ nach Gleichung \ref{eq:registrierungX_relativ}
					\EndFor
					\State Lösung des Optimierungsproblems aus Gleichung \ref{eq:optimierungsProblem_mehrstufig} $\rightarrow (n^1,\ldots, n^{N_{step}})$
					\State Berechnung von $x_L$ nach Gleichung \ref{eq:registrierungX_mehrstufig} $\rightarrow \text{\acrshort{lrx}}(x_B, y_B)$
				\EndFor
				\State \Return Deflektometrische Registrierung der Spalten \acrshort{lrx}
			\Else
				\State // Deflektometrische Registrierung ist nicht eindeutig zu berechnen
				\State \Return
			\EndIf
		\EndProcedure
	\end{algorithmic}
\end{algorithm}
}

\noindent
Der Algorithmus beschreibt die Berechnung der deflektometrischen Registrierung der Spaltenpositionen \acrshort{lrx}.
Analog lässt sich auch die deflektometrische Registrierung der Zeilenpositionen \acrshort{lry} bestimmen.
Nach Satz \ref{theo:separierbarkeitDeflektometrischeRegistrierung} kann schließlich aus den separierten deflektometrischen Registrierungen der Spalten- und Zeilenpositionen \acrshort{lrx} und \acrshort{lry} die deflektometrische Registrierung \acrshort{lr} des Prüfsystems gebildet werden.