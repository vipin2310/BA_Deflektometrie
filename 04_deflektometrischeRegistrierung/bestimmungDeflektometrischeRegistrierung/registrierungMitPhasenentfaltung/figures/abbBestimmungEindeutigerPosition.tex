\begin{adjustbox}{width=\textwidth}
	\begin{tikzpicture}
	
		% Koordinatensystem 1
		\draw[thick,-stealth,black] (0,0)--(16,0) node[below] {$x_L$};
		\draw[thick,-stealth,black] (0,0)--(0,4.25) node[left] {$m_1^1(x_L,y_0)$};
		\draw[thick,black] (0,0) -- (0,-0.1) node[anchor=north,fill=white] {$0$};
		\draw[thick,black] (15,0) -- (15,-0.1) node[anchor=north,fill=white] {\acrshort{lwidth}};
		\draw[thick,black] (0,0) -- (-0.1,0) node[anchor=east,fill=white] {$0$};
		\draw[thick,black] (0,3.75) -- (-0.1,3.75) node[anchor=east,fill=white] {$255$};
		
		% Koordinatensystem 2
		\draw[thick,-stealth,black] (0,-5.5)--(16,-5.5) node[below] {$x_L$};
		\draw[thick,-stealth,black] (0,-5.5)--(0,-1.25) node[left] {$m_1^2(x_L,y_0)$};
		\draw[thick,black] (0,-5.5) -- (0,-5.6) node[anchor=north,fill=white] {$0$};
		\draw[thick,black] (15,-5.5) -- (15,-5.6) node[anchor=north,fill=white] {\acrshort{lwidth}};
		\draw[thick,black] (0,-5.5) -- (-0.1,-5.5) node[anchor=east,fill=white] {$0$};
		\draw[thick,black] (0,-1.75) -- (-0.1,-1.75) node[anchor=east,fill=white] {$255$};
		
		% Koordinatensystem 3
		\draw[thick,-stealth,black] (0,-11)--(16,-11) node[below] {$x_L$};
		\draw[thick,-stealth,black] (0,-11)--(0,-6.75) node[left] {$m_1^3(x_L,y_0)$};
		\draw[thick,black] (0,-11) -- (0,-11.1) node[anchor=north,fill=white] {$0$};
		\draw[thick,black] (15,-11) -- (15,-11.1) node[anchor=north,fill=white] {\acrshort{lwidth}};
		\draw[thick,black] (0,-11) -- (-0.1,-11) node[anchor=east,fill=white] {$0$};
		\draw[thick,black] (0,-7.25) -- (-0.1,-7.25) node[anchor=east,fill=white] {$255$};
		
		% Funktion 1
		\draw[name path = func1, blue, thick, domain=0:15, samples=600] plot (\x,{1.875*cos(((2*pi*3)/15)*\x r) + 1.875}) node[anchor=north west] {$N_p^1 = 3$};
		
		% Funktion 2
		\draw[name path = func2, blue, thick, domain=0:15, samples=600] plot (\x,{1.875*cos(((2*pi*4)/15)*\x r) - 3.625}) node[anchor=north west] {$N_p^2 = 4$};
		
		% Funktion 3
		\draw[name path = func3, blue, thick, domain=0:15, samples=600] plot (\x,{1.875*cos(((2*pi*6)/15)*\x r) - 9.125}) node[anchor=north west] {$N_p^3 = 6$};
	
		% Perioden Funktion 1
		\draw[dashed,blue] (5,-0.5) -- (5,4.25);
		\draw[dashed,blue] (10,-0.5) -- (10,4.25);
		\node[anchor=north] at (2.5,4.25) {$\color{violet} \alpha^1 = 0$};
		\node[anchor=north] at (7.5,4.25) {$\color{red} \alpha^1 = 1$};
		\node[anchor=north] at (12.5,4.25) {$\color{cyan} \alpha^1 = 2$};
		
		% Perioden Funktion 2
		\draw[dashed,blue] (3.75,-6) -- (3.75,-1.25);
		\draw[dashed,blue] (7.5,-6) -- (7.5,-1.25);
		\draw[dashed,blue] (11.25,-6) -- (11.25,-1.25);
		\node[anchor=north] at (1.875,-1.25) {$\color{violet} \alpha^2 = 0$};
		\node[anchor=north] at (5.625,-1.25) {$\color{red} \alpha^2 = 1$};
		\node[anchor=north] at (9.375,-1.25) {$\color{cyan} \alpha^2 = 2$};
		\node[anchor=north] at (13.125,-1.25) {$\color{cyan} \alpha^2 = 3$};
		
		% Perioden Funktion 3
		\draw[dashed,blue] (2.5,-11.5) -- (2.5,-6.75);
		\draw[dashed,blue] (5,-11.5) -- (5,-6.75);
		\draw[dashed,blue] (7.5,-11.5) -- (7.5,-6.75);
		\draw[dashed,blue] (10,-11.5) -- (10,-6.75);
		\draw[dashed,blue] (12.5,-11.5) -- (12.5,-6.75);
		\node[anchor=north] at (1.25,-6.75) {$\color{violet} \alpha^3 = 0$};
		\node[anchor=north] at (3.75,-6.75) {$\color{cyan} \alpha^3 = 1$};
		\node[anchor=north] at (6.25,-6.75) {$\color{red} \alpha^3 = 2$};
		\node[anchor=north] at (8.75,-6.75) {$\color{cyan} \alpha^3 = 3$};
		\node[anchor=north] at (11.25,-6.75) {$\color{cyan} \alpha^3 = 4$};
		\node[anchor=north] at (13.75,-6.75) {$\color{cyan} \alpha^3 = 5$};
		
		% Absolute x_L Linie
		\draw[name path = absoluteLine, ultra thick, red] (6.9,4.25) -- (6.9,-11.1) node[anchor=north,fill=white] {$x_{L}$};
		
		% Punkte x_L absolut
		\path [name intersections={of=func1 and absoluteLine,by=a1}];
		\path [name intersections={of=func2 and absoluteLine,by=a2}];
		\path [name intersections={of=func3 and absoluteLine,by=a3}];
		
		% Punkte x_L relativ
		\coordinate (r1) at (1.9,{1.875*cos(((2*pi*3)/15)*1.9 r) + 1.875});
		\coordinate (r2) at (3.15,{1.875*cos(((2*pi*4)/15)*3.15 r) - 3.625});
		\coordinate (r3) at (1.9,{1.875*cos(((2*pi*6)/15)*1.9 r) - 9.125});
		
		% Horizontale Linien
		\draw[violet, thick, shorten <= -2cm, shorten >= -9.1cm] (r1) -- (a1);
		\draw[violet, thick, shorten <= -3.25cm, shorten >= -9.1cm] (r2) -- (a2);
		\draw[violet, thick, shorten <= -2cm, shorten >= -9.1cm] (r3) -- (a3);
		
		% Die Punkte für Funktion 1 zeichnen
		\draw[violet,fill=violet] (r1) circle (2.5pt);
		\draw[red,fill=red] (a1) circle (2.5pt);
		\draw[cyan,fill=cyan] (11.9,{1.875*cos(((2*pi*3)/15)*11.9 r) + 1.875}) circle (2.5pt);
		
		% Die Punkte für Funktion 2 zeichnen
		\draw[violet,fill=violet] (r2) circle (2.5pt);
		\draw[red,fill=red] (a2) circle (2.5pt);
		\draw[cyan,fill=cyan] (10.65,{1.875*cos(((2*pi*4)/15)*10.65 r) - 3.625}) circle (2.5pt);
		\draw[cyan,fill=cyan] (14.4,{1.875*cos(((2*pi*4)/15)*14.4 r) - 3.625}) circle (2.5pt);
		
		% Die Punkte für Funktion 3 zeichnen
		\draw[violet,fill=violet] (r3) circle (2.5pt);
		\draw[cyan,fill=cyan] ((4.4,{1.875*cos(((2*pi*6)/15)*4.4 r) - 9.125}) circle (2.5pt);
		\draw[red,fill=red] (a3) circle (2.5pt);
		\draw[cyan,fill=cyan] ((9.4,{1.875*cos(((2*pi*6)/15)*9.4 r) - 9.125}) circle (2.5pt);
		\draw[cyan,fill=cyan] ((11.9,{1.875*cos(((2*pi*6)/15)*11.9 r) - 9.125}) circle (2.5pt);
		\draw[cyan,fill=cyan] ((14.4,{1.875*cos(((2*pi*6)/15)*14.4 r) - 9.125}) circle (2.5pt);
		
		% x_L relativ Beschriftung an x-Achse
		\draw[thick,dashed,violet] (r1) -- (1.9,-0.1) node[anchor=north,fill=white] {$x_{L,relativ}^1$};
		\draw[thick,dashed,violet] (r2) -- (3.15,-5.6) node[anchor=north,fill=white] {$x_{L,relativ}^2$};
		\draw[thick,dashed,violet] (r3) -- (1.9,-11.1) node[anchor=north,fill=white] {$x_{L,relativ}^3$};
		
	\end{tikzpicture}
\end{adjustbox}
\caption[Bestimmung eindeutiger Position]{Bestimmung eindeutiger Spaltenposition $x_L$ durch Verwendung von unterschiedlichen Perioden bei fester Zeile $y_0$.}