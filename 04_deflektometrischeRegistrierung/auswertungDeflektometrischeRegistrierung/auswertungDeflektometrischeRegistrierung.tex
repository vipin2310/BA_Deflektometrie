Die deflektometrische Registrierung \acrshort{lr} kann ohne Weiteres nicht direkt ausgewertet werden.
Deshalb wird im Folgenden die Weiterverarbeitung der deflektometrischen Registrierung beschrieben, sodass bekannte Algorithmen aus dem Gebiet der Bildverarbeitung angewendet werden können.

\p
Die graphische Darstellung der deflektometrischen Registrierung \acrshort{lr} stellt sich zunächst als schwierig heraus, da man mit einer Abbildung der Form $\mathbb{R}^2 \rightarrow \mathbb{R}^2$ arbeitet.
Aus dem Grund wird die Separierbarkeit der deflektometrischen Registrierung aus Satz \ref{theo:separierbarkeitDeflektometrischeRegistrierung} angewendet.
Daraus erhält man die beiden Abbildungen der Form $\mathbb{R}^2 \rightarrow \mathbb{R}$:
%
\begin{equation*}
	\acrshortmath{lrx} : \mathbb{R}^{2} \supset A_{Cam} \rightarrow \mathbb{R} ,\quad (x_{B}, y_{B}) \mapsto x_{L}
\end{equation*}
%
\begin{equation*}
	\acrshortmath{lry} : \mathbb{R}^{2} \supset A_{Cam} \rightarrow \mathbb{R} ,\quad (x_{B}, y_{B}) \mapsto y_{L}
\end{equation*}
%
In der Form lässt sich die Analogie zu der mathematischen Beschreibung eines Graubildes $f$ erkennen:
%
\begin{equation*}
	f : \mathbb{R}^{2} \supseteq [x_{min},x_{max}] \times [y_{min},y_{max}] \rightarrow [I_{min},I_{max}] \subseteq \mathbb{R} ,\quad (x,y) \mapsto f(x,y)
\end{equation*}
%
Für die Darstellung als Bilder sind somit lediglich geeignete Transformationen der Wertemengen der deflektometrischen Registrierungen \acrshort{lrx} und \acrshort{lry} nötig.
%
\begin{Definition}{Darstellung der Deflektometrischen Registrierung}{def:graphDeflektometrischeRegistrierung}
	Die deflektometrischen Registrierung \acrshort{lr} kann als zwei einzelne Bilder \acrshort{frx} und \acrshort{fry} dargestellt werden.
	%	
	\begin{equation*}
		\acrshortmath{frx} : \mathbb{R}^2 \supset \acrshortmath{d}(\acrshortmath{lrx}) \rightarrow [I_{min},I_{max}] \subseteq \mathbb{R}
	\end{equation*}
	%
	\begin{equation*}
		\acrshortmath{fry} : \mathbb{R}^2 \supset \acrshortmath{d}(\acrshortmath{lry}) \rightarrow [I_{min},I_{max}] \subseteq \mathbb{R}
	\end{equation*}
	%
	Dabei lassen sich die Bilder \acrshort{frx} und \acrshort{fry} schreiben als:
	%	
	\begin{equation*}
		\acrshortmath{frx}(x,y) = t_x(\acrshortmath{lrx}(x,y))
	\end{equation*}
	%	
	\begin{equation*}
		\acrshortmath{fry}(x,y) = t_y(\acrshortmath{lry}(x,y))
	\end{equation*}
	%
	Es gilt $\acrshortmath{d}(\acrshortmath{frx}) = \acrshortmath{d}(\acrshortmath{lrx})$ und $\acrshortmath{d}(\acrshortmath{fry}) = \acrshortmath{d}(\acrshortmath{lry})$.
	Die Abbildungen $t_x$ und $t_y$ sind dabei lineare Transformationen der Wertemengen der deflektometrischen Abbildungen in Spalten und Zeilen zu den zulässigen Intensitäten für die Bilder \acrshort{frx} und \acrshort{fry}, angegeben durch das Intervall $[I_{min},I_{max}]$.
	%
	\begin{equation*}
		t_x : \mathbb{R} \supset \acrshortmath{w}(\acrshortmath{lry}) \rightarrow [I_{min},I_{max}] \subseteq \mathbb{R}
	\end{equation*}
	%
	\begin{equation*}
		t_y : \mathbb{R} \supset \acrshortmath{w}(\acrshortmath{lry}) \rightarrow [I_{min},I_{max}] \subseteq \mathbb{R}
	\end{equation*}
	%
	Die Transformationen $t_x$ und $t_y$ lassen sich schreiben als:
	%	
	\begin{equation*}
		t_x(x) = \left(\dfrac{x}{\acrshortmath{lwidth}}(I_{max} - I_{min})\right) + I_{min}
	\end{equation*}
	%	
	\begin{equation*}
		t_y(y) = \left(\dfrac{y}{\acrshortmath{lheight}}(I_{max} - I_{min})\right) + I_{min}
	\end{equation*}
	%
\end{Definition}
%
Erstellt man aus der berechneten deflektometrischen Registrierung \acrshort{lr} einer ungekrümmten Fläche die zugehörigen Bilder \acrshort{frx} und \acrshort{fry} nach Definition \ref{def:graphDeflektometrischeRegistrierung}, erhält man Darstellungen wie in Abbildung \ref{tikz:abbOptimaleSpaltenZeilenReg}:

% Abbildung: Optimale Spalten- und Zeilenregistrierung
{
	\begin{figure}[H]
		\centering
		\begin{adjustbox}{width=\textwidth}
	\begin{tikzpicture}[every node/.style={inner sep=0,outer sep=0}]
	
		\node [anchor=north west] (imgSpalten) at (0,0) {\includegraphics[width=.47\textwidth]{04_deflektometrischeRegistrierung/auswertungDeflektometrischeRegistrierung/figures/spaltenRegistrierung_optimal}};
		\node [below=0.2cm of imgSpalten] {Graubild der Spaltenzuordnung \acrshort{frx}$(x,y)$};
		\node [anchor=north west] (imgZeilen) at (0.53\textwidth,0) {\includegraphics[width=.47\textwidth]{04_deflektometrischeRegistrierung/auswertungDeflektometrischeRegistrierung/figures/zeilenRegistrierung_optimal}};
		\node [below=0.2cm of imgZeilen] {Graubild der Zeilenzuordnung \acrshort{fry}$(x,y)$};
	
	\end{tikzpicture}
\end{adjustbox}
\caption[Darstellung Spalten- und Zeilenregistrierung]{Darstellung der Spalten- und Zeilenregistrierung als Bilder in Graustufen mit $I_{min} = 0$ und $I_{max} = 255$. Je dunkler ein Pixel ist, desto weiter links bzw. oben befindet sich die zugeordnete Spalten- bzw. Zeilenposition.}
		\label{tikz:abbOptimaleSpaltenZeilenReg}
	\end{figure}
}

\noindent
In Abbildung \ref{tikz:abbOptimaleSpaltenZeilenReg} wird direkt das Muster auf dem Monitor betrachtet.
Aus dem Grund lässt sich erkennen, dass die Zuordnung von Monitor- und Kamerapixeln in den Spalten und Zeilen linear verläuft.
Werden nun die Streifen durch besondere Oberflächeneigenschaften gekrümmt oder verzerrt, dann werden an diesen Stellen in den Bildern der deflektometrischen Registrierung Abweichungen vom linearen Grauwerteverlauf sichtbar.

% Abbildung: Verzerrung der Spalten- und Zeilenregistrierung durch Einkerbungen
{
	\begin{figure}[H]
		\centering
		\begin{adjustbox}{width=\textwidth}
	\begin{tikzpicture}[every node/.style={inner sep=0,outer sep=0}]
	
		\node [anchor=north west] (imgSpalten) at (0,0) {\includegraphics[width=.47\textwidth]{04_deflektometrischeRegistrierung/auswertungDeflektometrischeRegistrierung/figures/streifenKrümmung_spalten}};
		\node [below=0.2cm of imgSpalten] {Graubild der Spaltenzuordnung \acrshort{frx}$(x,y)$};
		\node [anchor=north west] (imgZeilen) at (0.53\textwidth,0) {\includegraphics[width=.47\textwidth]{04_deflektometrischeRegistrierung/auswertungDeflektometrischeRegistrierung/figures/streifenKrümmung_zeilen}};
		\node [below=0.2cm of imgZeilen] {Graubild der Zeilenzuordnung \acrshort{fry}$(x,y)$};
	
	\end{tikzpicture}
\end{adjustbox}
\caption[Darstellung verzerrter Spalten- und Zeilenregistrierung]{Darstellung verzerrter Spalten- und Zeilenregistrierung als Bilder. Verzerrungen entstehen durch tiefe Eingravierungen im Glas.}
		\label{tikz:abbBrillenglasRegistrierung}
	\end{figure}
}

\noindent
Die deflektometrische Registrierung macht bestimmte Fehlstellen in Abbildung \ref{tikz:abbBrillenglasRegistrierung} kenntlich.
Diese Fehlstellen sind allerdings nur tiefe Eingravierungen, durch die die Phase des Streifenmusters lokal deformiert wird.
Normale Kratzer beeinflussen besonders den gemessenen Grauwert.
Die zugeordnete Monitorposition hingegen wird durch solche Kratzer nur geringfügig verändert, weshalb diese im Bild der deflektometrischen Registrierung kaum erkennbar sind.
Besser funktioniert die Fehlstellenerkennung, indem man die Reflexionen bzw. die Spiegelbilder der Muster aufnimmt.
So ist es z. B. möglich, wie in Abbildung \ref{tikz:abbRegistrierungDelle}, Dellen und Beulen auf reflektierenden Porzellanoberflächen durch die deflektometrische Registrierung deutlich hervorzuheben.

% Abbildung: Objekt mit Delle
{
	\begin{figure}[H]
		\centering
		\includegraphics[width = 0.47\textwidth]{04_deflektometrischeRegistrierung/auswertungDeflektometrischeRegistrierung/figures/delleBeleuchtet}
		\caption[Spiegelndes Porzellanbruchstück mit Delle]{Spiegelndes Porzellanbruchstück mit Delle.}
		\label{img:objektMitDelle}
	\end{figure}
}

% Abbildung: Deflektometrische Registrierung bei Objekt mit Delle
{
	\begin{figure}[H]
		\centering
		\begin{adjustbox}{width=\textwidth}
	\begin{tikzpicture}[every node/.style={inner sep=0,outer sep=0}]
	
		\node [anchor=north east] (imgSpalten) at (-0.03\textwidth,0) {\includegraphics[width=.47\textwidth]{04_deflektometrischeRegistrierung/auswertungDeflektometrischeRegistrierung/figures/spaltenRegistrierung_Delle}};
		\node [below=0.2cm of imgSpalten] {Graubild der Spaltenzuordnung \acrshort{frx}$(x,y)$};
		\node [anchor=north west] (imgZeilen) at (0.03\textwidth,0) {\includegraphics[width=.47\textwidth]{04_deflektometrischeRegistrierung/auswertungDeflektometrischeRegistrierung/figures/zeilenRegistrierung_Delle}};
		\node [below=0.2cm of imgZeilen] {Graubild der Zeilenzuordnung \acrshort{fry}$(x,y)$};
		
	\end{tikzpicture}
\end{adjustbox}
\caption[Deflektometrische Registrierung bei Delle]{Deflektometrische Registrierung des spiegelnden Keramikobjekts aus Abbildung \ref{img:objektMitDelle}.}
		\label{tikz:abbRegistrierungDelle}
	\end{figure}
}

\noindent
In Abbildung \ref{tikz:abbRegistrierungDelle} entsteht im Hintergrund um das Objekt herum eine Störumgebung.
Der Grund dafür ist die fehlende Reflexion des Musters und somit ähnliche Grauwerte in den phasenverschobenen Bildern.
Dies führt dazu, dass in der Bestimmung der Phase $\phi$ für solche Pixel numerisch instabile Ausdrücke und somit beliebige Werte vorkommen.

\p
Die resultierenden Bilder können durch herkömmliche Verfahren aus der Bildverarbeitung weiterverarbeitet und analysiert werden.
Durch die weichen Grauwertverläufe an gleichmäßig gekrümmten Oberflächen in den \acrshort{frx} und \acrshort{fry}, lassen sich die Bilder effizient über Gradienten analysieren (siehe Abbildung \ref{tikz:abbGradientenbildReg}).
Abrupte Änderungen der Grauwerte innerhalb des spiegelnden Objekts führen zu höheren Gradienten als in der Umgebung. Fehlstellen wie z. B. Dellen oder Pickel lassen sich damit gut detektieren.
Aus demselben Grund erweisen sich auch Hochpassfilterungen als hilfreich \cite{kit_werling}.

% Abbildung: Gradientenbild der deflektometrischen Registrierung.
{
	\begin{figure}[H]
		\centering
		\begin{adjustbox}{width=\textwidth}
	\begin{tikzpicture}[every node/.style={inner sep=0,outer sep=0}]
	
		\node [anchor=north east] (imgSpalten) at (-0.03\textwidth,0) {\includegraphics[width=.47\textwidth]{04_deflektometrischeRegistrierung/auswertungDeflektometrischeRegistrierung/figures/pickelDeflektometrischeRegistrierung}};
		\node [below=0.2cm of imgSpalten] {Graubild der Spaltenzuordnung \acrshort{frx}$(x,y)$};
		\node [anchor=north west] (imgGradienten) at (0.03\textwidth,0) {\includegraphics[width=.47\textwidth]{04_deflektometrischeRegistrierung/auswertungDeflektometrischeRegistrierung/figures/pickelGradientenbild}};
		\node [below=0.2cm of imgGradienten, align = center] {Gradientenbild von \acrshort{frx}$(x,y)$ über den \\ Sobel-Operator in $x$-Richtung};
		
	\end{tikzpicture}
\end{adjustbox}
\caption[Hervorhebung von Pickeln auf reflektierenden Oberflächen.]{Deflektometrische Spaltenregistrierung eines spiegelnden Porzellanbruch\-stücks und das Gradientenbild.}
		\label{tikz:abbGradientenbildReg}
	\end{figure}
}

\noindent
Das untersuchte Objekt aus Abbildung \ref{tikz:abbGradientenbildReg} weist am linken Rand vereinzelt matte Stellen auf.
Dadurch treten im Bild der deflektometrischen Registrierung \acrshort{frx} einzelne dunkle Punkte am linken Rand auf.
Diese erkennt man somit auch im Gradientenbild von \acrshort{frx} als Fehlstellen.

%TODO Überprüfen ob das stimmt. (Ich glaube schon)
\p
Betrachtet man die Änderung der deflektometrischen Registrierung in spezielle Richtungen, wie z. B. über das Gradientenbild in $x$-Richtung aus Abbildung \ref{tikz:abbGradientenbildReg}, kann man direkt die wahrgenommene Krümmung in dieser Richtung beobachten.
Die wahrgenommene Krümmung steht in direktem Zusammenhang mit der zweiten Ableitung der Objektoberfläche \cite{kit_werling}.
Man erkennt in solchen Gradientenbildern an helleren Stellen treten größere Änderungen der Spaltenpositionen 