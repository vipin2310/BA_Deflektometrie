Die deflektometrische Registrierung \acrshort{lr} kann ohne Weiteres nicht direkt ausgewertet werden.
Deshalb wird im Folgenden die Weiterverarbeitung der deflektometrischen Registrierung beschrieben, sodass bekannte Algorithmen aus dem Gebiet der Bildverarbeitung angewendet werden können.

\p
Die graphische Darstellung der deflektometrischen Registrierung \acrshort{lr} stellt sich zunächst als schwierig heraus, da man mit einer Abbildung der Form $\mathbb{R}^2 \rightarrow \mathbb{R}^2$ arbeitet.
Aus dem Grund wird die Separierbarkeit der deflektometrischen Registrierung aus Satz \ref{theo:separierbarkeitDeflektometrischeRegistrierung} angewendet.
Daraus erhält man die beiden Abbildungen der Form $\mathbb{R}^2 \rightarrow \mathbb{R}$:
%
\begin{equation*}
	\acrshortmath{lrx} : \mathbb{R}^{2} \supset A_{Cam} \rightarrow \mathbb{R} ,\quad (x_{B}, y_{B}) \mapsto x_{L}
\end{equation*}
%
\begin{equation*}
	\acrshortmath{lry} : \mathbb{R}^{2} \supset A_{Cam} \rightarrow \mathbb{R} ,\quad (x_{B}, y_{B}) \mapsto y_{L}
\end{equation*}
%
In der Form lässt sich die Analogie zu der mathematischen Beschreibung eines Graubildes $f$ erkennen:
%
\begin{equation*}
	f : \mathbb{R}^{2} \supseteq [x_{min},x_{max}] \times [y_{min},y_{max}] \rightarrow [I_{min},I_{max}] \subseteq \mathbb{R} ,\quad (x,y) \mapsto f(x,y)
\end{equation*}
%
Für die Darstellung als Bilder sind somit lediglich geeignete Transformationen der Wertemengen der deflektometrischen Registrierungen \acrshort{lrx} und \acrshort{lry} nötig.
%
\begin{Definition}{Darstellung der Deflektometrischen Registrierung}{def:graphDeflektometrischeRegistrierung}
	Die deflektometrischen Registrierung \acrshort{lr} kann als zwei einzelne Bilder \acrshort{frx} und \acrshort{fry} dargestellt werden.
	%	
	\begin{equation*}
		\acrshortmath{frx} : \mathbb{R}^2 \supset \acrshortmath{d}(\acrshortmath{lrx}) \rightarrow [I_{min},I_{max}] \subseteq \mathbb{R}
	\end{equation*}
	%
	\begin{equation*}
		\acrshortmath{fry} : \mathbb{R}^2 \supset \acrshortmath{d}(\acrshortmath{lry}) \rightarrow [I_{min},I_{max}] \subseteq \mathbb{R}
	\end{equation*}
	%
	Dabei hat lassen sich die Bilder \acrshort{frx} und \acrshort{fry} schreiben als:
	%	
	\begin{equation*}
		\acrshortmath{frx}(x,y) = t_x(\acrshortmath{lrx}(x,y))
		\qquad
	\end{equation*}
	%	
	\begin{equation*}
		\acrshortmath{fry}(x,y) = t_y(\acrshortmath{lry}(x,y))
	\end{equation*}
	%
	Es gilt $\acrshortmath{d}(\acrshortmath{frx}) = \acrshortmath{d}(\acrshortmath{lrx})$ und $\acrshortmath{d}(\acrshortmath{fry}) = \acrshortmath{d}(\acrshortmath{lry})$.
	Die Abbildungen $t_x$ und $t_y$ sind dabei lineare Transformationen der Wertemengen der deflektometrischen Abbildungen in Spalten und Zeilen zu den zulässigen Intensitäten für die Bilder \acrshort{frx} und \acrshort{fry}, angegeben durch das Intervall $[I_{min},I{max}]$.
	%
	\begin{equation*}
		t_x : \mathbb{R} \supset \acrshortmath{w}(\acrshortmath{lry}) \rightarrow [I_{min},I_{max}] \subseteq \mathbb{R}
	\end{equation*}
	%
	\begin{equation*}
		t_y : \mathbb{R} \supset \acrshortmath{w}(\acrshortmath{lry}) \rightarrow [I_{min},I_{max}] \subseteq \mathbb{R}
	\end{equation*}
	%
	Die Transformationen $t_x$ und $t_y$ lassen sich schreiben als:
	
\end{Definition}