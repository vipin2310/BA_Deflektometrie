Die deflektometrische Registrierung \acrshort{lr} kann ohne Weiteres nicht direkt ausgewertet werden.
Deshalb wird im Folgenden die Weiterverarbeitung der deflektometrischen Registrierung beschrieben, sodass bekannte Algorithmen aus dem Gebiet der Bildverarbeitung angewendet werden können.

\p
Die graphische Darstellung der deflektometrischen Registrierung \acrshort{lr} stellt sich zunächst als schwierig heraus, da man mit einer Abbildung der Form $\mathbb{R}^2 \rightarrow \mathbb{R}^2$ arbeitet.
Aus dem Grund wird die Separierbarkeit der deflektometrischen Registrierung aus Satz \ref{theo:separierbarkeitDeflektometrischeRegistrierung} angewendet.
Daraus erhält man die beiden Abbildungen der Form $\mathbb{R}^2 \rightarrow \mathbb{R}$:
%
\begin{equation*}
	\acrshortmath{lrx} : \mathbb{R}^{2} \supset A_{Cam} \rightarrow \mathbb{R} ,\quad (x_{B}, y_{B}) \mapsto x_{L}
\end{equation*}
%
\begin{equation*}
	\acrshortmath{lry} : \mathbb{R}^{2} \supset A_{Cam} \rightarrow \mathbb{R} ,\quad (x_{B}, y_{B}) \mapsto y_{L}
\end{equation*}
%
In der Form lässt sich die Analogie zu der mathematischen Beschreibung eines Graubildes $f$ erkennen:
%
\begin{equation*}
	f : \mathbb{R}^{2} \supseteq [x_{min},x_{max}] \times [y_{min},y_{max}] \rightarrow [I_{min},I_{max}] \subseteq \mathbb{R} ,\quad (x,y) \mapsto f(x,y)
\end{equation*}
%
Für die Darstellung als Bilder sind somit lediglich geeignete Transformationen der Wertemengen der deflektometrischen Registrierungen \acrshort{lrx} und \acrshort{lry} nötig.
%
\begin{Definition}{Darstellung der Deflektometrischen Registrierung}{def:graphDeflektometrischeRegistrierung}
	Die deflektometrischen Registrierung \acrshort{lr} kann als zwei einzelne Bilder \acrshort{frx} und \acrshort{fry} dargestellt werden.
	%	
	\begin{equation*}
		\acrshortmath{frx} : \mathbb{R}^2 \supset \acrshortmath{d}(\acrshortmath{lrx}) \rightarrow [I_{min},I_{max}] \subseteq \mathbb{R}
	\end{equation*}
	%
	\begin{equation*}
		\acrshortmath{fry} : \mathbb{R}^2 \supset \acrshortmath{d}(\acrshortmath{lry}) \rightarrow [I_{min},I_{max}] \subseteq \mathbb{R}
	\end{equation*}
	%
	Dabei lassen sich die Bilder \acrshort{frx} und \acrshort{fry} schreiben als:
	%	
	\begin{equation*}
		\acrshortmath{frx}(x,y) = t_x(\acrshortmath{lrx}(x,y))
	\end{equation*}
	%	
	\begin{equation*}
		\acrshortmath{fry}(x,y) = t_y(\acrshortmath{lry}(x,y))
	\end{equation*}
	%
	Es gilt $\acrshortmath{d}(\acrshortmath{frx}) = \acrshortmath{d}(\acrshortmath{lrx})$ und $\acrshortmath{d}(\acrshortmath{fry}) = \acrshortmath{d}(\acrshortmath{lry})$.
	Die Abbildungen $t_x$ und $t_y$ sind dabei lineare Transformationen der Wertemengen der deflektometrischen Abbildungen in Spalten und Zeilen zu den zulässigen Intensitäten für die Bilder \acrshort{frx} und \acrshort{fry}, angegeben durch das Intervall $[I_{min},I_{max}]$.
	%
	\begin{equation*}
		t_x : \mathbb{R} \supset \acrshortmath{w}(\acrshortmath{lry}) \rightarrow [I_{min},I_{max}] \subseteq \mathbb{R}
	\end{equation*}
	%
	\begin{equation*}
		t_y : \mathbb{R} \supset \acrshortmath{w}(\acrshortmath{lry}) \rightarrow [I_{min},I_{max}] \subseteq \mathbb{R}
	\end{equation*}
	%
	Die Transformationen $t_x$ und $t_y$ lassen sich schreiben als:
	%	
	\begin{equation*}
		t_x(x) = \left(\dfrac{x}{\acrshortmath{lwidth}}(I_{max} - I_{min})\right) + I_{min}
	\end{equation*}
	%	
	\begin{equation*}
		t_y(y) = \left(\dfrac{y}{\acrshortmath{lheight}}(I_{max} - I_{min})\right) + I_{min}
	\end{equation*}
	%
\end{Definition}
%
Erstellt man aus der berechneten deflektometrischen Registrierung \acrshort{lr} einer ungekrümmten Fläche die zugehörigen Bilder \acrshort{frx} und \acrshort{fry} nach Definition \ref{def:graphDeflektometrischeRegistrierung}, erhält man Darstellungen wie in Abbildung \ref{tikz:abbOptimaleSpaltenZeilenReg}:

% Abbildung: Optimale Spalten- und Zeilenregistrierung
{
	\begin{figure}[H]
		\centering
		\begin{adjustbox}{width=\textwidth}
	\begin{tikzpicture}[every node/.style={inner sep=0,outer sep=0}]
	
		\node [anchor=north west] (imgSpalten) at (0,0) {\includegraphics[width=.47\textwidth]{04_deflektometrischeRegistrierung/auswertungDeflektometrischeRegistrierung/figures/spaltenRegistrierung_optimal}};
		\node [below=0.2cm of imgSpalten] {Graubild der Spaltenzuordnung \acrshort{frx}$(x,y)$};
		\node [anchor=north west] (imgZeilen) at (0.53\textwidth,0) {\includegraphics[width=.47\textwidth]{04_deflektometrischeRegistrierung/auswertungDeflektometrischeRegistrierung/figures/zeilenRegistrierung_optimal}};
		\node [below=0.2cm of imgZeilen] {Graubild der Zeilenzuordnung \acrshort{fry}$(x,y)$};
	
	\end{tikzpicture}
\end{adjustbox}
\caption[Darstellung Spalten- und Zeilenregistrierung]{Darstellung der Spalten- und Zeilenregistrierung als Bilder in Graustufen mit $I_{min} = 0$ und $I_{max} = 255$. Je dunkler ein Pixel ist, desto weiter links bzw. oben befindet sich die zugeordnete Spalten- bzw. Zeilenposition.}
		\label{tikz:abbOptimaleSpaltenZeilenReg}
	\end{figure}
}

\noindent
In Abbildung \ref{tikz:abbOptimaleSpaltenZeilenReg} wird direkt das Muster auf dem Monitor betrachtet.
Aus dem Grund lässt sich erkennen, dass die Zuordnung von Monitor- und Kamerapixeln in den Spalten und Zeilen linear verläuft.
Werden nun die Streifen durch besondere Oberflächeneigenschaften gekrümmt oder verzerrt, dann werden an diesen Stellen in den Bildern der deflektometrischen Registrierung Abweichungen vom linearen Grauwerteverlauf sichtbar.

% Abbildung: Verzerrung der Spalten- und Zeilenregistrierung durch Einkerbungen
{
	\begin{figure}[H]
		\centering
		\input{04_deflektometrischeRegistrierung/auswertungDeflektometrischeRegistrierung/figures/abbVerzerrungRegistrierung}
		\label{tikz:abbVerzerrungRegistrierung}
	\end{figure}
}

\noindent
%TODO Über die Porzellanobjekte schreiben.
%
Diese resultierenden Bilder können durch herkömmliche Verfahren aus der Bildverarbeitung weiterverarbeitet und analysiert werden.
Als hilfreich erweist sich dabei die Analyse von Gradientenbildern oder einer Hochpassfilterung zur Hervorhebung von lokalen Defekten\cite{kit_werling}.