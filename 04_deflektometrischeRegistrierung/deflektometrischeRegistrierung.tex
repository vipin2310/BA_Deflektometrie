\begin{Definition}{Deflektometrische Registrierung}{def:deflektometrischeRegistrierung}
	Die \textit{deflektometrische Registrierung} \acrshort{lr} beschreibt die Abbildung von Bildpunkten zu Schirmpunkten. \cite{kit_werling}
	%
	\begin{equation*}
		\acrshortmath{lr} : A_{Cam} \rightarrow L \cup \varnothing ,\quad P_{B} \rightarrow P_{L}
	\end{equation*}
	%
	$ A_{Cam} \subset \mathbb{R}^{2} $ bezeichnet die Menge der Bildpunkte bzw. der Kamerapixel.
	$ L \subset \mathbb{R}^{2} $ bezeichnet die Menge der Schirmpunkte bzw. der Monitorpixel.
	In Koordinatenschreibweise lässt sich die Abbildung somit darstellen als:
	%
	\begin{equation*}\label{eq:deflektometrischeRegistrierungAbbildung}
		\acrshortmath{lr} : \mathbb{R}^{2} \supset A_{Cam} \rightarrow \mathbb{R}^2 ,\quad (x_{B}, y_{B}) \mapsto (x_{L}, y_{L})
	\end{equation*}
\end{Definition}

\begin{Satz}{Separierbarkeit der deflektometrischen Registrierung}{theo:separierbarkeitDeflektometrischeRegistrierung}
Ferner kann die \textit{deflektometrische Registrierung} \acrshort{lr} in zwei Abbildungen von Bildpunkten zu den Spalten- und Zeilenpositionen der Schirmpunkte separiert werden:
	%
	\begin{equation*}
		\acrshortmath{lrx} : \mathbb{R}^{2} \supset A_{Cam} \rightarrow \mathbb{R} ,\quad (x_{B}, y_{B}) \mapsto x_{L}
	\end{equation*}
	%
	\begin{equation*}
		\acrshortmath{lry} : \mathbb{R}^{2} \supset A_{Cam} \rightarrow \mathbb{R} ,\quad (x_{B}, y_{B}) \mapsto y_{L}
	\end{equation*}
	%
\end{Satz}

\noindent
Die deflektometrische Registrierung kann aus der Beobachtung des Monitors durch die Kamera aufgestellt werden.
Wird der Monitor indirekt über eine spiegelnde Oberfläche beobachtet, können aus der deflektometrischen Registrierung Krümmungsmerkmale der spiegelnden Oberfläche bestimmt werden.
Durch die Kalibrierung der Positionen der Systemkomponenten können mittels der deflektometrischen Registrierung Höheninformationen über die Oberfläche gewonnen werden.
Der zugehörige Ansatz wurde im Abschnitt \ref{sub:rekonstruktion} bereits beschrieben.

%Bestimmung der deflektometrischen Registrierung
{
	\FloatBarrier
    \section{Bestimmung der deflektometrischen Registrierung}
    \label{sec:bestimmungDeflektometrischeRegistrierung}
    Die Bestimmung der deflektometrischen Registrierung nach Definition \ref{def:deflektometrischeRegistrierung} bedeutet die Zuordnung von Kamerapixeln zu Monitorpixeln.
Grundsätzlich soll jeder Lichtstrahl, der über das Prüfobjekt in die Kamera reflektiert wird im Kamerabild eindeutig zugeordnet werden können.
Eine einfache Möglichkeit eine solche Zuordnung zu erreichen bekommt man, indem man die Pixel des Monitors einzeln einschaltet und dabei die Veränderung im Kamerabild betrachtet.
Bei kleinen Bildern ist dies noch umsetzbar, allerdings wird die Anzahl von Pixeln mit zunehmender Auflösung schnell sehr groß und unübersichtlich, sodass dieser Ansatz nicht praktikabel ist.
Aus dem Grund ist es effektiver Bilder auf dem Monitor anzuzeigen, die Positionen visuell codieren können.
Durch die visuelle Erfassung des Spiegelbilds können somit die zugehörigen Positionen auf dem Monitor berechnet werden.

\p
Ein herkömmliches Kodierverfahren für solche Prozesse ist das Phasenschiebeverfahren.
Der Ansatz bei dem Verfahren ist es die Zeilen- und Spaltenpositionen des Monitors durch periodische Muster zu kodieren.
Die Grauwerte von periodischen Mustern nehmen dabei Werte an, die über die periodische Funktion bestimmt wurden.
Die hier verwendeten Funktionen sind Kosinusfunktionen.
Damit können die Pixel des Musters Phaseninformationen eines lokalen Orts übertragen.
Durch die Periodizität der Funktionen kann aus der reinen Phaseninformation noch nicht die genaue Monitorposition bestimmt werden.
Die Lösung dieses Problems wird als Phasenentfaltung bezeichnet (siehe Definition \ref{def:phasenentfaltung}).
Hierzu stellt Werling in seiner Arbeit \cite{kit_werling} ein mehrstufiges Phasenschiebeverfahren vor, das Muster mit unterschiedlichen Perioden verwendet.

\begin{Definition}{Phasenentfaltung}{def:phasenentfaltung}
	Die \textit{Phasenentfaltung} bezeichnet den Vorgang zur Auflösung der mehrdeutigen Zuordnung der Phaseninformation zu unterschiedlichen Perioden.
\end{Definition}

%Deflektometrische Registrierung ohne Phasenentfaltung
{
	\FloatBarrier
    \subsection{Deflektometrische Registrierung ohne Phasenentfaltung}
    \label{sub:registrierungOhnePhasenentfaltung}
    Ohne eine Phasenentfaltung bekommt man eventuell eine mehrdeutige Zuordnung aufgrund der Periodizität der Kosinusfunktion.
Soll eine eindeutige Zuordnung über das Phasenschiebeverfahren ohne Phasenentfaltung erfolgen, darf es also nur eine einzige Musterperiode auf der Monitorbreite bzw. Monitorhöhe geben.
Damit lässt sich die Kodierung der Monitorkoordinaten $(x_{L}, y_{L})$ durch die Phasen $(\phi_{x}, \phi_{y})$ folgendermaßen aufstellen:
%
\begin{equation}
	x_{L} = \dfrac{\text{\acrshort{lwidth}}}{2\pi}\phi_{x},
	\qquad
	y_{L} = \dfrac{\text{\acrshort{lheight}}}{2\pi}\phi_{y}
\end{equation}
%
\noindent
Die Monitorbreite wird dabei mit \acrshort{lwidth} und die Monitorhöhe mit \acrshort{lheight} angegeben.

\p
O.B.d.A. wird nachfolgend nur die deflektometrische Registrierung der Spaltenpositionen \acrshort{lrx} ($x$-Richtung) betrachtet.
Die deflektometrische Registrierung der Zeilenpositionen \acrshort{lry} ($y$-Richtung) kann analog bestimmt werden.
Das $k$-te Muster $m_k$ zur Kodierung der Monitorpunkte wird durch eine Kosinusfunktion aufgebaut und hat die Form:
%
\begin{equation}
	\begin{split}	
		m_k(x_L,y_L) = A_m \left(1 + C_m \cos \left(\dfrac{2\pi}{\text{\acrshort{lwidth}}} x_L + \psi_k\right)\right),
		\qquad
		k \in \lbrace 1,\ldots,N_{shift}\rbrace, \\
		\psi_k = (k - 1)\left(\dfrac{2\pi}{N_{shift}}\right)
	\end{split}
\end{equation}
%
$A_m$ bezeichnet die Amplitude, $C_m$ den Kontrast, $\psi_k$ die Phasenverschiebung des $k$-ten Musters und $N_{shift}$ die Anzahl an Mustern für das Phasenschiebeverfahren.
}

%Deflektometrische Registrierung mit Phasenentfaltung
{
	\FloatBarrier
    \subsection{Deflektometrische Registrierung mit Phasenentfaltung}
    \label{sub:registrierungMitPhasenentfaltung}
    Die hier beschriebene Methodik zur Bestimmung der deflektometrischen Registrierung ist ein mehrstufiges Phasenschiebeverfahren.
Ein solches Verfahren wird von Kammel in seiner Dissertation \cite{kit_kammel} vorgestellt.
Das Verfahren von Kammel zeigt jedoch in der Praxis Kodierungsartefakte bzw. Phasensprünge insbesondere an den Periodengrenzen.
Aus dem Grund stellt Werling darauf aufbauend in seiner Dissertation \cite{kit_werling} einen anderen Ansatz eines mehrstufigen Phasenschiebeverfahrens vor, der das Problem mit den Phasensprünge minimiert.
Die Idee hinter dem Ansatz ist dabei, dass man sich zunächst, analog zum Verfahren aus Kapitel \ref{sub:registrierungOhnePhasenentfaltung}, die $x$- bzw. $y$-Koordinaten in Relation zu den Perioden der Muster bestimmt.
Durch mehrere Muster mit unterschiedlichen Perioden erhält man schließlich mehrere relative unterschiedliche $x$- bzw. $y$-Koordinaten.
Diese müssen im finalen Ergebnis der richtigen Periode zugeordnet werden, indem der Abstand der Koordinaten minimiert wird.

\p
Analog zum Verfahren aus Kapitel \ref{sub:registrierungOhnePhasenentfaltung} benötigt man $N_{shift}$-viele Muster um die Phase eines Bildpunkts $(x_B, y_B)^\top$ zu bestimmen.
Zusätzlich betrachtet man mehrere Stufen des Verfahrens.
Das Musters auf der Stufe $i$ hat $N_\lambda^i$-viele Perioden über die Monitorbreite \acrshort{lwidth} bzw. -höhe \acrshort{lheight}.
Dabei sollen sich für unterschiedliche Stufen auch die Perioden der verwendeten Muster unterscheiden.
Die Phasen $\phi_x^i$ und $\phi_y^i$ der kodierten Monitorpunkte $(x_L, y_L)^\top$ auf der $i$-ten Stufe sehen dann folgendermaßen aus:
%
\begin{equation}
	\phi_x^i = \dfrac{2\pi N_\lambda^i}{\text{\acrshort{lwidth}}} x_L
	\qquad
	\phi_y^i = \dfrac{2\pi N_\lambda^i}{\text{\acrshort{lheight}}} y_L
\end{equation}

\p
O.B.d.A. wird nachfolgend nur die deflektometrische Registrierung der Spaltenpositionen \acrshort{lrx} ($x$-Richtung) betrachtet.
Die deflektometrische Registrierung der Zeilenpositionen \acrshort{lry} ($y$-Richtung) kann analog bestimmt werden.
Auf der Stufe $i$ hat das $k$-te Muster $m_k^i$ zur Kodierung der Monitorpunkte $(x_L, y_L)^\top$ somit die Form:
%
\begin{equation}\label{eq:monitormuster_mehrstufig}
	\begin{gathered}	
		m_k^i(x_L,y_L) = A_m^i \left(1 + C_m^i \cos \left(\phi_x^i + \psi_k\right)\right),\\
		k \in \lbrace 1,\ldots,N_{shift}\rbrace,
		\quad
		\psi_k = (k - 1)\dfrac{2\pi}{N_{shift}}
	\end{gathered}
\end{equation}
%
Es bezeichnet $A_m^i$ die Amplitude und $C_m^i$ den Kontrast des Musters der $i$-ten Stufe.
Wie auch in Kapitel \ref{sub:registrierungOhnePhasenentfaltung} entspricht $\psi_k$ der Phasenverschiebung des $k$-ten Musters der Stufen.
Analog zu Kapitel \ref{sub:registrierungOhnePhasenentfaltung} nimmt die Kamera das Signal $g_k^i$ auf:
%
\begin{equation}\label{eq:kamerabild}
	g_k^i(x_B, y_B) = A_g^i(x_B, y_B) \left(1 + C_g^i(x_B, y_B) \cos \left(\dfrac{2\pi N_\lambda^i}{\text{\acrshort{lwidth}}}\text{\acrshort{lrx}}(x_B, y_B) + \psi_k\right)\right)
\end{equation}
}
}

%Auswertung der deflektometrischen Registrierung
{
	\FloatBarrier
    \section{Auswertung der deflektometrischen Registrierung}
    \label{sec:auswertungDeflektometrischeRegistrierung}
    Die deflektometrische Registrierung \acrshort{lr} kann ohne Weiteres nicht direkt ausgewertet werden.
Deshalb wird im Folgenden die Weiterverarbeitung der deflektometrischen Registrierung beschrieben, sodass bekannte Algorithmen aus dem Gebiet der Bildverarbeitung angewendet werden können.

\p
Die graphische Darstellung der deflektometrischen Registrierung \acrshort{lr} stellt sich zunächst als schwierig heraus, da man mit einer Abbildung der Form $\mathbb{R}^2 \rightarrow \mathbb{R}^2$ arbeitet.
Aus dem Grund wird die Separierbarkeit der deflektometrischen Registrierung aus Satz \ref{theo:separierbarkeitDeflektometrischeRegistrierung} angewendet.
Daraus erhält man die beiden Abbildungen der Form $\mathbb{R}^2 \rightarrow \mathbb{R}$:
%
\begin{equation*}
	\acrshortmath{lrx} : \mathbb{R}^{2} \supset A_{Cam} \rightarrow \mathbb{R} ,\quad (x_{B}, y_{B}) \mapsto x_{L}
\end{equation*}
%
\begin{equation*}
	\acrshortmath{lry} : \mathbb{R}^{2} \supset A_{Cam} \rightarrow \mathbb{R} ,\quad (x_{B}, y_{B}) \mapsto y_{L}
\end{equation*}
%
In der Form lässt sich die Analogie zu der mathematischen Beschreibung eines Graubildes $f$ erkennen:
%
\begin{equation*}
	f : \mathbb{R}^{2} \supseteq [x_{min},x_{max}] \times [y_{min},y_{max}] \rightarrow [I_{min},I_{max}] \subseteq \mathbb{R} ,\quad (x,y) \mapsto f(x,y)
\end{equation*}
%
Für die Darstellung als Bilder sind somit lediglich geeignete Transformationen der Wertemengen der deflektometrischen Registrierungen \acrshort{lrx} und \acrshort{lry} nötig.
%
\begin{Definition}{Darstellung der Deflektometrischen Registrierung}{def:graphDeflektometrischeRegistrierung}
	Die deflektometrischen Registrierung \acrshort{lr} kann als zwei einzelne Bilder \acrshort{frx} und \acrshort{fry} dargestellt werden.
	%	
	\begin{equation*}
		\acrshortmath{frx} : \mathbb{R}^2 \supset \acrshortmath{d}(\acrshortmath{lrx}) \rightarrow [I_{min},I_{max}] \subseteq \mathbb{R}
	\end{equation*}
	%
	\begin{equation*}
		\acrshortmath{fry} : \mathbb{R}^2 \supset \acrshortmath{d}(\acrshortmath{lry}) \rightarrow [I_{min},I_{max}] \subseteq \mathbb{R}
	\end{equation*}
	%
	Dabei lassen sich die Bilder \acrshort{frx} und \acrshort{fry} schreiben als:
	%	
	\begin{equation*}
		\acrshortmath{frx}(x,y) = t_x(\acrshortmath{lrx}(x,y))
	\end{equation*}
	%	
	\begin{equation*}
		\acrshortmath{fry}(x,y) = t_y(\acrshortmath{lry}(x,y))
	\end{equation*}
	%
	Es gilt $\acrshortmath{d}(\acrshortmath{frx}) = \acrshortmath{d}(\acrshortmath{lrx})$ und $\acrshortmath{d}(\acrshortmath{fry}) = \acrshortmath{d}(\acrshortmath{lry})$.
	Die Abbildungen $t_x$ und $t_y$ sind dabei lineare Transformationen der Wertemengen der deflektometrischen Abbildungen in Spalten und Zeilen zu den zulässigen Intensitäten für die Bilder \acrshort{frx} und \acrshort{fry}, angegeben durch das Intervall $[I_{min},I_{max}]$.
	%
	\begin{equation*}
		t_x : \mathbb{R} \supset \acrshortmath{w}(\acrshortmath{lry}) \rightarrow [I_{min},I_{max}] \subseteq \mathbb{R}
	\end{equation*}
	%
	\begin{equation*}
		t_y : \mathbb{R} \supset \acrshortmath{w}(\acrshortmath{lry}) \rightarrow [I_{min},I_{max}] \subseteq \mathbb{R}
	\end{equation*}
	%
	Die Transformationen $t_x$ und $t_y$ lassen sich schreiben als:
	%	
	\begin{equation*}
		t_x(x) = \left(\dfrac{x}{\acrshortmath{lwidth}}(I_{max} - I_{min})\right) + I_{min}
	\end{equation*}
	%	
	\begin{equation*}
		t_y(y) = \left(\dfrac{y}{\acrshortmath{lheight}}(I_{max} - I_{min})\right) + I_{min}
	\end{equation*}
	%
\end{Definition}
%
Erstellt man aus der berechneten deflektometrischen Registrierung \acrshort{lr} einer ungekrümmten Fläche die zugehörigen Bilder \acrshort{frx} und \acrshort{fry} nach Definition \ref{def:graphDeflektometrischeRegistrierung}, erhält man Darstellungen wie in Abbildung \ref{tikz:abbOptimaleSpaltenZeilenReg}:

% Abbildung: Optimale Spalten- und Zeilenregistrierung
{
	\begin{figure}[H]
		\centering
		\begin{adjustbox}{width=\textwidth}
	\begin{tikzpicture}[every node/.style={inner sep=0,outer sep=0}]
	
		\node [anchor=north west] (imgSpalten) at (0,0) {\includegraphics[width=.47\textwidth]{04_deflektometrischeRegistrierung/auswertungDeflektometrischeRegistrierung/figures/spaltenRegistrierung_optimal}};
		\node [below=0.2cm of imgSpalten] {Graubild der Spaltenzuordnung \acrshort{frx}$(x,y)$};
		\node [anchor=north west] (imgZeilen) at (0.53\textwidth,0) {\includegraphics[width=.47\textwidth]{04_deflektometrischeRegistrierung/auswertungDeflektometrischeRegistrierung/figures/zeilenRegistrierung_optimal}};
		\node [below=0.2cm of imgZeilen] {Graubild der Zeilenzuordnung \acrshort{fry}$(x,y)$};
	
	\end{tikzpicture}
\end{adjustbox}
\caption[Darstellung Spalten- und Zeilenregistrierung]{Darstellung der Spalten- und Zeilenregistrierung als Bilder in Graustufen mit $I_{min} = 0$ und $I_{max} = 255$. Je dunkler ein Pixel ist, desto weiter links bzw. oben befindet sich die zugeordnete Spalten- bzw. Zeilenposition.}
		\label{tikz:abbOptimaleSpaltenZeilenReg}
	\end{figure}
}

\noindent
In Abbildung \ref{tikz:abbOptimaleSpaltenZeilenReg} wird direkt das Muster auf dem Monitor betrachtet.
Aus dem Grund lässt sich erkennen, dass die Zuordnung von Monitor- und Kamerapixeln in den Spalten und Zeilen linear verläuft.
Werden nun die Streifen durch besondere Oberflächeneigenschaften gekrümmt oder verzerrt, dann werden an diesen Stellen in den Bildern der deflektometrischen Registrierung Abweichungen vom linearen Grauwerteverlauf sichtbar.

% Abbildung: Verzerrung der Spalten- und Zeilenregistrierung durch Einkerbungen
{
	\begin{figure}[H]
		\centering
		\input{04_deflektometrischeRegistrierung/auswertungDeflektometrischeRegistrierung/figures/abbVerzerrungRegistrierung}
		\label{tikz:abbVerzerrungRegistrierung}
	\end{figure}
}

\noindent
%TODO Über die Porzellanobjekte schreiben.
%
Diese resultierenden Bilder können durch herkömmliche Verfahren aus der Bildverarbeitung weiterverarbeitet und analysiert werden.
Als hilfreich erweist sich dabei die Analyse von Gradientenbildern oder einer Hochpassfilterung zur Hervorhebung von lokalen Defekten\cite{kit_werling}.
}