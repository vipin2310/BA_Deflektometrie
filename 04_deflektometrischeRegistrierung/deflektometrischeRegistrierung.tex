\begin{Definition}{Deflektometrische Registrierung}{def:deflektometrischeRegistrierung}
	Die \textit{deflektometrische Registrierung} \acrshort{lr} beschreibt die Abbildung von Bildpunkten zu Schirmpunkten. \cite{kit_sbw}
	\begin{equation}
		\text{\acrshort{lr}} : A_{Cam} \rightarrow L \cup \varnothing ,~ P_{B} \rightarrow P_{L}
	\end{equation}
	$ A_{Cam} \subset \mathbb{R}^{2} $ bezeichnet die Menge der Bildpunkte bzw. der Kamerapixel.
	$ L \subset \mathbb{R}^{2} $ bezeichnet die Menge der Schirmpunkte bzw. der Monitorpixel.
	In Koordinatenschreibweise lässt sich die Abbildung somit darstellen als:
	\begin{equation} \label{eq:deflektometrischeRegistrierung}
		\text{\acrshort{lr}} : \mathbb{R}^{2} \supset A_{Cam} \rightarrow \mathbb{R}^2 ,~ (x_{B}, y_{B}) \mapsto (x_{L}, y_{L})
	\end{equation}
\end{Definition}

\noindent
Die deflektometrische Registrierung kann aus der Beobachtung des Monitors mithilfe der Kamera aufgestellt werden.
Wird der Monitor indirekt über eine spiegelnde Oberfläche beobachtet, können aus der deflektometrischen Registrierung Krümmungsmerkmale über die spiegelnde Oberfläche bestimmt werden.
Durch die Berücksichtigung der Positionen der Systemkomponenten können, wie im Abschnitt \ref{sub:rekonstruktion} bereits beschrieben, Höheninformationen über die Oberfläche gewonnen werden.