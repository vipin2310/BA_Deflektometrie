Durch die optische Besonderheit von spiegelnd glänzenden Oberflächen treten solche in der Industrie an vielen Stellen auf.
Speziell in der Automobilindustrie werden täglich große Karosserieflächen glänzend lackiert.
Alleine in Deutschland wurden in den Jahren von 1990 bis 2021 im Durchschnitt ungefähr 5 Millionen Personenkraftwagen pro Jahr produziert\cite{StatistaPKW}.
Ein großer Teil der Fahrzeuge erhalten nach der Lackierung eine spiegelnde Oberfläche.
Solche Oberflächen müssen durch besondere Verfahren auf Defekte überprüft werden.
Dabei sorgen spekulare Reflexionen dafür, dass die Oberflächen nicht direkt, sondern über ihre Spiegelbilder der Umgebung betrachtet werden. %TODO Beispielbilder

\p
Durch die riesige Menge an Oberflächen ist es für die Qualitätssicherung unumgänglich die Prüfung als automatisierten Prozess zu integrieren.
Die üblichen Verfahren der industriellen Bildverarbeitung stoßen dabei auf ihre Grenzen, sodass neue Methoden eingeführt werden müssen.
Diese speziellen Anwendungen erfordern den Einsatz von deflektometrischen Prüfaufbauten.
In der Industrie sind solche Verfahren schon seit Längerem zur Analyse der Topographie von spiegelnden Freiformflächen etabliert.
Deflektometrische Verfahren funktionieren nach einem ähnlichen Prinzip, wie auch die Inspektion von spiegelnden Oberflächen durch Menschen.
Das wissenschaftliche Gebiet der Deflektometrie ist auch heute noch Thema für viele Forschungsarbeiten und wird stetig weiterentwickelt.

\p
Welche Informationen können aus der Beobachtung von Spiegelbildern gewonnen werden?
Wie sehen allgemein anwendbare Methoden aus, um spiegelnde Oberflächen erfolgreich zu bewerten?
Das Ziel der Arbeit ist es diese Fragen zu erforschen und aufzuklären.