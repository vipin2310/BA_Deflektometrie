Im Rahmen der Arbeit werden spekular reflektierende Objektoberflächen unter Abbildung von bekannten Mustern durch eine Kamera aufgenommen, anschließend ausgewertet und auf Defekte überprüft.
Welche Informationen können aus der Beobachtung von Spiegelbildern gewonnen werden?
Wie sehen allgemein anwendbare Methoden aus, um spiegelnde Oberflächen qualitativ zu bewerten?
Das Ziel der Arbeit ist es, diese Fragen zu erforschen und aufzuklären.
Des Weiteren sollen ein Aufbau, die Ansteuerung von Beleuchtung und Kamera und die notwendige Auswertung des Bildmaterials entwickelt werden, durch welche eine Erkennung von Oberflächendefekten ermöglicht wird.
Die Umsetzung soll dabei in Form einer Softwareerweiterung, eines sogenannten \textit{Plug-ins}, für NeuroCheck erfolgen.

\p
Während der Arbeit soll außerdem auch ein bestimmter Sonderfall betrachtet werden - transparente Prüfobjekte.
Die Problematik ist dabei, dass man neben der Reflexion des Lichts mit der Transmission zu kämpfen hat.
Wie auch bei den regulären deflektometrischen Verfahren ist das Ziel, anhand einer verzerrten Szene Aussagen über die Oberflächenbeschaffenheit zu treffen.
Für die transparenten Objekte gibt es dafür verschiedene Lösungsansätze, um den negativen Einflüssen der Transmission entgegenzuwirken.