Die am weitest eingesetzte Kodierungsmethode in dem Kontext der Deflektometrie ist die Phasenkodierung.
Diese Verfahren werden in dem Themengebiet der \glqq Phasenmessende Deflektometrie\grqq ~beschrieben.
Dabei verwendet man Streifenmuster, die entlang der Ausbreitung der Streifen den Grauwerteverlauf einer Sinus-Funktion annehmen.
Solche Muster nennt man auch sinusoidale Streifenmuster.
Die Szene bzw. der Monitor wird dabei über die Phase der Sinus-Funktion kodiert.
Das heißt, jeder Punkt auf einem Monitor, angegeben durch eine $x$- und eine $y$-Koordinate, wird durch eine Phase $\phi_x$ in $x$-Richtung und eine Phase $\phi_y$ in $y$-Richtung kodiert.
Verwendet man Streifenmuster, stellt man die Kodierung in zwei Bildern dar.
Das erste Bild kodiert die Spaltenpositionen durch die Phasen $\phi_x$ und das zweite Bild kodiert die Zeilenpositionen $\phi_y$ (siehe Abbildung \ref{tikz:abbSinusoidaleStreifenmuster}).

\begin{figure}[H]
	\centering
		\begin{adjustbox}{width=\textwidth}
	\begin{tikzpicture}[every node/.style={inner sep=0,outer sep=0}]
	
		\node [anchor=north east] (imgSpalten) at (-0.03\textwidth,0) {\includegraphics[frame,width=.47\textwidth]{02_grundlagenDerDeflektometrie/rekonstruktion/phasenKodierung/figures/sinusoidalesXMuster}};
		\node [below=0.2cm of imgSpalten, align = center] {Sinusoidales Muster zur Kodierung \\ der Spalten};
		\node [anchor=north west] (imgZeilen) at (0.03\textwidth,0) {\includegraphics[frame,width=.47\textwidth]{02_grundlagenDerDeflektometrie/rekonstruktion/phasenKodierung/figures/sinusoidalesYMuster}};
		\node [below=0.2cm of imgZeilen, align = center] {Sinusoidales Muster zur Kodierung \\ der Zeilen};
		
	\end{tikzpicture}
\end{adjustbox}
\caption[Sinusoidale Streifenmuster]{Sinusoidale Streifenmuster zur Kodierung der Szene durch die Phasen $\left(\phi_x,\phi_y\right)$.}
		\label{tikz:abbSinusoidaleStreifenmuster}
\end{figure}

\noindent
Der Vorteil ist dabei die Kodierung durch die Grauwerte, die unabhängig von benachbarten Positionen dekodiert werden können.
Zur Dekodierung müssen aus den Grauwerten zunächst die Phasen bestimmt werden.
Dies funktioniert über ein sogenanntes Phasenschiebeverfahren \cite{carre}, bei dem weitere Bildaufnahmen mit Phasenverschiebungen der Sinus-Funktion vorgenommen werden.
Durch die Periodizität der Sinus-Funktion sind die bestimmten Phasen zunächst noch relativ zu den einzelnen Perioden angegeben.
In einem weiteren Schritt muss eine sogenannte Phasenentfaltung bzw. \glqq Unwrapping\grqq ~durchgeführt werden (siehe auch Definition \ref{def:phasenentfaltung}), damit die absoluten Phasen $\left(\phi_x,\phi_y\right)$ bestimmt werden.
Die Dekodierung über das \glqq Unwrapping\grqq ~erfolgt dabei durch die Verwendung von weiteren sinusoidalen Mustern mit unterschiedlicher Frequenz.
Diese Art der Kodierung erfordert damit mehrere Bilder.
Ein solches Verfahren wird im Kapitel \ref{sec:bestimmungDeflektometrischeRegistrierung} genauer beschrieben.

\p
Es sind damit zunächst mehrere Bildaufnahmen erforderlich.
Da solche Verfahren damit mehr Ressourcen verwenden, fokussieren sich einige Forschungsarbeiten die Anzahl der benötigten Muster zu reduzieren.
Der heutige technische Stand, ermöglicht es bereits, z. B. durch Überlagerung der Muster und weitere Optimierungen, diese Dekodierung durch eine einzige Kameraaufnahme umzusetzen.