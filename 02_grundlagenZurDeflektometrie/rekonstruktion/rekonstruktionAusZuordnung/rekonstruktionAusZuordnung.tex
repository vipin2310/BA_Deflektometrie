Ist erstmal die Zuordnung durch die Dekodierung erfolgt, kann man weitergehend die Oberfläche rekonstruieren.
Nimmt man zusätzlich die Informationen über den Systemaufbau hinzu bzw. führt eine Systemkalibrierung durch, kann aus der Zuordnung von Kamera- und Bildschirmpunkten der Reflexionswinkel $\theta$ aus Abbildung \ref{img:aufbau} berechnet werden.
Da bei einer Reflexion die Lichtstrahlen genau an der Oberflächennormalen gespiegelt werden, lässt sich dann in einem weiteren Schritt die Oberflächennormale in diesem Punkt bestimmen.
Führt man dies für jeden Punkt im Kamerabild durch, erhält man daraus die Neigungsinformationen des zu prüfenden Objektes in der Form eines Normalenfelds.

\p
Schließlich ist es möglich, aus diesem Normalenfeld räumliche Information der Oberfläche zu berechnen.
Dafür kann man zunächst aus den Normalenvektoren die zugehörigen Tangentialebenen berechnen, die über je zwei Richtungsvektoren definiert sind.
Diese Richtungsvektoren bilden die Tangentialfelder des Prüfobjekts.
Man kann über eine Integration der Tangentialfelder in ausgewählte Richtungen Kurven bestimmen, die in der Oberfläche des Objekts liegen.
Durch diese Integration erhält man einen Höhenzusammen\-hang der Oberflächenpunkte.
Wenn zusätzlich ein Oberflächenpunkt gegeben ist, kann man die dreidimensionalen Positionen der Oberflächenpunkte im Raum angeben \cite{kit_werling}.