Das Ziel dieser Arbeit war es, allgemein anwendbare Methoden zur Analyse von spiegelnden und transparenten Oberflächen zu untersuchen, auszuarbeiten und anschließend anzuwenden.
Hierzu wurden zu Beginn der vorliegenden Arbeit die theoretischen Grundlagen der Deflektometrie betrachtet.

\p
Innerhalb der Betrachtungen wurden die Ansätze und Möglichkeiten der Deflektometrie aufgezeigt.
Im Zentrum stand dabei, dass spiegelnde Oberflächen stets über ihre Umgebung wahrgenommen werden müssen und durch Spiegelungen geeigneter Szenen Informationen über die Krümmung gewonnen werden können.
Es wurden dabei konkrete Mustersequenzen erklärt, die auf dem heutigen Stand der Technik eingesetzt werden, um die Informationen aus der Szene kodiert in den Bildkanal zu übertragen.

\p
Nachdem die deflektometrischen Verfahren zur Rekonstruktion der Objektoberfläche mit ihren Schwierigkeiten und Problemen behandelt wurden, sollten Verfahren zur qualitativen Sichtprüfung dargestellt werden.
Diese Verfahren sind aufgrund weniger zu be\-rück\-sich\-ti\-gen\-der Parameter und Einschränkungen besser geeignet für die allgemeine Anwendung.
Durch das bessere Verständnis der Grundlagen konnten im Anschluss Ansätze aufgestellt und Verfahren entwickelt werden, mit denen die Problemstellungen aus der Einführung bewältigt werden können.
Es wurden zwei Verfahren im Detail beschrieben, um sowohl transparente Objekte und spiegelnde Oberflächen zu untersuchen:
%
\begin{itemize}
	\item Das erste Verfahren wurde in Kapitel \ref{chp:sichtpruefungDurchLichtstreuung} beschrieben. Es werden Streifenmuster verwendet, um die abweichende Lichtstreuung an Oberflächendefekten zu nutzen. Im Vergleich zur spiegelnden Oberfläche sind diese Stellen nicht spekular reflektierend. Durch bestimmte Verknüpfungsregeln von Bildern ist es damit möglich, die Stellen mit abweichender Lichtstreuung hervorzuheben und Fehlstellen zu erfassen.
	
	\item Das zweite Verfahren wurde in Kapitel \ref{chp:deflektometrischeRegistrierung} beschrieben. Es wird hierbei der Ansatz der Kodierung der Objektoberfläche mithilfe von sinusoidalen Streifenmustern verfolgt (vgl. Abschnitt \ref{sub:phasenKodierung}). Durch bestimmte Methoden ist es mithilfe von speziellen Mustersequenzen möglich, eine eindeutige Zuordnung von Kamerapunkten und Punkten auf einem Bildschirm bzw. der Szene anzugeben. Die Zuordnung wurde als deflektometrische Registrierung bezeichnet. Es wurde gezeigt, dass es möglich ist, herkömmliche Bildverarbeitungsalgorithmen anzuwenden, um Fehlstellen aus der deflektometrischen Registrierung zu erfassen (siehe Abschnitt \ref{sec:auswertungDeflektometrischeRegistrierung}).
\end{itemize}
%
Diese beiden Verfahren sind zur Anwendung für die Bildverarbeitungssoftware NeuroCheck implementiert worden.
Im Anschluss wurden die Ergebnisse der Arbeit an der Implementierung getestet.
Die Tests sollten dazu dienen, die Funktionalität der Verfahren und der Implementierung zu prüfen.
Außerdem sollten im Rahmen der Präsentation der Ergebnisse auch die beiden Verfahren für die Anwendung auf spiegelnde und transparente Oberflächen verglichen werden.
Bei den Anwendungen wurden zwei verschiedene Aufbauten gegeneinander getestet (siehe Abbildung \ref{tikz:abbAufbauFotos}).
%
\begin{itemize}
	\item Der erste Aufbau ermöglicht die Auswertung der Bilder durch eine Durchlichtszene. Dies ist nur für transparente Prüfobjekte umsetzbar, da das Objekt zwischen den Szenen (Bild eines LCD-Bildschirm) und der Kamera liegt. Die Szenen werden direkt durch das Prüfobjekt hindurch beobachtet und in der Kameraaufnahme durch die Objektkrümmungen verzerrt aufgenommen.
	
	\item Der zweite Aufbau ermöglicht die Auswertung der Bilder durch eine Spiegelbildszene. Dies lässt sich sowohl für spiegelnde, aber auch für transparente Prüfobjekte anwenden. Die Szene wird über die Spiegelung auf der Objektoberfläche beobachtet und in der Kameraaufnahme durch die Oberflächenkrümmungen verzerrt.
\end{itemize}
%
Bei den Tests ist deutlich geworden, dass sich das erste Verfahren (Sichtprüfung durch Lichtstreuung) für transparente Objekte gut eignet.
Der Rückseitenreflex der transparenten Objekte konnte verhindert werden, indem der Durchlichtaufbau verwendet wurde.
Mit der Spiegelbildauswertung für spiegelnde Oberflächen zeigte dieses Verfahren Schwächen bei stärkeren Krümmungen und unterschiedlichen Farben bzw. Beschriftungen auf den Objekten.
Für spiegelnde Objekte ließ sich feststellen, dass das zweite Verfahren (Deflektometrische Registrierung) präzisere Ergebnisse liefert hinsichtlich größerer Krümmungen oder unterschiedlichen Farben oder Beschriftungen auf einer Oberfläche.
Außerdem wird es durch die Ergebnisse des Verfahrens möglich, tatsächliche Krümmungen der Oberfläche mit einer bestimmten Genauigkeit zu erkennen.
Hierfür wurden die Richtungsableitung der deflektometrischen Registrierung hinzugezogen.

\p
Im Rahmen dieser Arbeit konnte gezeigt werden, dass für spiegelnde Objekte durch die Beobachtung von Spiegelbildern zuverlässig kleinere und größere Oberflächendefekte erfasst werden können.
Für transparente Objekte ist dies auch möglich, allerdings muss das Problem des Rückseitenreflexes dafür gelöst werden. 
Dies wurde durch die Verwendung des Durchlichtaufbaus erreicht, wodurch auch kleinere und größere Defekte in transparenten Objekten zuverlässig erfasst werden konnten.