Im Ausblick dieser Arbeit stehen mögliche Verbesserungen, die an den beschriebenen Verfahren vorgenommen werden können.

\p
Ein wichtiger Punkt für die Anwendbarkeit im industriellen Umfeld ist die Laufzeit der Verfahren.
Im Rahmen der Tests ist aufgefallen, dass der Aufnahmeprozess mit der Mustererzeugung, der Bildschirm- und der Kameraansteuerung ein zeitintensiver Vorgang während der Prüfung ist.
Betrachtet man die Anzahl nötiger Bildaufnahmen zur Bestimmung der deflektometrischen Registrierung, bemerkt man, dass dabei Optimierungsbedarf besteht.
Für die Ergebnisse der deflektometrischen Registrierung aus Kapitel \ref{chp:ergebnisse} wurden stets 12 Bildaufnahmen verwendet.
Dabei wurde aber nur die deflektometrische Registrierung der Zeilen betrachtet.
Würde man die vollständige deflektometrische Registrierung bestimmen, wären 24 Bildaufnahmen nötig gewesen.
Für präzisere Messergebnisse könnten noch weitere Bildaufnahmen hinzugezogen werden.
Zur Optimierung der Anzahl nötiger Bildaufnahmen könnte man eine Überlagerung der sinusoidalen Streifenmuster in $x$- und in $y$-Richtung durchführen.
Durch die Fourier-Analyse wäre es möglich, die überlagerten Muster im Kamerabild zu trennen und somit die Anzahl der benötigten Bildaufnahmen für eine vollständige deflektometrische Registrierung um die Hälfte zu reduzieren (vgl. \cite{singleShotPMD}).

\p
In Kapitel \ref{chp:ergebnisse} konnte erkannt werden, dass kleine Kratzer oder Laser-Gravuren nicht durch das Bild der deflektometrischen Registrierung sichtbar wurden.
Dies könnte sich durch Ungenauigkeiten bei der Bilderzeugung aus der deflektometrische Registrierung begründen lassen.
Die Darstellung der Zuordnung als Bild bewirkt, dass eventuell manche Zuordnungen auf denselben Grauwert fallen.
Das liegt an der mangelnden Anzahl an zu vergebenden Grauwerten.
Durch eine höhere Anzahl an differenzierbaren Grauwerten könnte es möglich sein, kleinere Oberflächendefekte zu erkennen.
Dies könnte z. B. durch die Verwendung einer Farbtiefe von 16-Bit erreicht werden.

\p
Die Deflektometrie bietet eine große Menge an Verfahren und wird durch stetige Weiterentwicklung immer genauer und performanter.
Aus dem Grund lassen sich in den beschriebenen Verfahren auch durchaus weitere Optimierungsmöglichkeiten finden.
Abschließend lässt sich sagen, dass die deflektometrischen Verfahren vielversprechende Perspektiven bereitstellen, um automatisiert spiegelnde oder transparente Oberflächen vollständig zu erfassen und zu prüfen.