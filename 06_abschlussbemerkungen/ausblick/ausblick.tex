Im Ausblick dieser Arbeit stehen mögliche Verbesserungen die an den beschriebenen hier Verfahren vorgenommen werden können.

\p
Ein wichtiger Punkt für die Anwendbarkeit im industriellen Umfeld ist die Laufzeit der Verfahren.
Betrachtet man die Anzahl nötiger Bildaufnahmen zur Bestimmung der deflektometrischen Registrierung, bemerkt man, dass noch Optimierungsbedarf besteht.
Für die Ergebnisse der deflektometrischen Registrierung aus Kapitel \ref{chp:ergebnisse} wurden stets 12 Bildaufnahmen verwendet.
Dabei wurde aber nur die deflektometrische Registrierung der Zeilen betrachtet.
Würde man die vollständige deflektometrische Registrierung bestimmen wollen, so wären dabei 24 Bildaufnahmen nötig gewesen.
Für genauere Messergebnisse können noch weitere Bildaufnahmen hinzugezogen werden.
Eine Überlagerung der sinusoidalen Streifenmuster mit anschließender Trennung im Kamerabild durch eine Fourier-Analyse würde es möglich machen die Anzahl der benötigten Bildaufnahmen für eine vollständige deflektometrische Registrierung um die Hälfte zu reduzieren (vgl. \cite{waveletPMD}).

\p
In Kapitel \ref{chp:ergebnisse} konnte gesehen werden, dass sehr kleine Kratzer oder Laser-Gravuren durch das Bild der deflektometrischen Registrierung nicht sichtbar wurde, dies könnte eine Ungenauigkeit aus der Bilderzeugung sein.
Die Darstellung der Zuordnung als Bild bewirkt, dass eventuell manche Zuordnungen auf denselben Grauwert fallen.
Dies liegt an der mangelnden Anzahl an zu vergebenden Grauwerten.
Durch eine höhere Anzahl an Grauwerten könnte es möglich sein kleinere Oberflächendefekte zu erkennen.
Dies könnte z. B. durch die Verwendung von 16-Bit Informationen für jeden Bildpunkt erreicht werden.

\p
Die Deflektometrie bietet eine große Menge an Verfahren und wird durch stetiger Weiterentwicklung immer genauer und performanter.
Aus dem Grund lassen sich in den beschriebenen Verfahren auch durchaus viele weitere Optimierungsmöglichkeiten finden.
Abschließend lässt sich sagen, dass die deflektometrischen Verfahren vielversprechende Perspektiven bereitstellen um automatisiert spiegelnde oder transparente Oberflächen vollständig zu erfassen und zu prüfen.