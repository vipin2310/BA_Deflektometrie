\noindent
Das Ziel der vorliegenden Arbeit ist es, Verfahren zur Sichtprüfung von spiegelnden und transparenten Prüfoberflächen einzuführen und mathematisch zu erklären.
Dabei wird erklärt, welche Methoden im wissenschaftlichen Gebiet der Deflektometrie eingesetzt werden, um Oberflächen vollständig zu erfassen.
Zur Bearbeitung des Themas werden transparente Brillengläser und spiegelnde Keramikobjekte analysiert und mit den eingeführten Verfahren automatisiert ausgewertet.
Die Ergebnisse der Auswertung durch die einführten Verfahren zeigten, dass es möglich ist, Anomalien der Oberflächenkrümmung spiegelnder und transparenter Prüfobjekte, wie z. B. Pickel, Dellen, Kratzer oder Gravuren, zu erfassen.
Dadurch wird es ermöglicht, Oberflächendefekte spiegelnder und transparenter Prüfobjekte zu lokalisieren und qualitative Aussagen der Oberflächenbeschaffenheit zu formulieren.