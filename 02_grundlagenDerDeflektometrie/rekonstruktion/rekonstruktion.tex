Das Hauptforschungsgebiet der gegenwärtigen Deflektometrie ist die Generierung von dreidimensionalen Modellen von spiegelnden Objektoberflächen.
Der Aufbau eines solchen Anwendungsfalls sieht eine Beleuchtungseinheit (z. B. einen Bildschirm), einen Sensor (z. B. eine Kamera) und ein zu untersuchendes Objekt vor.

\begin{figure}[H]
	\centering
	\includegraphics[width=0.7\textwidth]{02_grundlagenDerDeflektometrie/rekonstruktion/figures/nature-articel-nr1}
	\caption[Aufbau einer Deflektometrie-Prüfstation]{Aufbau einer Deflektometrie-Prüfstation. \textit{in Anlehnung an} \cite{aufbau}}
	\label{img:aufbau}
\end{figure}

\noindent
Wie in Abbildung \ref{img:aufbau} angedeutet, wird ein Muster als bekannte Szene auf ein Prüfobjekt abgebildet und anschließend von einer Kamera aufgenommen.
Das grundlegende Prinzip basiert darauf, dass jeder durch die Kamera aufgenommene Punkt des Objekts dem Punkt auf dem Bildschirm zugeordnet wird, der den Objektpunkt beleuchtet bzw. der am Objektpunkt in die Kamera reflektiert wird.
Dabei ordnet man jedem Pixel des projizierten Musters sein zugehöriges Pixel des erzeugten Musters auf dem Bildschirm zu.
Durch diese Zuordnung von Kamera- und Bildschirmpunkten lassen sich Neigungsinformationen der Oberfläche berechnen.
Dies kann durch Strahlenverfolgungen erreicht werden.
In Abbildung \ref{img:aufbau} lässt sich das über die in Rot eingezeichneten Vektoren erkennen.
Die Schwierigkeit liegt in der eindeutigen Zuordnung zwischen der Szene und dem Spiegelbild.
Hierfür gibt es verschiedene Ansätze, dies zu erreichen.
Grundlegend ist dabei die Kodierung der Objektoberfläche (siehe Definition \ref{def:kodierungOberflaeche}), damit diese durch die Kamera aufgenommen werden kann.
Die Kamera digitalisiert dann die kodierte Oberfläche zu einem oder mehreren Bildern.
Unter Berücksichtigung der Kodierung können die Bilder durch einen entsprechenden Softwarealgorithmus dekodiert werden und man erhält damit die Zuordnung zwischen der Szene und dem Spiegelbild (siehe Abbildung \ref{tikz:abbKodierungUndDekodierung}).
%
% Definition: Kodierung der Objektoberfläche
\begin{Definition}{Kodierung der Objektoberfläche}{def:kodierungOberflaeche}
	Die Abbildung einer oder mehrerer vordefinierter Szenen auf eine spiegelnde Oberfläche wird als \textit{Kodierung der Objektoberfläche} bezeichnet.
	Das Ziel ist es dabei die Punkte aus der Szene eindeutig durch eine oder mehrere Aufnahmen der Oberfläche zu identifizieren.
\end{Definition}
%
%
% Abbildung: Kodierung und Dekodierung
{
	\begin{figure}[H]
		\centering
		\begin{adjustbox}{width=\textwidth}
	\begin{tikzpicture}
	[
		rectnode/.style={rectangle, draw=red!60, fill=red!5, very thick, minimum width={3cm}, font=\small, rounded corners},
		ellipsenode/.style={ellipse, draw=green!60, fill=green!5, very thick, minimum width={3cm}, font=\small},
		calloutnode/.style={rectangle callout, draw=RoyalPurple, fill=RoyalPurple!5, text=RoyalPurple, font=\footnotesize, rounded corners},
	]

		\node[ellipsenode] (source) {Quelle};
		\node[rectnode] (encoder) [right=of source] {Kodierung};
		\node[rectnode] (channel) [below right=of encoder] {Digitalisierung};
		\node[rectnode] (decoder) [below left=of channel] {Dekodierung};
		\node[ellipsenode] (drain) [left=of decoder]{Ergebnis};		
		
		\draw[->] (source.east) -- (encoder.west);
		\draw[->] (encoder.east) -| (channel.north);
		\draw[->] (channel.south) |- (decoder.east);
		\draw[->] (decoder.west) -- (drain.east);
		
		\node[right=of channel, font=\small	] (noise) {Rauschen};
		\draw[->] (noise.west) -- (channel.east);
		
		\node[calloutnode] (surface) [callout absolute pointer=(source.north), above=of source.east, anchor=east] {Objektoberfläche};
		\node[calloutnode] (illumination) [callout absolute pointer=(encoder.north), above=of encoder.east, anchor=east] {Beleuchtung};
		\node[calloutnode] (camera) [callout absolute pointer=(channel.30), above=of channel.east, anchor=east] {Kamera};
		\node[calloutnode] (software) [callout absolute pointer=(decoder.north), above=of decoder.east, anchor=east] {Algorithmus};
		\node[calloutnode] (result) [callout absolute pointer=(drain.north), above=of drain.east, anchor=east] {Zuordnung};
		
	\end{tikzpicture}
\end{adjustbox}
\caption[Kodierung und Dekodierung der Objektoberfläche]{Kodierung und Dekodierung der Objektoberfläche.}
		\label{tikz:abbKodierungUndDekodierung}
	\end{figure}
}

\noindent
Die Art, wie man die Informationen in den digitalen Kanal überträgt, ist entscheidend für eine gute Zuordnung.
Aus dem Grund werden sich in wissenschaftlichen Forschungsarbeiten einige Gedanken über die Kodierung der Objektoberfläche gemacht.
Auf eine Auswahl von Möglichkeiten aus dem heutigen wissenschaftlichen Stand soll im Folgenden eingegangen werden.
%
% Phasenkodierung
{
	\FloatBarrier
    \subsection{Phasenkodierung}
    \label{sub:phasenKodierung}
    Die am weitest eingesetzte Kodierungsmethode in dem Kontext der Deflektometrie ist die Phasenkodierung.
Diese Verfahren werden in dem Themengebiet der \glqq Phasenmessende Deflektometrie\grqq ~beschrieben.
Dabei verwendet man Streifenmuster, die entlang der Ausbreitung der Streifen den Grauwerteverlauf einer Sinus-Funktion annehmen.
Solche Muster nennt man auch sinusoidale Streifenmuster.
Die Szene bzw. der Monitor wird dabei über die Phase der Sinus-Funktion kodiert.
Das heißt, jeder Punkt auf einem Monitor, angegeben durch eine $x$- und eine $y$-Koordinate, wird durch eine Phase $\phi_x$ in $x$-Richtung und eine Phase $\phi_y$ in $y$-Richtung kodiert.
Verwendet man Streifenmuster, stellt man die Kodierung in zwei Bildern dar.
Das erste Bild kodiert die Spaltenpositionen durch die Phasen $\phi_x$ und das zweite Bild kodiert die Zeilenpositionen $\phi_y$ (siehe Abbildung \ref{tikz:abbSinusoidaleStreifenmuster}).

\begin{figure}[H]
	\centering
		\begin{adjustbox}{width=\textwidth}
	\begin{tikzpicture}[every node/.style={inner sep=0,outer sep=0}]
	
		\node [anchor=north east] (imgSpalten) at (-0.03\textwidth,0) {\includegraphics[frame,width=.47\textwidth]{02_grundlagenDerDeflektometrie/rekonstruktion/phasenKodierung/figures/sinusoidalesXMuster}};
		\node [below=0.2cm of imgSpalten, align = center] {Sinusoidales Muster zur Kodierung \\ der Spalten};
		\node [anchor=north west] (imgZeilen) at (0.03\textwidth,0) {\includegraphics[frame,width=.47\textwidth]{02_grundlagenDerDeflektometrie/rekonstruktion/phasenKodierung/figures/sinusoidalesYMuster}};
		\node [below=0.2cm of imgZeilen, align = center] {Sinusoidales Muster zur Kodierung \\ der Zeilen};
		
	\end{tikzpicture}
\end{adjustbox}
\caption[Sinusoidale Streifenmuster]{Sinusoidale Streifenmuster zur Kodierung der Szene durch die Phasen $\left(\phi_x,\phi_y\right)$.}
		\label{tikz:abbSinusoidaleStreifenmuster}
\end{figure}

\noindent
Der Vorteil ist dabei die Kodierung durch die Grauwerte, die unabhängig von benachbarten Positionen dekodiert werden können.
Zur Dekodierung müssen aus den Grauwerten zunächst die Phasen bestimmt werden.
Dies funktioniert über ein sogenanntes Phasenschiebeverfahren \cite{carre}, bei dem weitere Bildaufnahmen mit Phasenverschiebungen der Sinus-Funktion vorgenommen werden.
Durch die Periodizität der Sinus-Funktion sind die bestimmten Phasen zunächst noch relativ zu den einzelnen Perioden angegeben.
In einem weiteren Schritt muss eine sogenannte Phasenentfaltung bzw. \glqq Unwrapping\grqq ~durchgeführt werden (siehe auch Definition \ref{def:phasenentfaltung}), damit die absoluten Phasen $\left(\phi_x,\phi_y\right)$ bestimmt werden.
Die Dekodierung über das \glqq Unwrapping\grqq ~erfolgt dabei durch die Verwendung von weiteren sinusoidalen Mustern mit unterschiedlicher Frequenz.
Diese Art der Kodierung erfordert damit mehrere Bilder.
Ein solches Verfahren wird im Kapitel \ref{sec:bestimmungDeflektometrischeRegistrierung} genauer beschrieben.

\p
Es sind damit zunächst mehrere Bildaufnahmen erforderlich.
Da solche Verfahren damit mehr Ressourcen verwenden, fokussieren sich einige Forschungsarbeiten die Anzahl der benötigten Muster zu reduzieren.
Der heutige technische Stand, ermöglicht es bereits, z. B. durch Überlagerung der Muster und weitere Optimierungen, diese Dekodierung durch eine einzige Kameraaufnahme umzusetzen.
}
%
% Frequenzkodierung
{
	\FloatBarrier
    \subsection{Frequenzkodierung}
    \label{sub:frequenzKodierung}
    Als Alternative zur Phasenkodierung kann man die Sinus-Funktion auch nutzen um die Ortskoordinaten des Bildschirms über Frequenzen zu kodieren.
Hierfür eignet sich die Darstellung der Koordinaten in der Polarform.
%
\begin{equation*}
	\left(x,y\right) \mapsto \left(r,\phi\right)
\end{equation*}
%
Wenn man im Folgenden den Radius $r$ einer speziellen Frequenz und die Phase $\phi$ als Phasenverschiebung einer Sinus-Funktion zuweist, erhält man zeitabhängige Muster.
So könnte zum Beispiel durch
%
\begin{equation*}
	f_t \left(r,\phi\right) = 1 + \sin \left(2 \pi r t + \phi \right)
\end{equation*}
%
die Kodierung der Szene in Abhängigkeit der Zeit $t$ angegeben sein.
In Abbildung \ref{tikz:abbFrequenzkodierteMuster} wird eine Kodierung dieser Art zu bestimmten Zeitpunkten abgebildet.
%
\begin{figure}[H]
	\centering
	\begin{adjustbox}{width=\textwidth}
	\begin{tikzpicture}[every node/.style={inner sep=0,outer sep=0}]
	
		\node [anchor=north east] (freq0) at (-0.03\textwidth,0) {\includegraphics[frame,width=.47\textwidth]{02_grundlagenZurDeflektometrie/rekonstruktion/frequenzKodierung/figures/frequenzKodiertT0}};
		\node [below=0.2cm of freq0] (cap0) {Muster zum Zeitpunkt $t = 0$};
		\node [anchor=north west] (freq1) at (0.03\textwidth,0) {\includegraphics[frame,width=.47\textwidth]{02_grundlagenZurDeflektometrie/rekonstruktion/frequenzKodierung/figures/frequenzKodiertT1}};
		\node [below=0.2cm of freq1] (cap1) {Muster zum Zeitpunkt $t = 1$};
		
		\node [below=1cm of freq0.south east, anchor=north east] (freq4) {\includegraphics[frame,width=.47\textwidth]{02_grundlagenZurDeflektometrie/rekonstruktion/frequenzKodierung/figures/frequenzKodiertT4}};
		\node [below=0.2cm of freq4] {Muster zum Zeitpunkt $t = 4$};
		\node [below=1cm of freq1.south west, anchor=north west] (freq10) {\includegraphics[frame,width=.47\textwidth]{02_grundlagenZurDeflektometrie/rekonstruktion/frequenzKodierung/figures/frequenzKodiertT10}};
		\node [below=0.2cm of freq10] {Muster zum Zeitpunkt $t = 10$};
		
	\end{tikzpicture}
\end{adjustbox}
\caption[Muster der Frequenzkodierung]{Muster der Frequenzkodierung der Szene zu festen Zeitpunkten $t$.}
	\label{tikz:abbFrequenzkodierteMuster}
\end{figure}
%
\noindent
Zur Dekodierung muss hier über eine gewisse Zeit die Frequenz der einzelnen Bildpunkte gemessen werden.
Zusätzlich muss der Phasenwinkel bestimmt werden.
Dafür kann ein ähnliches Verfahren wie auch bei der Bestimmung der Phase im vorhergehenden Abschnitt \ref{sub:phasenKodierung} eingesetzt werden.
Zur Optimierung des Rechenaufwands ist es auch möglich, jeder Ortskoordinate eine eigene Frequenz zuzuweisen, damit man sich die Bestimmung der Phase sparen kann.

\p
Da in diesem Kodierverfahren nicht mehr die Grauwerte des Bildes selbst den Ort kodieren, ist es unempfindlich gegenüber Nichtlinearitäten in den Anzeige- oder Aufnahmefarben.
Außerdem lassen sich mehrere Signale überlagern und genau durch die Frequenz trennen.
Ein großer Nachteil im Vergleich zur Phasenkodierung ist allerdings die lange Messzeit der Frequenz \cite{jenaerOK}.
}
%
% Stochastische Kodierung
{
	\FloatBarrier
    \subsection{Stochastische Kodierung}
    \label{sub:stochastischeKodierung}
    Die stochastische Kodierung nutzt ein zufällig generiertes Muster als Szene.
Geeignet für dieses Verfahren sind sogenannte \glqq Specklemuster\grqq ~\cite{specklePattern}.
Es handelt sich dabei um bandbegrenzte Muster mit zufällig verteilten Grauwerten.
Aufgrund der Unschärfe und dem Rauschen, welche in der Aufnahme durch eine Kamera einfließen, ist die Bandbegrenztheit notwendig um eine Dekodierung zu ermöglichen.
Abbildung \ref{img:speckleMuster} zeigt ein solches Muster.
%
\begin{figure}[H]
	\centering
	\includegraphics[frame,width=0.5\textwidth]{02_grundlagenZurDeflektometrie/rekonstruktion/stochastischeKodierung/figures/speckleMuster}
	\caption[Specklemuster]{Specklemuster.}
	\label{img:speckleMuster}
\end{figure}
%
\noindent
Das Grundprinzip der Dekodierung für dieses Kodierverfahren ist das Verfolgen der Punkte beim Verschieben des Specklemusters.
Es wird zunächst die Spiegelung des Specklemusters aufgenommen und ein Referenzpunkt definiert.
Nach der Verschiebung des Specklemusters (z. B. in x-Richtung) wird dieser Referenzpunkt in dem zweiten Bild über seine Umgebung gesucht.
Dabei verwendet man einen Algorithmus zur Berechnung der zweidimensionalen Bildkorrelation, die die höchste Übereinstimmung mit der Umgebung sucht.
Wenn der verschobene Referenzpunkt gefunden wurde, wird an derselben Stelle der Punkt im ersten Bild markiert.
Der markierte Punkt ist dann der neue Referenzpunkt, der im zweiten Bild gesucht wird.
Dieser Vorgang wiederholt sich, solange die Referenzpunkte im Bild liegen.
Abbildung \ref{img:prinzipStoKodierung} zeigt schematisch das Verfahren zur Dekodierung.
%
\begin{figure}[H]
	\centering
	\includegraphics[width=0.9\textwidth]{02_grundlagenZurDeflektometrie/rekonstruktion/stochastischeKodierung/figures/Prinzip}
	\caption[Prinzip der Zuordnung einer stochastischen Kodierung]{Prinzip der Zuordnung einer stochastischen Kodierung. \textit{in Anlehnung an} \cite{specklePattern}}
	\label{img:prinzipStoKodierung}
\end{figure}
%
\noindent
Führt man dasselbe auch für die andere Richtung (z. B. in y-Richtung) durch, erhält man ein Raster auf dem Objekt, an dem die Objektpunkte zugeordnet wurden (siehe Abbildung \ref{img:ergebnisStoKodierung}).
%
\begin{figure}[H]
	\centering
	\includegraphics[width=0.4\textwidth]{02_grundlagenZurDeflektometrie/rekonstruktion/stochastischeKodierung/figures/Ergebnis}
	\caption[Ergebnis der Zuordnung einer stochastischen Kodierung]{Ergebnis der Zuordnung einer stochastischen Kodierung. Die grünen Pfeile zeigen an, von welchem Referenzpunkt man auf die nächste Zuordnung kam. \textit{in Anlehnung an} \cite{specklePattern}}
	\label{img:ergebnisStoKodierung}
\end{figure}
%
\noindent
Im Vergleich zu den vorgestellten Kodierverfahren in den vorherigen Abschnitten \ref{sub:phasenKodierung} und \ref{sub:frequenzKodierung}, liefert die Dekodierung in diesem Fall nur eine begrenzte Auflösung und keine vollflächige Zuordnung.
Außerdem ist dieses Verfahren nicht anwendbar für Objekte, die eine große Verzerrung der Musters erzeugen, da die zweidimensionalen Bildkorrelation ansonsten die Referenzpunkte nicht finden kann.
Dennoch sind die Vorteile dieses Verfahren der geringe Rechenaufwand und die Möglichkeit durch drei Bilder eine erfolgreiche Dekodierung durchzuführen \cite{specklePattern}.
}
%
% Rekonstruktion der Oberfläche und Regularisierungsproblem
{
	\FloatBarrier
    \subsection{Rekonstruktion der Oberfläche und Regularisierungsproblem}
    \label{sub:rekonstruktionUndRegularisierungsproblem}
    Durch das Vorgehen nach Abbildung \ref{tikz:abbKodierungUndDekodierung} und die beschriebenen Kodiermöglichkeiten kann die Zuordnung der Kamerapunkte und der Bildschirmpunkte erfolgen.
Mithilfe weiterer Schritte kann man daraus die Oberfläche rekonstruieren.
%
\begin{figure}[H]
	\centering
	\includegraphics[width=0.7\textwidth]{02_grundlagenZurDeflektometrie/rekonstruktion/rekonstruktionUndRegularisierungsproblem/figures/regularisierungsproblem}
	\caption[Regularisierungsproblem]{Regularisierungsproblem. \textit{in Anlehnung an} \cite{stereoDeflektometrie}}
	\label{img:regularisierungsproblem}
\end{figure}
%
\noindent
Abbildung \ref{img:regularisierungsproblem} zeigt die Strahlenverfolgung zur Bestimmung der Oberflächennormalen $n$.
Zunächst benötigt man neben der Zuordnung zusätzliche Informationen über den Systemaufbau.
Das umfasst die Positionen und Ausrichtungen der Kamera und des Monitors im Raum.
Dadurch ist der Vektor $m$, sowie die Richtung des Sichtvektors $u$ bestimmt.
Man weiß zwar welcher Kamerapunkt welchen Punkt des Bildschirms sieht, allerdings ist dadurch das optische System nicht ausreichend beschrieben um die Länge des Sichtvektors $u$ anzugeben.
Es fehlt die Lage der Oberfläche.
Wäre diese bekannt, könnte der Reflexionsvektor $v$ mit 
\begin{equation*}
	v = m - u
\end{equation*}
bestimmt werden.
Mithilfe des Reflexionsgesetzes kann man aus dem Reflexionsvektor $v$ und dem Sichtvektor $u$ den Normalenvektor $n$ bestimmen:
%
\begin{equation}
	n = \dfrac{v - u}{\left\Vert v - u \right\Vert}	
\end{equation}
%
Durch die Unbestimmtheit des Sichtvektors $u$, erhält man entlang seiner Richtung unendlich viele potentielle Normalenvektoren $n$ des beobachteten Oberflächenpunkts.
Das heißt, es gibt unendlich viele mögliche Positionen des Oberflächenpunkts entlang der Sichtrichtung.
Diese Mehrdeutigkeit wird als Regularisierungs- oder Deflektometrieproblem bezeichnet \cite{regularisierungsproblem}.

%TODO Regularisierungsverfahren beschreiben? z. B. Stereo-Verfahren
\p
Schließlich ist es möglich, aus diesem Normalenfeld die räumlichen Informationen der Oberfläche zu berechnen.
Dafür kann man zunächst aus den Normalenvektoren die zugehörigen Tangentialebenen berechnen, die über je zwei Richtungsvektoren definiert sind.
Diese Richtungsvektoren bilden die Tangentialfelder des Prüfobjekts.
Man kann über eine Integration der Tangentialfelder in ausgewählte Richtungen Kurven bestimmen, die in der Oberfläche des Objekts liegen.
Durch diese Integration erhält man einen Höhenzusammen\-hang der Oberflächenpunkte.
Wenn zusätzlich ein Oberflächenpunkt gegeben ist, kann man die dreidimensionalen Positionen der Oberflächenpunkte im Raum angeben \cite{kit_werling}.
}
%
