Als Alternative zur Phasenkodierung kann man die Sinus-Funktion auch nutzen um die Ortskoordinaten des Bildschirms über Frequenzen zu kodieren.
Hierfür eignet sich die Darstellung der Koordinaten in der Polarform.
%
\begin{equation*}
	\left(x,y\right) \mapsto \left(r,\phi\right)
\end{equation*}
%
Wenn man im Folgenden den Radius $r$ einer speziellen Frequenz und die Phase $\phi$ als Phasenverschiebung einer Sinus-Funktion zuweist, erhält man zeitabhängige Muster.
So könnte zum Beispiel durch
%
\begin{equation*}
	f_t \left(r,\phi\right) = 1 + \sin \left(2 \pi r t + \phi \right)
\end{equation*}
%
die Kodierung der Szene in Abhängigkeit der Zeit $t$ angegeben sein.
In Abbildung \ref{tikz:abbFrequenzkodierteMuster} wird eine Kodierung dieser Art zu bestimmten Zeitpunkten abgebildet.
%
\begin{figure}[H]
	\centering
	\begin{adjustbox}{width=\textwidth}
	\begin{tikzpicture}[every node/.style={inner sep=0,outer sep=0}]
	
		\node [anchor=north east] (freq0) at (-0.03\textwidth,0) {\includegraphics[frame,width=.47\textwidth]{02_grundlagenZurDeflektometrie/rekonstruktion/frequenzKodierung/figures/frequenzKodiertT0}};
		\node [below=0.2cm of freq0] (cap0) {Muster zum Zeitpunkt $t = 0$};
		\node [anchor=north west] (freq1) at (0.03\textwidth,0) {\includegraphics[frame,width=.47\textwidth]{02_grundlagenZurDeflektometrie/rekonstruktion/frequenzKodierung/figures/frequenzKodiertT1}};
		\node [below=0.2cm of freq1] (cap1) {Muster zum Zeitpunkt $t = 1$};
		
		\node [below=1cm of freq0.south east, anchor=north east] (freq4) {\includegraphics[frame,width=.47\textwidth]{02_grundlagenZurDeflektometrie/rekonstruktion/frequenzKodierung/figures/frequenzKodiertT4}};
		\node [below=0.2cm of freq4] {Muster zum Zeitpunkt $t = 4$};
		\node [below=1cm of freq1.south west, anchor=north west] (freq10) {\includegraphics[frame,width=.47\textwidth]{02_grundlagenZurDeflektometrie/rekonstruktion/frequenzKodierung/figures/frequenzKodiertT10}};
		\node [below=0.2cm of freq10] {Muster zum Zeitpunkt $t = 10$};
		
	\end{tikzpicture}
\end{adjustbox}
\caption[Muster der Frequenzkodierung]{Muster der Frequenzkodierung der Szene zu festen Zeitpunkten $t$.}
	\label{tikz:abbFrequenzkodierteMuster}
\end{figure}
%
\noindent
Zur Dekodierung muss hier über eine gewisse Zeit die Frequenz der einzelnen Bildpunkte gemessen werden.
Zusätzlich muss der Phasenwinkel bestimmt werden.
Dafür kann ein ähnliches Verfahren wie auch bei der Bestimmung der Phase im vorhergehenden Abschnitt \ref{sub:phasenKodierung} eingesetzt werden.
Zur Optimierung des Rechenaufwands ist es auch möglich, jeder Ortskoordinate eine eigene Frequenz zuzuweisen, damit man sich die Bestimmung der Phase sparen kann.

\p
Da in diesem Kodierverfahren nicht mehr die Grauwerte des Bildes selbst den Ort kodieren, ist es unempfindlich gegenüber Nichtlinearitäten in den Anzeige- oder Aufnahmefarben.
Außerdem lassen sich mehrere Signale überlagern und genau durch die Frequenz trennen.
Ein großer Nachteil im Vergleich zur Phasenkodierung ist allerdings die lange Messzeit der Frequenz \cite{jenaerOK}.