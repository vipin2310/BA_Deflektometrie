\begin{tikzpicture}[every node/.style={inner sep=0,outer sep=0}]

	\node [anchor=north east] (img1) at (-0.03\textwidth,0) {\includegraphics[width=.4\textwidth]{05_ergebnisse/ergDiskussion/figures/lasergravur}};
	\node [below=0.2cm of img1, align=center] {Ausschnitt von Brillenglas 1 \\ aus Abbildung \ref{tikz:abbNachbearbeitung}};
	\node [anchor=north west] (img2) at (0.03\textwidth,0) {\includegraphics[width=.4\textwidth]{05_ergebnisse/ergDiskussion/figures/polierfehler}};
	\node [below=0.2cm of img2, align=center] {Ausschnitt von Brillenglas 2 \\ aus Abbildung \ref{tikz:abbNachbearbeitung}};
	
\end{tikzpicture}
\caption[Erkennbare kleine Defekte oder Laser-Gravur]{Erkennbare kleine Defekte oder Laser-Gravuren durch Durchlichtauswertung mit Verfahren \glqq Sichtprüfung durch Lichtstreuung\grqq ~(siehe Kapitel \ref{chp:sichtpruefungDurchLichtstreuung}). Linkes Teilbild: Erkennbare Lasergravur, Rechtes Teilbild: Leichte Kratzer durch Fehler bei der Polierung}