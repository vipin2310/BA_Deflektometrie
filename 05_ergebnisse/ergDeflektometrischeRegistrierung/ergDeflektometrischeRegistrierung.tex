Das Verfahren zur Bestimmung der \glqq deflektometrischen Registrierung\grqq ~wurde in Kapitel \ref{chp:deflektometrischeRegistrierung} eingeführt und bestimmt eine Zuordnung zwischen Bildschirm- und Kamerapunkten.
Nach Abschnitt \ref{sec:auswertungDeflektometrischeRegistrierung} lässt sich diese Zuordnung auswerten und als Bilder darstellen.
%TODO mehr Allgemeines über das Verfahren? -> Ich denke nicht nötig
Die Parameter zur Erzeugung der Muster nach Gleichung \ref{eq:monitormuster_mehrstufig} können in Tabelle \ref{tab:paramSichtpruefung} abgelesen werden.

\begin{table}[H]
	\centering
	\begin{tabular}{lll}
	\hline 
	\textbf{Beschreibung} & \textbf{Name} & \textbf{Wert} \\ 
	\hline 
	Amplitude (beeinflusst die Helligkeit) & $A_m$ & 127.5 \\
	Kontrast & $C_m$ & 1.0 \\
	Anzahl Perioden des ersten Muster & $N_p^1$ & 3 \\ 
	Anzahl Perioden des zweiten Muster & $N_p^2$ & 5 \\ 
	Anzahl Perioden des dritten Muster & $N_p^3$ & 7 \\  
	Anzahl Phasenverschiebungen jedes Musters & $N_{shift}$ & 4 \\ 
	Bildschirmbreite (in Pixel) & \acrshort{lwidth} & 1920 \\
	Bildschirmhöhe (in Pixel) & \acrshort{lheight} & 1080 \\
	\hline 
	\end{tabular} 
	\caption{Parameter des Verfahrens}
	\label{tab:paramDeflektometrischRegistrierung}
\end{table}

\noindent
Durch diese Parameter erhält man eine Sequenz aus 12 Mustern die auf einem LCD-Bildschirm angezeigt und anschließend durch eine Kamera aufgenommen werden können.
Zur Kodierung der Oberfläche werden insgesamt 256 Helligkeitsstufen auf die Oberfläche abgebildet.
Zur Verbesserung der Ergebnisse werden die verwendeten Aufbauten durch geeignete Abschirmungen vor Fremdlicht geschützt.