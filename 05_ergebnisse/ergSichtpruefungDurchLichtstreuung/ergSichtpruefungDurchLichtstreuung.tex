Das Verfahren \glqq Sichtprüfung durch Lichtstreuung\grqq ~wurde in Kapitel \ref{chp:sichtpruefungDurchLichtstreuung} eingeführt und lässt kleine Defekte auf spiegelnden Oberflächen sichtbar werden.

\begin{table}[H]
	\centering
	\label{tab:paramSichtpruefung}
	\begin{tabular}{lll}
		\hline 
		\textbf{Beschreibung} & \textbf{Name} & \textbf{Wert} \\ 
		\hline 
		Tastgrad (beeinflusst die Streifenbreiten) & $D$ & $\tfrac{2}{5}$ \\ 
		Amplitude (beeinflusst die Helligkeit) & $A_m$ & 127.5 \\ 
		Kontrast & $C_m$ & 1.0 \\ 
		Anzahl Phasenverschiebungen & $N_{shift}$ & 16 \\ 
		Bildschirmbreite (in Pixel) & \acrshort{lwidth} & 1920 \\
		Bildschirmhöhe (in Pixel) & \acrshort{lheight} & 1080 \\ 
		\hline 
	\end{tabular}
	\caption{Parameter des Verfahrens}
\end{table}